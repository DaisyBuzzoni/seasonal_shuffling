\PassOptionsToPackage{unicode=true}{hyperref} % options for packages loaded elsewhere
\PassOptionsToPackage{hyphens}{url}
%
\documentclass[]{article}
\usepackage{lmodern}
\usepackage{amssymb,amsmath}
\usepackage{ifxetex,ifluatex}
\usepackage{fixltx2e} % provides \textsubscript
\ifnum 0\ifxetex 1\fi\ifluatex 1\fi=0 % if pdftex
  \usepackage[T1]{fontenc}
  \usepackage[utf8]{inputenc}
  \usepackage{textcomp} % provides euro and other symbols
\else % if luatex or xelatex
  \usepackage{unicode-math}
  \defaultfontfeatures{Ligatures=TeX,Scale=MatchLowercase}
\fi
% use upquote if available, for straight quotes in verbatim environments
\IfFileExists{upquote.sty}{\usepackage{upquote}}{}
% use microtype if available
\IfFileExists{microtype.sty}{%
\usepackage[]{microtype}
\UseMicrotypeSet[protrusion]{basicmath} % disable protrusion for tt fonts
}{}
\IfFileExists{parskip.sty}{%
\usepackage{parskip}
}{% else
\setlength{\parindent}{0pt}
\setlength{\parskip}{6pt plus 2pt minus 1pt}
}
\usepackage{hyperref}
\hypersetup{
            pdftitle={Buzzoni D., Cunning R. and Baker A. C. (2021) Initial Algal Symbiont Communities Affect Shuffling to Durusdinium after Coral Bleaching and Reflect Seasonal Differences in Bleaching Sensitivity},
            pdfborder={0 0 0},
            breaklinks=true}
\urlstyle{same}  % don't use monospace font for urls
\usepackage[margin=1in]{geometry}
\usepackage{color}
\usepackage{fancyvrb}
\newcommand{\VerbBar}{|}
\newcommand{\VERB}{\Verb[commandchars=\\\{\}]}
\DefineVerbatimEnvironment{Highlighting}{Verbatim}{commandchars=\\\{\}}
% Add ',fontsize=\small' for more characters per line
\usepackage{framed}
\definecolor{shadecolor}{RGB}{248,248,248}
\newenvironment{Shaded}{\begin{snugshade}}{\end{snugshade}}
\newcommand{\AlertTok}[1]{\textcolor[rgb]{0.94,0.16,0.16}{#1}}
\newcommand{\AnnotationTok}[1]{\textcolor[rgb]{0.56,0.35,0.01}{\textbf{\textit{#1}}}}
\newcommand{\AttributeTok}[1]{\textcolor[rgb]{0.77,0.63,0.00}{#1}}
\newcommand{\BaseNTok}[1]{\textcolor[rgb]{0.00,0.00,0.81}{#1}}
\newcommand{\BuiltInTok}[1]{#1}
\newcommand{\CharTok}[1]{\textcolor[rgb]{0.31,0.60,0.02}{#1}}
\newcommand{\CommentTok}[1]{\textcolor[rgb]{0.56,0.35,0.01}{\textit{#1}}}
\newcommand{\CommentVarTok}[1]{\textcolor[rgb]{0.56,0.35,0.01}{\textbf{\textit{#1}}}}
\newcommand{\ConstantTok}[1]{\textcolor[rgb]{0.00,0.00,0.00}{#1}}
\newcommand{\ControlFlowTok}[1]{\textcolor[rgb]{0.13,0.29,0.53}{\textbf{#1}}}
\newcommand{\DataTypeTok}[1]{\textcolor[rgb]{0.13,0.29,0.53}{#1}}
\newcommand{\DecValTok}[1]{\textcolor[rgb]{0.00,0.00,0.81}{#1}}
\newcommand{\DocumentationTok}[1]{\textcolor[rgb]{0.56,0.35,0.01}{\textbf{\textit{#1}}}}
\newcommand{\ErrorTok}[1]{\textcolor[rgb]{0.64,0.00,0.00}{\textbf{#1}}}
\newcommand{\ExtensionTok}[1]{#1}
\newcommand{\FloatTok}[1]{\textcolor[rgb]{0.00,0.00,0.81}{#1}}
\newcommand{\FunctionTok}[1]{\textcolor[rgb]{0.00,0.00,0.00}{#1}}
\newcommand{\ImportTok}[1]{#1}
\newcommand{\InformationTok}[1]{\textcolor[rgb]{0.56,0.35,0.01}{\textbf{\textit{#1}}}}
\newcommand{\KeywordTok}[1]{\textcolor[rgb]{0.13,0.29,0.53}{\textbf{#1}}}
\newcommand{\NormalTok}[1]{#1}
\newcommand{\OperatorTok}[1]{\textcolor[rgb]{0.81,0.36,0.00}{\textbf{#1}}}
\newcommand{\OtherTok}[1]{\textcolor[rgb]{0.56,0.35,0.01}{#1}}
\newcommand{\PreprocessorTok}[1]{\textcolor[rgb]{0.56,0.35,0.01}{\textit{#1}}}
\newcommand{\RegionMarkerTok}[1]{#1}
\newcommand{\SpecialCharTok}[1]{\textcolor[rgb]{0.00,0.00,0.00}{#1}}
\newcommand{\SpecialStringTok}[1]{\textcolor[rgb]{0.31,0.60,0.02}{#1}}
\newcommand{\StringTok}[1]{\textcolor[rgb]{0.31,0.60,0.02}{#1}}
\newcommand{\VariableTok}[1]{\textcolor[rgb]{0.00,0.00,0.00}{#1}}
\newcommand{\VerbatimStringTok}[1]{\textcolor[rgb]{0.31,0.60,0.02}{#1}}
\newcommand{\WarningTok}[1]{\textcolor[rgb]{0.56,0.35,0.01}{\textbf{\textit{#1}}}}
\usepackage{graphicx,grffile}
\makeatletter
\def\maxwidth{\ifdim\Gin@nat@width>\linewidth\linewidth\else\Gin@nat@width\fi}
\def\maxheight{\ifdim\Gin@nat@height>\textheight\textheight\else\Gin@nat@height\fi}
\makeatother
% Scale images if necessary, so that they will not overflow the page
% margins by default, and it is still possible to overwrite the defaults
% using explicit options in \includegraphics[width, height, ...]{}
\setkeys{Gin}{width=\maxwidth,height=\maxheight,keepaspectratio}
\setlength{\emergencystretch}{3em}  % prevent overfull lines
\providecommand{\tightlist}{%
  \setlength{\itemsep}{0pt}\setlength{\parskip}{0pt}}
\setcounter{secnumdepth}{0}
% Redefines (sub)paragraphs to behave more like sections
\ifx\paragraph\undefined\else
\let\oldparagraph\paragraph
\renewcommand{\paragraph}[1]{\oldparagraph{#1}\mbox{}}
\fi
\ifx\subparagraph\undefined\else
\let\oldsubparagraph\subparagraph
\renewcommand{\subparagraph}[1]{\oldsubparagraph{#1}\mbox{}}
\fi

% set default figure placement to htbp
\makeatletter
\def\fps@figure{htbp}
\makeatother


\title{Buzzoni D., Cunning R. and Baker A. C. (2021) Initial Algal Symbiont
Communities Affect Shuffling to Durusdinium after Coral Bleaching and
Reflect Seasonal Differences in Bleaching Sensitivity}
\author{}
\date{\vspace{-2.5em}}

\begin{document}
\maketitle

\hypertarget{seasonal-temperatures}{%
\subsection{SEASONAL TEMPERATURES}\label{seasonal-temperatures}}

\hypertarget{figure-1a-temperatures-are-seasonal}{%
\subsubsection{Figure 1a: Temperatures are
seasonal}\label{figure-1a-temperatures-are-seasonal}}

\begin{Shaded}
\begin{Highlighting}[]
\NormalTok{temp3=}\KeywordTok{read.csv}\NormalTok{(}\StringTok{"foweytemps5yrs.csv"}\NormalTok{)}
\NormalTok{temp3}\OperatorTok{$}\NormalTok{date=}\KeywordTok{as.Date}\NormalTok{(}\KeywordTok{with}\NormalTok{(temp3, }\KeywordTok{paste}\NormalTok{(}\StringTok{'2019'}\NormalTok{,MM, DD,}\DataTypeTok{sep=}\StringTok{"-"}\NormalTok{)), }\StringTok{"%Y-%m-%d"}\NormalTok{)}
\NormalTok{ temp3}\OperatorTok{$}\NormalTok{WTMP=}\KeywordTok{as.numeric}\NormalTok{(}\KeywordTok{paste}\NormalTok{(temp3}\OperatorTok{$}\NormalTok{WTMP))}
\NormalTok{ temp=}\KeywordTok{read.csv}\NormalTok{(}\StringTok{"temp.csv"}\NormalTok{)}
\NormalTok{temp}\OperatorTok{$}\NormalTok{Date=}\KeywordTok{as.Date}\NormalTok{(temp}\OperatorTok{$}\NormalTok{Date,}\DataTypeTok{format =} \StringTok{"%d/%m/%Y"}\NormalTok{)}
 
 \KeywordTok{ggplot}\NormalTok{()}\OperatorTok{+}\KeywordTok{stat_summary}\NormalTok{(}\DataTypeTok{data=}\NormalTok{temp3,}\KeywordTok{aes}\NormalTok{(}\DataTypeTok{x=}\NormalTok{date,}\DataTypeTok{y=}\NormalTok{WTMP,}\DataTypeTok{fill=}\StringTok{'darkgreen'}\NormalTok{),}\DataTypeTok{fun.data=}\NormalTok{mean_se,}\DataTypeTok{geom=}\StringTok{'ribbon'}\NormalTok{,}\DataTypeTok{alpha=}\FloatTok{0.6}\NormalTok{)}\OperatorTok{+}\KeywordTok{theme_minimal}\NormalTok{(}\DataTypeTok{base_size =} \DecValTok{15}\NormalTok{)}\OperatorTok
\StringTok{    }\KeywordTok{theme}\NormalTok{(}\DataTypeTok{panel.border =} \KeywordTok{element_rect}\NormalTok{(}\DataTypeTok{colour=}\StringTok{'grey20'}\NormalTok{,}\DataTypeTok{fill=}\OtherTok{NA}\NormalTok{),}
          \DataTypeTok{panel.grid.minor.x =} \KeywordTok{element_blank}\NormalTok{(),}
          \DataTypeTok{panel.grid.minor.y =} \KeywordTok{element_blank}\NormalTok{())}\OperatorTok{+}
\StringTok{   }\KeywordTok{labs}\NormalTok{(}\DataTypeTok{x=}\StringTok{'Month (2019-20)'}\NormalTok{,}\DataTypeTok{y=}\StringTok{'Temperature/°C'}\NormalTok{)}\OperatorTok{+}\KeywordTok{stat_summary}\NormalTok{(}\DataTypeTok{data=}\NormalTok{temp,}\KeywordTok{aes}\NormalTok{(}\DataTypeTok{x=}\NormalTok{Date,}\DataTypeTok{y=}\NormalTok{Temp,}\DataTypeTok{fill=}\StringTok{'chartreuse2'}\NormalTok{),}\DataTypeTok{fun.data=}\NormalTok{mean_se,}\DataTypeTok{geom=}\StringTok{'ribbon'}\NormalTok{,}\DataTypeTok{alpha=}\FloatTok{0.6}\NormalTok{)}\OperatorTok{+}\KeywordTok{scale_fill_identity}\NormalTok{(}\DataTypeTok{guide=}\StringTok{'legend'}\NormalTok{,}\DataTypeTok{breaks=}\KeywordTok{c}\NormalTok{(}\StringTok{'darkgreen'}\NormalTok{,}\StringTok{'chartreuse2'}\NormalTok{),}\DataTypeTok{labels=}\KeywordTok{c}\NormalTok{(}\StringTok{'Fowey Rock 2015-19 average'}\NormalTok{,}\StringTok{'Emerald Reef 2019'}\NormalTok{))}\OperatorTok{+}\KeywordTok{guides}\NormalTok{(}\DataTypeTok{fill=}\KeywordTok{guide_legend}\NormalTok{(}\DataTypeTok{title=}\StringTok{''}\NormalTok{))}\OperatorTok{+}\KeywordTok{theme}\NormalTok{(}\DataTypeTok{legend.position =} \KeywordTok{c}\NormalTok{(}\FloatTok{0.6}\NormalTok{,}\FloatTok{0.3}\NormalTok{))}\OperatorTok{+}\KeywordTok{scale_x_date}\NormalTok{(}\DataTypeTok{date_labels =} \StringTok{'%b'}\NormalTok{,}\DataTypeTok{limits=}\KeywordTok{as.Date}\NormalTok{(}\KeywordTok{c}\NormalTok{(}\StringTok{'2019-01-01'}\NormalTok{,}\StringTok{'2019-12-31'}\NormalTok{)))}
\end{Highlighting}
\end{Shaded}

\includegraphics{SEASONAL_SHUFFLING_files/figure-latex/unnamed-chunk-3-1.pdf}

\begin{Shaded}
\begin{Highlighting}[]
\CommentTok{#ggsave('temp3.pdf',device='pdf', width=7,height=5) #add labels to figure to show two collection timepoints}
\end{Highlighting}
\end{Shaded}

\hypertarget{initial-symbiont-communities}{%
\subsection{INITIAL SYMBIONT
COMMUNITIES}\label{initial-symbiont-communities}}

\hypertarget{figure-1b-m.-cavernosa-symbionts-per-host-cell-increased-from-april-to-october.}{%
\subsubsection{Figure 1b: M. cavernosa symbionts per host cell increased
from April to
October.}\label{figure-1b-m.-cavernosa-symbionts-per-host-cell-increased-from-april-to-october.}}

Add arrows to show that points above line show higher in april, below
line shows higher in April.

\begin{Shaded}
\begin{Highlighting}[]
\CommentTok{#calculate seasonal per colony change in symbiont density (ie the average symbiont density between the two cores taken from each colony per batch)}
\NormalTok{batches}\OperatorTok{$}\NormalTok{pre_sh[}\KeywordTok{which}\NormalTok{(batches}\OperatorTok{$}\NormalTok{pre_sh}\OperatorTok{==}\DecValTok{0}\NormalTok{)]=}\OtherTok{NA}
\NormalTok{Apr_Oct_sh=}\StringTok{ }\NormalTok{batches }\OperatorTok
\StringTok{  }\KeywordTok{group_by}\NormalTok{(Colony,Batch) }\OperatorTok
\StringTok{  }\KeywordTok{summarise_at}\NormalTok{(}\KeywordTok{vars}\NormalTok{(pre_sh), }\KeywordTok{funs}\NormalTok{(}\KeywordTok{mean}\NormalTok{(., }\DataTypeTok{na.rm=}\OtherTok{TRUE}\NormalTok{)))}

\NormalTok{Apr_sh=}\StringTok{ }\KeywordTok{filter}\NormalTok{(Apr_Oct_sh, Batch}\OperatorTok{==}\StringTok{'April'}\NormalTok{)}
\NormalTok{Apr_sh}\OperatorTok{$}\NormalTok{Apr_pre_sh=}\KeywordTok{paste}\NormalTok{(Apr_sh}\OperatorTok{$}\NormalTok{pre_sh)}
\NormalTok{Oct_sh=}\StringTok{ }\KeywordTok{filter}\NormalTok{(Apr_Oct_sh, Batch}\OperatorTok{==}\StringTok{'October'}\NormalTok{)}
\NormalTok{Oct_sh}\OperatorTok{$}\NormalTok{Oct_pre_sh=}\KeywordTok{paste}\NormalTok{(Oct_sh}\OperatorTok{$}\NormalTok{pre_sh)}
\NormalTok{Apr_Oct_sh_change=}\KeywordTok{join}\NormalTok{(Apr_sh,Oct_sh,}\DataTypeTok{by=}\StringTok{'Colony'}\NormalTok{)}
\NormalTok{Apr_Oct_sh_change=Apr_Oct_sh_change[,}\KeywordTok{c}\NormalTok{(}\DecValTok{1}\NormalTok{,}\DecValTok{4}\NormalTok{,}\DecValTok{7}\NormalTok{)]}
\NormalTok{Apr_Oct_sh_change}\OperatorTok{$}\NormalTok{Apr_pre_sh=}\KeywordTok{as.numeric}\NormalTok{(Apr_Oct_sh_change}\OperatorTok{$}\NormalTok{Apr_pre_sh)}
\NormalTok{Apr_Oct_sh_change}\OperatorTok{$}\NormalTok{Oct_pre_sh=}\KeywordTok{as.numeric}\NormalTok{(Apr_Oct_sh_change}\OperatorTok{$}\NormalTok{Oct_pre_sh)}
\NormalTok{Apr_Oct_sh_change}\OperatorTok{$}\NormalTok{pre_sh_change=}\StringTok{ }\NormalTok{Apr_Oct_sh_change}\OperatorTok{$}\NormalTok{Oct_pre_sh}\OperatorTok{-}\StringTok{ }\NormalTok{Apr_Oct_sh_change}\OperatorTok{$}\NormalTok{Apr_pre_sh}
\NormalTok{Apr_Oct_sh_change}\OperatorTok{$}\NormalTok{rel_sh_change=}\StringTok{ }\NormalTok{((Apr_Oct_sh_change}\OperatorTok{$}\NormalTok{Oct_pre_sh}\OperatorTok{-}\StringTok{ }\NormalTok{Apr_Oct_sh_change}\OperatorTok{$}\NormalTok{Apr_pre_sh)}\OperatorTok{/}\NormalTok{Apr_Oct_sh_change}\OperatorTok{$}\NormalTok{Apr_pre_sh)}\OperatorTok{*}\DecValTok{100}
\NormalTok{Apr_Oct_sh_change}\OperatorTok{$}\NormalTok{octtoapr<-}\StringTok{ }\NormalTok{Apr_Oct_sh_change}\OperatorTok{$}\NormalTok{Oct_pre_sh}\OperatorTok{/}\NormalTok{Apr_Oct_sh_change}\OperatorTok{$}\NormalTok{Apr_pre_sh}
\NormalTok{Apr_Oct_sh_change}\OperatorTok{$}\NormalTok{Species=}\KeywordTok{factor}\NormalTok{(Apr_Oct_sh_change}\OperatorTok{$}\NormalTok{Colony)}
\NormalTok{Apr_Oct_sh_change}\OperatorTok{$}\NormalTok{Species=}\StringTok{ }\KeywordTok{mapvalues}\NormalTok{(Apr_Oct_sh_change}\OperatorTok{$}\NormalTok{Species,}
                          \DataTypeTok{from=}\KeywordTok{c}\NormalTok{(}\StringTok{'100'}\NormalTok{,}\StringTok{'2'}\NormalTok{,}\StringTok{'27'}\NormalTok{,}\StringTok{'28'}\NormalTok{,}\StringTok{'39'}\NormalTok{,}\StringTok{'67'}\NormalTok{,}\StringTok{'68'}\NormalTok{,}\StringTok{'71'}\NormalTok{,}\StringTok{'72'}\NormalTok{,}\StringTok{'87'}\NormalTok{),}\DataTypeTok{to=}\NormalTok{(}\KeywordTok{rep}\NormalTok{(}\StringTok{'M.cavernosa'}\NormalTok{,}\DataTypeTok{times=}\DecValTok{10}\NormalTok{)))}
\NormalTok{ Apr_Oct_sh_change}\OperatorTok{$}\NormalTok{Species=}\StringTok{ }\KeywordTok{mapvalues}\NormalTok{(Apr_Oct_sh_change}\OperatorTok{$}\NormalTok{Species,}
                          \DataTypeTok{from=}\KeywordTok{c}\NormalTok{(}\StringTok{'13'}\NormalTok{,}\StringTok{'21'}\NormalTok{,}\StringTok{'34'}\NormalTok{,}\StringTok{'36'}\NormalTok{,}\StringTok{'4'}\NormalTok{,}\StringTok{'62'}\NormalTok{,}\StringTok{'64'}\NormalTok{,}\StringTok{'65'}\NormalTok{,}\StringTok{'66'}\NormalTok{,}\StringTok{'81'}\NormalTok{),}\DataTypeTok{to=}\NormalTok{(}\KeywordTok{rep}\NormalTok{(}\StringTok{'O.faveolata'}\NormalTok{,}\DataTypeTok{times=}\DecValTok{10}\NormalTok{)))                         }
\NormalTok{  Apr_Oct_sh_change}\OperatorTok{$}\NormalTok{Species=}\StringTok{ }\KeywordTok{mapvalues}\NormalTok{(Apr_Oct_sh_change}\OperatorTok{$}\NormalTok{Species,}
                          \DataTypeTok{from=}\KeywordTok{c}\NormalTok{(}\StringTok{'16'}\NormalTok{,}\StringTok{'18'}\NormalTok{,}\StringTok{'19'}\NormalTok{,}\StringTok{'22'}\NormalTok{,}\StringTok{'23'}\NormalTok{,}\StringTok{'26'}\NormalTok{,}\StringTok{'3'}\NormalTok{,}\StringTok{'35'}\NormalTok{,}\StringTok{'41'}\NormalTok{,}\StringTok{'48'}\NormalTok{),}\DataTypeTok{to=}\NormalTok{(}\KeywordTok{rep}\NormalTok{(}\StringTok{'S.siderea'}\NormalTok{,}\DataTypeTok{times=}\DecValTok{10}\NormalTok{))) }
\CommentTok{# this dataframe shows the average change in s:h per colony}


\CommentTok{##test statistical significance in seasonal change in S:H}
 \KeywordTok{hist}\NormalTok{(}\KeywordTok{log10}\NormalTok{(batches}\OperatorTok{$}\NormalTok{pre_sh)) }\CommentTok{#log10 transformed data then used linear mixed effects model on s:h data to test effect of batch within each species, with colony as a random factor}
\end{Highlighting}
\end{Shaded}

\includegraphics{SEASONAL_SHUFFLING_files/figure-latex/unnamed-chunk-4-1.pdf}

\begin{Shaded}
\begin{Highlighting}[]
\NormalTok{batches}\OperatorTok{$}\NormalTok{transf_presh=}\StringTok{ }\KeywordTok{log10}\NormalTok{(batches}\OperatorTok{$}\NormalTok{pre_sh) }\CommentTok{# transform the response variable}
\NormalTok{mcavpreshmod=}\KeywordTok{lmer}\NormalTok{(}\KeywordTok{log10}\NormalTok{(pre_sh)}\OperatorTok{~}\NormalTok{Batch}\OperatorTok{+}\NormalTok{(}\DecValTok{1}\OperatorTok{|}\NormalTok{Colony),}\DataTypeTok{data=}\KeywordTok{filter}\NormalTok{(batches,batches}\OperatorTok{$}\NormalTok{Species}\OperatorTok{==}\StringTok{'M.cavernosa'}\NormalTok{))}
\NormalTok{rc_resids<-}\StringTok{ }\KeywordTok{compute_redres}\NormalTok{(mcavpreshmod)}
\NormalTok{resids<-}\StringTok{ }\KeywordTok{subset}\NormalTok{(batches,batches}\OperatorTok{$}\NormalTok{Species}\OperatorTok{==}\StringTok{'M.cavernosa'}\NormalTok{)}
\NormalTok{resids}\OperatorTok{$}\NormalTok{logpresh<-}\StringTok{ }\KeywordTok{log10}\NormalTok{(resids}\OperatorTok{$}\NormalTok{pre_sh)}
\NormalTok{resids<-resids[,}\KeywordTok{c}\NormalTok{(}\DecValTok{12}\NormalTok{)]}
\NormalTok{logpre<-resids[}\OperatorTok{-}\KeywordTok{c}\NormalTok{(}\DecValTok{5}\NormalTok{,}\DecValTok{6}\NormalTok{)]}
\NormalTok{resids<-}\StringTok{ }\KeywordTok{data.frame}\NormalTok{(logpre, rc_resids) }
\KeywordTok{plot_resqq}\NormalTok{(mcavpreshmod) }\CommentTok{# check residuals are normally distributed}
\end{Highlighting}
\end{Shaded}

\includegraphics{SEASONAL_SHUFFLING_files/figure-latex/unnamed-chunk-4-2.pdf}

\begin{Shaded}
\begin{Highlighting}[]
\NormalTok{mcavpreshmod2<-}\StringTok{ }\KeywordTok{as_lmerModLmerTest}\NormalTok{(mcavpreshmod)}
\KeywordTok{summary}\NormalTok{(mcavpreshmod2) }\CommentTok{#p for batch, p=0.001126** }
\end{Highlighting}
\end{Shaded}

\begin{verbatim}
## Linear mixed model fit by REML. t-tests use Satterthwaite's method [
## lmerModLmerTest]
## Formula: log10(pre_sh) ~ Batch + (1 | Colony)
##    Data: filter(batches, batches$Species == "M.cavernosa")
## 
## REML criterion at convergence: 56
## 
## Scaled residuals: 
##      Min       1Q   Median       3Q      Max 
## -2.17638 -0.65717 -0.08901  0.55392  2.38020 
## 
## Random effects:
##  Groups   Name        Variance  Std.Dev.
##  Colony   (Intercept) 3.240e-08 0.00018 
##  Residual             2.357e-01 0.48546 
## Number of obs: 38, groups:  Colony, 10
## 
## Fixed effects:
##              Estimate Std. Error      df t value Pr(>|t|)    
## (Intercept)   -2.3177     0.1144 35.9720  -20.25  < 2e-16 ***
## BatchOctober   0.5583     0.1577 35.9823    3.54  0.00113 ** 
## ---
## Signif. codes:  0 '***' 0.001 '**' 0.01 '*' 0.05 '.' 0.1 ' ' 1
## 
## Correlation of Fixed Effects:
##             (Intr)
## BatchOctobr -0.725
\end{verbatim}

\begin{Shaded}
\begin{Highlighting}[]
\CommentTok{# now let's get the batch parameter estimates and CIs:}
\NormalTok{mcavemm.sh<-}\StringTok{ }\KeywordTok{emmeans}\NormalTok{(mcavpreshmod, }\DataTypeTok{specs=}\NormalTok{revpairwise}\OperatorTok{~}\NormalTok{Batch) }
\KeywordTok{summary}\NormalTok{(mcavemm.sh)}
\end{Highlighting}
\end{Shaded}

\begin{verbatim}
## $emmeans
##  Batch   emmean    SE   df lower.CL upper.CL
##  April    -2.32 0.115 25.7    -2.55    -2.08
##  October  -1.76 0.109 25.7    -1.98    -1.54
## 
## Degrees-of-freedom method: kenward-roger 
## Results are given on the log10 (not the response) scale. 
## Confidence level used: 0.95 
## 
## $contrasts
##  contrast        estimate    SE   df t.ratio p.value
##  October - April    0.558 0.158 28.7 3.525   0.0014 
## 
## Note: contrasts are still on the log10 scale 
## Degrees-of-freedom method: kenward-roger
\end{verbatim}

\begin{Shaded}
\begin{Highlighting}[]
\NormalTok{mcavemmsh_contrasts<-}\StringTok{ }\NormalTok{mcavemm.sh}\OperatorTok{$}\NormalTok{contrasts }\OperatorTok
\StringTok{     }\KeywordTok{confint}\NormalTok{()}\OperatorTok
\StringTok{     }\KeywordTok{as.data.frame}\NormalTok{() }\CommentTok{# NB these results are given on a log10 scale}
\CommentTok{#also include April prediction in this dataframe for later calculations}

\NormalTok{mcavemmsh_contrasts<-}\KeywordTok{as.data.frame}\NormalTok{(}\KeywordTok{c}\NormalTok{(mcavemmsh_contrasts,mcavemm.sh}\OperatorTok{$}\NormalTok{emmeans[}\DecValTok{1}\NormalTok{]))  }\CommentTok{# since we're taking the }

\NormalTok{ofavpreshmod=}\KeywordTok{lmer}\NormalTok{(}\KeywordTok{log10}\NormalTok{(pre_sh)}\OperatorTok{~}\NormalTok{Batch}\OperatorTok{+}\NormalTok{(}\DecValTok{1}\OperatorTok{|}\NormalTok{Colony),}\DataTypeTok{data=}\KeywordTok{filter}\NormalTok{(batches,batches}\OperatorTok{$}\NormalTok{Species}\OperatorTok{==}\StringTok{'O.faveolata'}\NormalTok{))}
\NormalTok{rc_resids<-}\StringTok{ }\KeywordTok{compute_redres}\NormalTok{(ofavpreshmod)}
\NormalTok{resids<-}\StringTok{ }\KeywordTok{subset}\NormalTok{(batches,batches}\OperatorTok{$}\NormalTok{Species}\OperatorTok{==}\StringTok{'O.faveolata'}\NormalTok{)}
\NormalTok{resids}\OperatorTok{$}\NormalTok{logpresh<-}\StringTok{ }\KeywordTok{log10}\NormalTok{(resids}\OperatorTok{$}\NormalTok{pre_sh)}
\NormalTok{resids<-resids[,}\KeywordTok{c}\NormalTok{(}\DecValTok{12}\NormalTok{)]}
\NormalTok{logpre<-resids}
\NormalTok{resids<-}\StringTok{ }\KeywordTok{data.frame}\NormalTok{(logpre, rc_resids) }
\KeywordTok{plot_resqq}\NormalTok{(mcavpreshmod) }\CommentTok{# check residuals are normally distributed}
\end{Highlighting}
\end{Shaded}

\includegraphics{SEASONAL_SHUFFLING_files/figure-latex/unnamed-chunk-4-3.pdf}

\begin{Shaded}
\begin{Highlighting}[]
\NormalTok{ofavpreshmod2<-}\StringTok{ }\KeywordTok{as_lmerModLmerTest}\NormalTok{(ofavpreshmod)}
\KeywordTok{summary}\NormalTok{(ofavpreshmod2) }\CommentTok{#p for batch, p=0.938}
\end{Highlighting}
\end{Shaded}

\begin{verbatim}
## Linear mixed model fit by REML. t-tests use Satterthwaite's method [
## lmerModLmerTest]
## Formula: log10(pre_sh) ~ Batch + (1 | Colony)
##    Data: filter(batches, batches$Species == "O.faveolata")
## 
## REML criterion at convergence: 29.6
## 
## Scaled residuals: 
##     Min      1Q  Median      3Q     Max 
## -2.3754 -0.5262  0.1305  0.6129  1.6623 
## 
## Random effects:
##  Groups   Name        Variance Std.Dev.
##  Colony   (Intercept) 0.12853  0.3585  
##  Residual             0.07002  0.2646  
## Number of obs: 35, groups:  Colony, 10
## 
## Fixed effects:
##               Estimate Std. Error        df t value Pr(>|t|)    
## (Intercept)  -1.491446   0.128795 11.205854 -11.580  1.4e-07 ***
## BatchOctober  0.007186   0.091378 24.399854   0.079    0.938    
## ---
## Signif. codes:  0 '***' 0.001 '**' 0.01 '*' 0.05 '.' 0.1 ' ' 1
## 
## Correlation of Fixed Effects:
##             (Intr)
## BatchOctobr -0.320
\end{verbatim}

\begin{Shaded}
\begin{Highlighting}[]
\CommentTok{# now let's get the batch parameter estimates and CIs:}
\NormalTok{ofavemm.sh<-}\StringTok{ }\KeywordTok{emmeans}\NormalTok{(ofavpreshmod, }\DataTypeTok{specs=}\NormalTok{revpairwise}\OperatorTok{~}\NormalTok{Batch) }
\KeywordTok{summary}\NormalTok{(ofavemm.sh)}
\end{Highlighting}
\end{Shaded}

\begin{verbatim}
## $emmeans
##  Batch   emmean    SE   df lower.CL upper.CL
##  April    -1.49 0.129 11.1    -1.77    -1.21
##  October  -1.48 0.132 12.1    -1.77    -1.20
## 
## Degrees-of-freedom method: kenward-roger 
## Results are given on the log10 (not the response) scale. 
## Confidence level used: 0.95 
## 
## $contrasts
##  contrast        estimate     SE   df t.ratio p.value
##  October - April  0.00719 0.0915 24.3 0.078   0.9381 
## 
## Note: contrasts are still on the log10 scale 
## Degrees-of-freedom method: kenward-roger
\end{verbatim}

\begin{Shaded}
\begin{Highlighting}[]
\NormalTok{ofavemmsh_contrasts<-}\StringTok{ }\NormalTok{ofavemm.sh}\OperatorTok{$}\NormalTok{contrasts }\OperatorTok
\StringTok{     }\KeywordTok{confint}\NormalTok{()}\OperatorTok
\StringTok{     }\KeywordTok{as.data.frame}\NormalTok{() }\CommentTok{# NB these results are given on a log10 scale}
\CommentTok{#also include April prediction in this dataframe for later calculations}
\NormalTok{ofavemmsh_contrasts<-}\KeywordTok{as.data.frame}\NormalTok{(}\KeywordTok{c}\NormalTok{(ofavemmsh_contrasts,ofavemm.sh}\OperatorTok{$}\NormalTok{emmeans[}\DecValTok{1}\NormalTok{])) }

\NormalTok{ssidpreshmod=}\KeywordTok{lmer}\NormalTok{(}\KeywordTok{log10}\NormalTok{(pre_sh)}\OperatorTok{~}\NormalTok{Batch}\OperatorTok{+}\NormalTok{(}\DecValTok{1}\OperatorTok{|}\NormalTok{Colony),}\DataTypeTok{data=}\KeywordTok{filter}\NormalTok{(batches,batches}\OperatorTok{$}\NormalTok{Species}\OperatorTok{==}\StringTok{'S.siderea'}\NormalTok{))}
\KeywordTok{plot}\NormalTok{(ssidpreshmod)}
\end{Highlighting}
\end{Shaded}

\includegraphics{SEASONAL_SHUFFLING_files/figure-latex/unnamed-chunk-4-4.pdf}

\begin{Shaded}
\begin{Highlighting}[]
\NormalTok{rc_resids<-}\StringTok{ }\KeywordTok{compute_redres}\NormalTok{(ssidpreshmod)}
\NormalTok{resids<-}\StringTok{ }\KeywordTok{subset}\NormalTok{(batches,batches}\OperatorTok{$}\NormalTok{Species}\OperatorTok{==}\StringTok{'S.siderea'}\NormalTok{)}
\NormalTok{resids}\OperatorTok{$}\NormalTok{logpresh<-}\StringTok{ }\KeywordTok{log10}\NormalTok{(resids}\OperatorTok{$}\NormalTok{pre_sh)}
\NormalTok{resids<-resids[,}\KeywordTok{c}\NormalTok{(}\DecValTok{12}\NormalTok{)]}
\NormalTok{logpre<-resids[}\OperatorTok{-}\KeywordTok{c}\NormalTok{(}\DecValTok{19}\NormalTok{,}\DecValTok{20}\NormalTok{,}\DecValTok{39}\NormalTok{,}\DecValTok{40}\NormalTok{)]}
\NormalTok{resids<-}\StringTok{ }\KeywordTok{data.frame}\NormalTok{(logpre, rc_resids) }
\KeywordTok{plot_resqq}\NormalTok{(mcavpreshmod) }\CommentTok{# check residuals are normally distributed}
\end{Highlighting}
\end{Shaded}

\includegraphics{SEASONAL_SHUFFLING_files/figure-latex/unnamed-chunk-4-5.pdf}

\begin{Shaded}
\begin{Highlighting}[]
\NormalTok{ssidpreshmod2<-}\StringTok{ }\KeywordTok{as_lmerModLmerTest}\NormalTok{(ssidpreshmod)}
\KeywordTok{summary}\NormalTok{(ssidpreshmod2) }\CommentTok{#p for batch, p=0.04833*}
\end{Highlighting}
\end{Shaded}

\begin{verbatim}
## Linear mixed model fit by REML. t-tests use Satterthwaite's method [
## lmerModLmerTest]
## Formula: log10(pre_sh) ~ Batch + (1 | Colony)
##    Data: filter(batches, batches$Species == "S.siderea")
## 
## REML criterion at convergence: 55
## 
## Scaled residuals: 
##      Min       1Q   Median       3Q      Max 
## -2.53246 -0.34681  0.00358  0.49889  1.82110 
## 
## Random effects:
##  Groups   Name        Variance Std.Dev.
##  Colony   (Intercept) 0.2705   0.5201  
##  Residual             0.1524   0.3903  
## Number of obs: 36, groups:  Colony, 9
## 
## Fixed effects:
##              Estimate Std. Error      df t value Pr(>|t|)    
## (Intercept)   -3.0287     0.1963 10.0494 -15.431 2.51e-08 ***
## BatchOctober  -0.2696     0.1301 26.0000  -2.072   0.0483 *  
## ---
## Signif. codes:  0 '***' 0.001 '**' 0.01 '*' 0.05 '.' 0.1 ' ' 1
## 
## Correlation of Fixed Effects:
##             (Intr)
## BatchOctobr -0.331
\end{verbatim}

\begin{Shaded}
\begin{Highlighting}[]
\CommentTok{# now let's get the batch parameter estimates and CIs:}
\NormalTok{ssidemm.sh<-}\StringTok{ }\KeywordTok{emmeans}\NormalTok{(ssidpreshmod, }\DataTypeTok{specs=}\NormalTok{revpairwise}\OperatorTok{~}\NormalTok{Batch) }
\KeywordTok{summary}\NormalTok{(ssidemm.sh)}
\end{Highlighting}
\end{Shaded}

\begin{verbatim}
## $emmeans
##  Batch   emmean    SE   df lower.CL upper.CL
##  April    -3.03 0.196 10.1    -3.47    -2.59
##  October  -3.30 0.196 10.1    -3.74    -2.86
## 
## Degrees-of-freedom method: kenward-roger 
## Results are given on the log10 (not the response) scale. 
## Confidence level used: 0.95 
## 
## $contrasts
##  contrast        estimate   SE df t.ratio p.value
##  October - April    -0.27 0.13 26 -2.072  0.0483 
## 
## Note: contrasts are still on the log10 scale 
## Degrees-of-freedom method: kenward-roger
\end{verbatim}

\begin{Shaded}
\begin{Highlighting}[]
\NormalTok{ssidemmsh_contrasts<-}\StringTok{ }\NormalTok{ssidemm.sh}\OperatorTok{$}\NormalTok{contrasts }\OperatorTok
\StringTok{     }\KeywordTok{confint}\NormalTok{()}\OperatorTok
\StringTok{     }\KeywordTok{as.data.frame}\NormalTok{() }\CommentTok{# NB these results are given on a log10 scale}
\CommentTok{#also include April prediction in this dataframe for later calculations}
\NormalTok{ssidemmsh_contrasts<-}\KeywordTok{as.data.frame}\NormalTok{(}\KeywordTok{c}\NormalTok{(ssidemmsh_contrasts,ssidemm.sh}\OperatorTok{$}\NormalTok{emmeans[}\DecValTok{1}\NormalTok{])) }

\CommentTok{#collate the dataframe}
\NormalTok{sh_emmcontrasts<-}\KeywordTok{rbind}\NormalTok{(mcavemmsh_contrasts,ofavemmsh_contrasts,ssidemmsh_contrasts)}
\NormalTok{sh_emmcontrasts}\OperatorTok{$}\NormalTok{Species<-}\KeywordTok{c}\NormalTok{(}\StringTok{'M.cavernosa'}\NormalTok{,}\StringTok{'O.faveolata'}\NormalTok{,}\StringTok{'S.siderea'}\NormalTok{)}
\NormalTok{sh_emmcontrasts}\OperatorTok{$}\NormalTok{Species<-}\KeywordTok{as.factor}\NormalTok{(sh_emmcontrasts}\OperatorTok{$}\NormalTok{Species)}
\NormalTok{sh_emmcontrasts}\OperatorTok{$}\NormalTok{change_untransformed=}\StringTok{ }\DecValTok{10}\OperatorTok{^}\NormalTok{(sh_emmcontrasts}\OperatorTok{$}\NormalTok{estimate)}
\NormalTok{sh_emmcontrasts}\OperatorTok{$}\NormalTok{lowerCI_change_untransformed=}\StringTok{ }\DecValTok{10}\OperatorTok{^}\NormalTok{(sh_emmcontrasts}\OperatorTok{$}\NormalTok{lower.CL)}
\NormalTok{sh_emmcontrasts}\OperatorTok{$}\NormalTok{upperCI_change_untransformed=}\StringTok{ }\DecValTok{10}\OperatorTok{^}\NormalTok{(sh_emmcontrasts}\OperatorTok{$}\NormalTok{upper.CL)}
\NormalTok{sh_emmcontrasts}\OperatorTok{$}\NormalTok{April_untransformed=}\StringTok{ }\DecValTok{10}\OperatorTok{^}\NormalTok{(sh_emmcontrasts}\OperatorTok{$}\NormalTok{emmean)  }

\CommentTok{#plot set-up}
\NormalTok{ofav=}\KeywordTok{expression}\NormalTok{(}\KeywordTok{paste}\NormalTok{(}\KeywordTok{italic}\NormalTok{(}\StringTok{"O. faveolata"}\NormalTok{)))}
\NormalTok{ssid=}\KeywordTok{expression}\NormalTok{(}\KeywordTok{paste}\NormalTok{(}\KeywordTok{italic}\NormalTok{(}\StringTok{"S. siderea"}\NormalTok{)))}
\NormalTok{mcav=}\KeywordTok{expression}\NormalTok{(}\KeywordTok{paste}\NormalTok{(}\KeywordTok{italic}\NormalTok{(}\StringTok{"M. cavernosa"}\NormalTok{)))}

\CommentTok{#PLOT 2A.SEASONAL CHANGE IN INITIAL S:H}
\KeywordTok{ggplot}\NormalTok{()}\OperatorTok{+}
\StringTok{  }\KeywordTok{geom_point}\NormalTok{(}\DataTypeTok{data=}\NormalTok{sh_emmcontrasts, }\KeywordTok{aes}\NormalTok{(}\DataTypeTok{x=}\NormalTok{Species,}\DataTypeTok{y=}\NormalTok{(change_untransformed), }\DataTypeTok{colour=}\NormalTok{Species), }\DataTypeTok{size=}\DecValTok{2}\NormalTok{)}\OperatorTok{+}\StringTok{ }\CommentTok{#large points are showing estimated october s:h - april s:h}
\StringTok{  }\KeywordTok{geom_point}\NormalTok{(}\DataTypeTok{data=}\NormalTok{Apr_Oct_sh_change, }\KeywordTok{aes}\NormalTok{(}\DataTypeTok{x=}\NormalTok{Species,}\DataTypeTok{y=}\NormalTok{pre_sh_change, }\DataTypeTok{colour=}\NormalTok{Species), }\DataTypeTok{size=}\FloatTok{0.5}\NormalTok{,  }\DataTypeTok{position=}\KeywordTok{position_jitter}\NormalTok{(}\DataTypeTok{width=}\FloatTok{0.1}\NormalTok{))}\OperatorTok{+}\StringTok{ }\CommentTok{#small points are showing october s:h / april s:h}
\StringTok{  }\KeywordTok{geom_errorbar}\NormalTok{(}\DataTypeTok{data=}\NormalTok{sh_emmcontrasts, }\KeywordTok{aes}\NormalTok{(}\DataTypeTok{x=}\NormalTok{Species, }\DataTypeTok{ymin=}\NormalTok{(lowerCI_change_untransformed), }\DataTypeTok{ymax=}\NormalTok{(upperCI_change_untransformed), }\DataTypeTok{colour=}\NormalTok{Species), }\DataTypeTok{size=}\FloatTok{0.5}\NormalTok{, }\DataTypeTok{width=}\FloatTok{0.2}\NormalTok{)}\OperatorTok{+}
\StringTok{  }\KeywordTok{theme_minimal}\NormalTok{(}\DataTypeTok{base_size =} \DecValTok{15}\NormalTok{)}\OperatorTok
\StringTok{    }\KeywordTok{theme}\NormalTok{(}\DataTypeTok{panel.border =} \KeywordTok{element_rect}\NormalTok{(}\DataTypeTok{colour=}\StringTok{'grey20'}\NormalTok{,}\DataTypeTok{fill=}\OtherTok{NA}\NormalTok{),}
          \DataTypeTok{panel.grid.major.x =} \KeywordTok{element_blank}\NormalTok{(),}
          \DataTypeTok{panel.grid.minor.x =} \KeywordTok{element_blank}\NormalTok{(),}
          \DataTypeTok{panel.grid.minor.y =} \KeywordTok{element_blank}\NormalTok{())}\OperatorTok{+}
\StringTok{  }\KeywordTok{labs}\NormalTok{(}\DataTypeTok{y=}\StringTok{'October : April symbionts per host cell'}\NormalTok{, }\DataTypeTok{x=}\StringTok{''}\NormalTok{)}\OperatorTok{+}
\StringTok{   }\KeywordTok{scale_x_discrete}\NormalTok{(}\DataTypeTok{labels=}\KeywordTok{c}\NormalTok{(mcav,ofav,ssid), }\DataTypeTok{expand=}\KeywordTok{expansion}\NormalTok{(}\DataTypeTok{add=}\KeywordTok{c}\NormalTok{(}\FloatTok{0.4}\NormalTok{,}\FloatTok{1.2}\NormalTok{)))}\OperatorTok{+}
\StringTok{  }\KeywordTok{scale_color_manual}\NormalTok{(}\DataTypeTok{values=}\KeywordTok{c}\NormalTok{(}\StringTok{'deeppink2'}\NormalTok{,}\StringTok{'darkorange1'}\NormalTok{, }\StringTok{'darkturquoise'}\NormalTok{))}\OperatorTok{+}
\StringTok{  }\KeywordTok{guides}\NormalTok{(}\DataTypeTok{colour=}\NormalTok{F)}\OperatorTok{+}
\StringTok{  }\KeywordTok{geom_hline}\NormalTok{(}\DataTypeTok{yintercept=}\DecValTok{1}\NormalTok{,}\DataTypeTok{linetype=}\StringTok{'dashed'}\NormalTok{)}\OperatorTok{+}
\StringTok{  }\KeywordTok{scale_y_continuous}\NormalTok{(}\DataTypeTok{breaks=}\KeywordTok{c}\NormalTok{(}\FloatTok{0.5}\NormalTok{,}\DecValTok{1}\NormalTok{,}\DecValTok{2}\NormalTok{,}\DecValTok{4}\NormalTok{,}\DecValTok{6}\NormalTok{,}\DecValTok{8}\NormalTok{))}\OperatorTok{+}
\StringTok{  }\KeywordTok{coord_cartesian}\NormalTok{(}\DataTypeTok{ylim=}\KeywordTok{c}\NormalTok{(}\FloatTok{0.3}\NormalTok{,}\DecValTok{8}\NormalTok{))}
\end{Highlighting}
\end{Shaded}

\includegraphics{SEASONAL_SHUFFLING_files/figure-latex/unnamed-chunk-4-6.pdf}

\begin{Shaded}
\begin{Highlighting}[]
\CommentTok{#ggsave('initialSH_seasonal_emmeans.pdf',device='pdf',width=7,height=5) # add 'higher in Oct'/'higher in Apr' labels to figure}

\CommentTok{#UPDATES FOR ROSS: This is now the emmeans predicted seasonal differnece, with 95% confidence intervals. Note some individual mcav datapoints are beyond the limits of the graph but accounted for in the predicted means and statistical calculations. }
\CommentTok{#potential problem with this plot is that by plotting the ratio, a ratio of 0.5 (50% decrease) may be equivalent to a ratio of 2 (100% increase)...? Is this plot disproportionately inflating positive values?}
\end{Highlighting}
\end{Shaded}

\hypertarget{figure-1c-proportion-durusdinium-did-not-change-between-april-and-october-for-any-of-the-3-coral-species.}{%
\subsubsection{Figure 1c: Proportion Durusdinium did not change between
April and October for any of the 3 coral
species.}\label{figure-1c-proportion-durusdinium-did-not-change-between-april-and-october-for-any-of-the-3-coral-species.}}

\begin{Shaded}
\begin{Highlighting}[]
\NormalTok{batch_comparevi=}
\KeywordTok{ggplot}\NormalTok{()}\OperatorTok{+}
\StringTok{   }\KeywordTok{theme_minimal}\NormalTok{(}\DataTypeTok{base_size =} \DecValTok{15}\NormalTok{)}\OperatorTok
\StringTok{    }\KeywordTok{theme}\NormalTok{(}\DataTypeTok{panel.border =} \KeywordTok{element_rect}\NormalTok{(}\DataTypeTok{colour=}\StringTok{'grey20'}\NormalTok{,}\DataTypeTok{fill=}\OtherTok{NA}\NormalTok{),}
          \DataTypeTok{panel.grid.major.x =} \KeywordTok{element_blank}\NormalTok{(),}
          \DataTypeTok{panel.grid.minor.x =} \KeywordTok{element_blank}\NormalTok{())}\OperatorTok{+}\StringTok{ }
\StringTok{  }\KeywordTok{labs}\NormalTok{(}\DataTypeTok{y=}\StringTok{'Initial proportion *Durusdinium*'}\NormalTok{)}\OperatorTok{+}
\StringTok{  }\KeywordTok{scale_fill_manual}\NormalTok{(}\DataTypeTok{values=}\KeywordTok{c}\NormalTok{(}\StringTok{'#3C5488B2'}\NormalTok{,}\StringTok{'#00A087B2'}\NormalTok{),}\DataTypeTok{labels=}\KeywordTok{c}\NormalTok{(}\StringTok{'April'}\NormalTok{,}\StringTok{'October'}\NormalTok{))}\OperatorTok{+}
\StringTok{  }\KeywordTok{geom_violin}\NormalTok{(}\DataTypeTok{data=}\NormalTok{batches,}\KeywordTok{aes}\NormalTok{(}\DataTypeTok{x=}\NormalTok{Species,}\DataTypeTok{y=}\NormalTok{pre_propD,}\DataTypeTok{colour=}\NormalTok{Batch),}\DataTypeTok{scale =} \StringTok{'width'}\NormalTok{)}\OperatorTok{+}
\StringTok{  }\KeywordTok{geom_dotplot}\NormalTok{(}\DataTypeTok{data=}\NormalTok{batches,}\DataTypeTok{bins=}\DecValTok{30}\NormalTok{,}\DataTypeTok{binaxis=}\StringTok{'y'}\NormalTok{,}\DataTypeTok{dotsize=}\FloatTok{0.7}\NormalTok{,}\DataTypeTok{stackratio=}\FloatTok{0.5}\NormalTok{,}\DataTypeTok{stackdir=}\StringTok{'center'}\NormalTok{,}\DataTypeTok{stackgroups=}\NormalTok{F, }\DataTypeTok{position=}\StringTok{'dodge'}\NormalTok{,}\KeywordTok{aes}\NormalTok{(}\DataTypeTok{x=}\NormalTok{ Species,}\DataTypeTok{y=}\NormalTok{pre_propD,}\DataTypeTok{fill=}\NormalTok{Batch))}\OperatorTok{+}
\StringTok{  }\KeywordTok{scale_colour_manual}\NormalTok{(}\DataTypeTok{values=}\KeywordTok{c}\NormalTok{(}\StringTok{'#3C5488B2'}\NormalTok{,}\StringTok{'#00A087B2'}\NormalTok{),}\DataTypeTok{labels=}\KeywordTok{c}\NormalTok{(}\StringTok{'April'}\NormalTok{,}\StringTok{'October'}\NormalTok{))}\OperatorTok{+}
\StringTok{  }\KeywordTok{scale_x_discrete}\NormalTok{(}\DataTypeTok{labels=}\KeywordTok{c}\NormalTok{(mcav,ofav,ssid))}\OperatorTok{+}
\StringTok{   }\KeywordTok{theme}\NormalTok{(}\DataTypeTok{axis.title.y.left  =} \KeywordTok{element_markdown}\NormalTok{(), }\DataTypeTok{legend.position =} \KeywordTok{c}\NormalTok{(}\FloatTok{0.2}\NormalTok{,}\FloatTok{0.5}\NormalTok{))}\OperatorTok{+}
\StringTok{  }\KeywordTok{labs}\NormalTok{(}\DataTypeTok{x=}\StringTok{''}\NormalTok{)}\OperatorTok{+}
\StringTok{  }\KeywordTok{theme}\NormalTok{(}\DataTypeTok{legend.title =} \KeywordTok{element_text}\NormalTok{(}\DataTypeTok{size=}\DecValTok{0}\NormalTok{))}
\NormalTok{batch_comparevi}
\end{Highlighting}
\end{Shaded}

\includegraphics{SEASONAL_SHUFFLING_files/figure-latex/unnamed-chunk-5-1.pdf}

\begin{Shaded}
\begin{Highlighting}[]
\CommentTok{#ggsave('batchcompare_propd_vi_dot.pdf',device='pdf',width=7,height=5)}

\KeywordTok{hist}\NormalTok{(batches}\OperatorTok{$}\NormalTok{pre_propD)}
\end{Highlighting}
\end{Shaded}

\includegraphics{SEASONAL_SHUFFLING_files/figure-latex/unnamed-chunk-5-2.pdf}

\begin{Shaded}
\begin{Highlighting}[]
\NormalTok{mcavpredmod=}\KeywordTok{glmer}\NormalTok{(pre_propD}\OperatorTok{~}\NormalTok{Batch}\OperatorTok{+}\NormalTok{(}\DecValTok{1}\OperatorTok{|}\NormalTok{Colony),}\DataTypeTok{data=}\KeywordTok{filter}\NormalTok{(batches,batches}\OperatorTok{$}\NormalTok{Species}\OperatorTok{==}\StringTok{'M.cavernosa'}\NormalTok{), }\DataTypeTok{family =} \StringTok{'binomial'}\NormalTok{)}
\KeywordTok{summary}\NormalTok{(mcavpredmod) }\CommentTok{# p=1}
\end{Highlighting}
\end{Shaded}

\begin{verbatim}
## Generalized linear mixed model fit by maximum likelihood (Laplace
##   Approximation) [glmerMod]
##  Family: binomial  ( logit )
## Formula: pre_propD ~ Batch + (1 | Colony)
##    Data: filter(batches, batches$Species == "M.cavernosa")
## 
##      AIC      BIC   logLik deviance df.resid 
##      6.0     11.1      0.0      0.0       37 
## 
## Scaled residuals: 
##     Min      1Q  Median      3Q     Max 
##       0       0       0       0 2904892 
## 
## Random effects:
##  Groups Name        Variance Std.Dev.
##  Colony (Intercept) 0        0       
## Number of obs: 40, groups:  Colony, 10
## 
## Fixed effects:
##                Estimate Std. Error z value Pr(>|z|)
## (Intercept)  -3.806e+01  1.501e+07       0        1
## BatchOctober -3.033e+02  2.122e+07       0        1
## 
## Correlation of Fixed Effects:
##             (Intr)
## BatchOctobr -0.707
## optimizer (Nelder_Mead) convergence code: 0 (OK)
## boundary (singular) fit: see ?isSingular
\end{verbatim}

\begin{Shaded}
\begin{Highlighting}[]
\NormalTok{ofavpredmod=}\KeywordTok{glmer}\NormalTok{(pre_propD}\OperatorTok{~}\NormalTok{Batch}\OperatorTok{+}\NormalTok{(}\DecValTok{1}\OperatorTok{|}\NormalTok{Colony),}\DataTypeTok{data=}\KeywordTok{filter}\NormalTok{(batches,batches}\OperatorTok{$}\NormalTok{Species}\OperatorTok{==}\StringTok{'O.faveolata'}\NormalTok{), }\DataTypeTok{family =} \StringTok{'binomial'}\NormalTok{)}
\KeywordTok{summary}\NormalTok{(ofavpredmod) }\CommentTok{# p=0.2}
\end{Highlighting}
\end{Shaded}

\begin{verbatim}
## Generalized linear mixed model fit by maximum likelihood (Laplace
##   Approximation) [glmerMod]
##  Family: binomial  ( logit )
## Formula: pre_propD ~ Batch + (1 | Colony)
##    Data: filter(batches, batches$Species == "O.faveolata")
## 
##      AIC      BIC   logLik deviance df.resid 
##     31.9     36.5    -12.9     25.9       32 
## 
## Scaled residuals: 
##      Min       1Q   Median       3Q      Max 
## -0.76647 -0.04036 -0.00779  0.25178  1.00309 
## 
## Random effects:
##  Groups Name        Variance Std.Dev.
##  Colony (Intercept) 80.35    8.964   
## Number of obs: 35, groups:  Colony, 10
## 
## Fixed effects:
##              Estimate Std. Error z value Pr(>|z|)
## (Intercept)    -6.153      3.793  -1.622    0.105
## BatchOctober   -3.286      2.769  -1.187    0.235
## 
## Correlation of Fixed Effects:
##             (Intr)
## BatchOctobr 0.343
\end{verbatim}

\begin{Shaded}
\begin{Highlighting}[]
\NormalTok{ssidpredmod=}\KeywordTok{glmer}\NormalTok{(pre_propD}\OperatorTok{~}\NormalTok{Batch}\OperatorTok{+}\NormalTok{(}\DecValTok{1}\OperatorTok{|}\NormalTok{Colony),}\DataTypeTok{data=}\KeywordTok{filter}\NormalTok{(batches,batches}\OperatorTok{$}\NormalTok{Species}\OperatorTok{==}\StringTok{'S.siderea'}\NormalTok{), }\DataTypeTok{family =} \StringTok{'binomial'}\NormalTok{)}
\KeywordTok{summary}\NormalTok{(ssidpredmod)}\CommentTok{# p0.09}
\end{Highlighting}
\end{Shaded}

\begin{verbatim}
## Generalized linear mixed model fit by maximum likelihood (Laplace
##   Approximation) [glmerMod]
##  Family: binomial  ( logit )
## Formula: pre_propD ~ Batch + (1 | Colony)
##    Data: filter(batches, batches$Species == "S.siderea")
## 
##      AIC      BIC   logLik deviance df.resid 
##     36.2     41.0    -15.1     30.2       33 
## 
## Scaled residuals: 
##     Min      1Q  Median      3Q     Max 
## -2.9098 -0.3598 -0.0362  0.2155  0.9639 
## 
## Random effects:
##  Groups Name        Variance Std.Dev.
##  Colony (Intercept) 24.15    4.914   
## Number of obs: 36, groups:  Colony, 9
## 
## Fixed effects:
##              Estimate Std. Error z value Pr(>|z|)  
## (Intercept)    -1.039      2.132  -0.487   0.6259  
## BatchOctober    4.593      2.713   1.693   0.0905 .
## ---
## Signif. codes:  0 '***' 0.001 '**' 0.01 '*' 0.05 '.' 0.1 ' ' 1
## 
## Correlation of Fixed Effects:
##             (Intr)
## BatchOctobr -0.425
\end{verbatim}

\begin{Shaded}
\begin{Highlighting}[]
\CommentTok{#UPDATES FOR ROSS: Make it clear that this plot is showing raw data, not predictive model. glmer is now 'family quasibinomial'. }
\end{Highlighting}
\end{Shaded}

\hypertarget{bleaching-sensitivity-rate}{%
\subsection{BLEACHING SENSITIVITY \&
RATE}\label{bleaching-sensitivity-rate}}

\hypertarget{figure-1d-m.-cavernosa-showed-higher-sensitivity-to-bleaching-in-october-compared-to-april.}{%
\subsubsection{Figure 1d: M. cavernosa showed higher sensitivity to
bleaching in October compared to
April.}\label{figure-1d-m.-cavernosa-showed-higher-sensitivity-to-bleaching-in-october-compared-to-april.}}

\begin{Shaded}
\begin{Highlighting}[]
\NormalTok{blch_sensitivity=blch_sensitivity[}\OperatorTok{-}\KeywordTok{c}\NormalTok{(}\DecValTok{31}\NormalTok{,}\DecValTok{32}\NormalTok{,}\DecValTok{48}\NormalTok{,}\DecValTok{59}\NormalTok{,}\DecValTok{56}\NormalTok{,}\DecValTok{57}\NormalTok{,}\DecValTok{75}\NormalTok{),] }\CommentTok{#remove 're'_drop_sh' NAs, and 5 cores that increased symbiont density}
\NormalTok{mcavsensitivity=}\KeywordTok{subset}\NormalTok{(blch_sensitivity,Species}\OperatorTok{==}\StringTok{'M.cavernosa'}\NormalTok{)}
\NormalTok{mcavsensitivity}\OperatorTok{$}\NormalTok{transformedshdrop<-}\StringTok{ }\NormalTok{(mcavsensitivity}\OperatorTok{$}\NormalTok{rel_drop_sh)}\OperatorTok{^}\DecValTok{2}
\KeywordTok{plot}\NormalTok{(mcavsensitivity}\OperatorTok{$}\NormalTok{rel_drop_sh}\OperatorTok{~}\NormalTok{mcavsensitivity}\OperatorTok{$}\NormalTok{rel_drop_y2)}
\end{Highlighting}
\end{Shaded}

\includegraphics{SEASONAL_SHUFFLING_files/figure-latex/unnamed-chunk-7-1.pdf}

\begin{Shaded}
\begin{Highlighting}[]
\NormalTok{mcavblchresmod=}\KeywordTok{glmer}\NormalTok{((rel_drop_sh}\OperatorTok{^}\DecValTok{2}\NormalTok{)}\OperatorTok{~}\NormalTok{rel_drop_y2}\OperatorTok{*}\NormalTok{Batch}\OperatorTok{+}\NormalTok{(}\DecValTok{1}\OperatorTok{|}\NormalTok{Colony),}\DataTypeTok{data=}\NormalTok{mcavsensitivity)}
\KeywordTok{plot_resqq}\NormalTok{(mcavblchresmod)}
\end{Highlighting}
\end{Shaded}

\includegraphics{SEASONAL_SHUFFLING_files/figure-latex/unnamed-chunk-7-2.pdf}

\begin{Shaded}
\begin{Highlighting}[]
\KeywordTok{summary}\NormalTok{(mcavblchresmod)}
\end{Highlighting}
\end{Shaded}

\begin{verbatim}
## Linear mixed model fit by REML ['lmerMod']
## Formula: (rel_drop_sh^2) ~ rel_drop_y2 * Batch + (1 | Colony)
##    Data: mcavsensitivity
## 
## REML criterion at convergence: 363.3
## 
## Scaled residuals: 
##      Min       1Q   Median       3Q      Max 
## -1.90415 -0.22251  0.04061  0.26312  2.78896 
## 
## Random effects:
##  Groups   Name        Variance Std.Dev.
##  Colony   (Intercept)  30138   173.6   
##  Residual             746489   864.0   
## Number of obs: 25, groups:  Colony, 10
## 
## Fixed effects:
##                    Estimate Std. Error t value
## (Intercept)        -5141.38    2432.88  -2.113
## rel_drop_y2          -97.15      34.76  -2.795
## Batch2             15479.84    2674.24   5.789
## rel_drop_y2:Batch2   106.79      40.53   2.635
## 
## Correlation of Fixed Effects:
##             (Intr) rl_d_2 Batch2
## rel_drop_y2  0.993              
## Batch2      -0.912 -0.906       
## rl_drp_2:B2 -0.856 -0.862  0.985
\end{verbatim}

\begin{Shaded}
\begin{Highlighting}[]
\NormalTok{mcavblchresmod2<-}\StringTok{ }\KeywordTok{as_lmerModLmerTest}\NormalTok{(mcavblchresmod)}
\KeywordTok{summary}\NormalTok{(mcavblchresmod2)}\CommentTok{#the interaction between batch and drop in y2 is significant (p=0.0156).}
\end{Highlighting}
\end{Shaded}

\begin{verbatim}
## Linear mixed model fit by REML. t-tests use Satterthwaite's method [
## lmerModLmerTest]
## Formula: (rel_drop_sh^2) ~ rel_drop_y2 * Batch + (1 | Colony)
##    Data: mcavsensitivity
## 
## REML criterion at convergence: 363.3
## 
## Scaled residuals: 
##      Min       1Q   Median       3Q      Max 
## -1.90415 -0.22251  0.04061  0.26312  2.78896 
## 
## Random effects:
##  Groups   Name        Variance Std.Dev.
##  Colony   (Intercept)  30138   173.6   
##  Residual             746489   864.0   
## Number of obs: 25, groups:  Colony, 10
## 
## Fixed effects:
##                    Estimate Std. Error       df t value Pr(>|t|)    
## (Intercept)        -5141.38    2432.88    21.00  -2.113   0.0467 *  
## rel_drop_y2          -97.15      34.76    21.00  -2.795   0.0109 *  
## Batch2             15479.84    2674.24    20.94   5.789 9.67e-06 ***
## rel_drop_y2:Batch2   106.79      40.53    20.73   2.635   0.0156 *  
## ---
## Signif. codes:  0 '***' 0.001 '**' 0.01 '*' 0.05 '.' 0.1 ' ' 1
## 
## Correlation of Fixed Effects:
##             (Intr) rl_d_2 Batch2
## rel_drop_y2  0.993              
## Batch2      -0.912 -0.906       
## rl_drp_2:B2 -0.856 -0.862  0.985
\end{verbatim}

\begin{Shaded}
\begin{Highlighting}[]
\NormalTok{ofavsensitivity<-}\KeywordTok{subset}\NormalTok{(blch_sensitivity,Species}\OperatorTok{==}\StringTok{'O.faveolata'}\NormalTok{)}
\KeywordTok{plot}\NormalTok{(ofavsensitivity}\OperatorTok{$}\NormalTok{rel_drop_sh}\OperatorTok{~}\NormalTok{ofavsensitivity}\OperatorTok{$}\NormalTok{rel_drop_y2)}
\end{Highlighting}
\end{Shaded}

\includegraphics{SEASONAL_SHUFFLING_files/figure-latex/unnamed-chunk-7-3.pdf}

\begin{Shaded}
\begin{Highlighting}[]
\NormalTok{ofavblchresmod=}\KeywordTok{glmer}\NormalTok{((rel_drop_sh}\OperatorTok{^}\DecValTok{2}\NormalTok{)}\OperatorTok{~}\NormalTok{rel_drop_y2}\OperatorTok{*}\NormalTok{Batch}\OperatorTok{+}\NormalTok{InitialDom}\OperatorTok{+}\NormalTok{(}\DecValTok{1}\OperatorTok{|}\NormalTok{Colony),}\DataTypeTok{data=}\NormalTok{ofavsensitivity)}
\KeywordTok{plot_resqq}\NormalTok{(ofavblchresmod)}
\end{Highlighting}
\end{Shaded}

\includegraphics{SEASONAL_SHUFFLING_files/figure-latex/unnamed-chunk-7-4.pdf}

\begin{Shaded}
\begin{Highlighting}[]
\KeywordTok{summary}\NormalTok{(ofavblchresmod)}
\end{Highlighting}
\end{Shaded}

\begin{verbatim}
## Linear mixed model fit by REML ['lmerMod']
## Formula: (rel_drop_sh^2) ~ rel_drop_y2 * Batch + InitialDom + (1 | Colony)
##    Data: ofavsensitivity
## 
## REML criterion at convergence: 335.6
## 
## Scaled residuals: 
##      Min       1Q   Median       3Q      Max 
## -2.48257 -0.28194  0.08217  0.54525  1.33783 
## 
## Random effects:
##  Groups   Name        Variance Std.Dev.
##  Colony   (Intercept) 195290   441.9   
##  Residual             756932   870.0   
## Number of obs: 24, groups:  Colony, 10
## 
## Fixed effects:
##                    Estimate Std. Error t value
## (Intercept)         7700.76    1267.88   6.074
## rel_drop_y2          -21.06      22.82  -0.923
## Batch2              3731.23    2596.95   1.437
## InitialDomnond       469.74     584.43   0.804
## rel_drop_y2:Batch2    64.05      41.25   1.553
## 
## Correlation of Fixed Effects:
##             (Intr) rl_d_2 Batch2 IntlDm
## rel_drop_y2  0.926                     
## Batch2      -0.576 -0.704              
## InitilDmnnd  0.036  0.349 -0.525       
## rl_drp_2:B2 -0.593 -0.742  0.988 -0.539
\end{verbatim}

\begin{Shaded}
\begin{Highlighting}[]
\NormalTok{ofavblchresmod2<-}\StringTok{ }\KeywordTok{as_lmerModLmerTest}\NormalTok{(ofavblchresmod)}
\KeywordTok{summary}\NormalTok{(ofavblchresmod2) }\CommentTok{#when separated by symbiont genus, the interaction between bacth and drop in y2 is insignificant (p=0.137)}
\end{Highlighting}
\end{Shaded}

\begin{verbatim}
## Linear mixed model fit by REML. t-tests use Satterthwaite's method [
## lmerModLmerTest]
## Formula: (rel_drop_sh^2) ~ rel_drop_y2 * Batch + InitialDom + (1 | Colony)
##    Data: ofavsensitivity
## 
## REML criterion at convergence: 335.6
## 
## Scaled residuals: 
##      Min       1Q   Median       3Q      Max 
## -2.48257 -0.28194  0.08217  0.54525  1.33783 
## 
## Random effects:
##  Groups   Name        Variance Std.Dev.
##  Colony   (Intercept) 195290   441.9   
##  Residual             756932   870.0   
## Number of obs: 24, groups:  Colony, 10
## 
## Fixed effects:
##                    Estimate Std. Error      df t value Pr(>|t|)    
## (Intercept)         7700.76    1267.88   18.99   6.074  7.7e-06 ***
## rel_drop_y2          -21.06      22.82   18.40  -0.923    0.368    
## Batch2              3731.23    2596.95   18.97   1.437    0.167    
## InitialDomnond       469.74     584.43   11.22   0.804    0.438    
## rel_drop_y2:Batch2    64.05      41.25   19.00   1.553    0.137    
## ---
## Signif. codes:  0 '***' 0.001 '**' 0.01 '*' 0.05 '.' 0.1 ' ' 1
## 
## Correlation of Fixed Effects:
##             (Intr) rl_d_2 Batch2 IntlDm
## rel_drop_y2  0.926                     
## Batch2      -0.576 -0.704              
## InitilDmnnd  0.036  0.349 -0.525       
## rl_drp_2:B2 -0.593 -0.742  0.988 -0.539
\end{verbatim}

\begin{Shaded}
\begin{Highlighting}[]
\NormalTok{ssidsensitivity<-}\KeywordTok{subset}\NormalTok{(blch_sensitivity,Species}\OperatorTok{==}\StringTok{'S.siderea'}\NormalTok{)}
\KeywordTok{plot}\NormalTok{(ssidsensitivity}\OperatorTok{$}\NormalTok{rel_drop_sh}\OperatorTok{~}\NormalTok{ssidsensitivity}\OperatorTok{$}\NormalTok{rel_drop_y2)}
\end{Highlighting}
\end{Shaded}

\includegraphics{SEASONAL_SHUFFLING_files/figure-latex/unnamed-chunk-7-5.pdf}

\begin{Shaded}
\begin{Highlighting}[]
\NormalTok{ssidblchresmod=}\KeywordTok{glmer}\NormalTok{((rel_drop_sh}\OperatorTok{^}\DecValTok{2}\NormalTok{)}\OperatorTok{~}\NormalTok{rel_drop_y2}\OperatorTok{*}\NormalTok{Batch}\OperatorTok{+}\NormalTok{InitialDom}\OperatorTok{+}\NormalTok{(}\DecValTok{1}\OperatorTok{|}\NormalTok{Colony),}\DataTypeTok{data=}\NormalTok{ssidsensitivity)}
\KeywordTok{plot_resqq}\NormalTok{(ssidblchresmod)}
\end{Highlighting}
\end{Shaded}

\includegraphics{SEASONAL_SHUFFLING_files/figure-latex/unnamed-chunk-7-6.pdf}

\begin{Shaded}
\begin{Highlighting}[]
\KeywordTok{summary}\NormalTok{(ssidblchresmod)}
\end{Highlighting}
\end{Shaded}

\begin{verbatim}
## Linear mixed model fit by REML ['lmerMod']
## Formula: (rel_drop_sh^2) ~ rel_drop_y2 * Batch + InitialDom + (1 | Colony)
##    Data: ssidsensitivity
## 
## REML criterion at convergence: 269.4
## 
## Scaled residuals: 
##     Min      1Q  Median      3Q     Max 
## -1.7786 -0.3998  0.2009  0.6135  1.1655 
## 
## Random effects:
##  Groups   Name        Variance Std.Dev.
##  Colony   (Intercept)      0     0.0   
##  Residual             868824   932.1   
## Number of obs: 20, groups:  Colony, 9
## 
## Fixed effects:
##                    Estimate Std. Error t value
## (Intercept)         8273.79    1130.84   7.316
## rel_drop_y2          -15.86      22.12  -0.717
## Batch2              1418.57    1413.00   1.004
## InitialDomnond       218.95     459.78   0.476
## rel_drop_y2:Batch2    21.21      27.17   0.781
## 
## Correlation of Fixed Effects:
##             (Intr) rl_d_2 Batch2 IntlDm
## rel_drop_y2  0.942                     
## Batch2      -0.747 -0.757              
## InitilDmnnd -0.224  0.014 -0.060       
## rl_drp_2:B2 -0.733 -0.816  0.947 -0.163
## optimizer (nloptwrap) convergence code: 0 (OK)
## boundary (singular) fit: see ?isSingular
\end{verbatim}

\begin{Shaded}
\begin{Highlighting}[]
\NormalTok{ssidblchresmod2<-}\StringTok{ }\KeywordTok{as_lmerModLmerTest}\NormalTok{(ssidblchresmod)}
\KeywordTok{summary}\NormalTok{(ssidblchresmod2) }\CommentTok{#when separated by symbiont genus, the interaction between bacth and drop in y2 is insignificant (p=0.447)}
\end{Highlighting}
\end{Shaded}

\begin{verbatim}
## Linear mixed model fit by REML. t-tests use Satterthwaite's method [
## lmerModLmerTest]
## Formula: (rel_drop_sh^2) ~ rel_drop_y2 * Batch + InitialDom + (1 | Colony)
##    Data: ssidsensitivity
## 
## REML criterion at convergence: 269.4
## 
## Scaled residuals: 
##     Min      1Q  Median      3Q     Max 
## -1.7786 -0.3998  0.2009  0.6135  1.1655 
## 
## Random effects:
##  Groups   Name        Variance Std.Dev.
##  Colony   (Intercept)      0     0.0   
##  Residual             868824   932.1   
## Number of obs: 20, groups:  Colony, 9
## 
## Fixed effects:
##                    Estimate Std. Error      df t value Pr(>|t|)    
## (Intercept)         8273.79    1130.84   15.00   7.316 2.54e-06 ***
## rel_drop_y2          -15.86      22.12   15.00  -0.717    0.484    
## Batch2              1418.57    1413.00   15.00   1.004    0.331    
## InitialDomnond       218.95     459.78   15.00   0.476    0.641    
## rel_drop_y2:Batch2    21.21      27.17   15.00   0.781    0.447    
## ---
## Signif. codes:  0 '***' 0.001 '**' 0.01 '*' 0.05 '.' 0.1 ' ' 1
## 
## Correlation of Fixed Effects:
##             (Intr) rl_d_2 Batch2 IntlDm
## rel_drop_y2  0.942                     
## Batch2      -0.747 -0.757              
## InitilDmnnd -0.224  0.014 -0.060       
## rl_drp_2:B2 -0.733 -0.816  0.947 -0.163
## optimizer (nloptwrap) convergence code: 0 (OK)
## boundary (singular) fit: see ?isSingular
\end{verbatim}

\begin{Shaded}
\begin{Highlighting}[]
\NormalTok{mcavsensitivity}\OperatorTok{$}\NormalTok{predicted<-}\StringTok{ }\KeywordTok{predict}\NormalTok{(mcavblchresmod)}
\NormalTok{mcavsensitivity}\OperatorTok{$}\NormalTok{residuals<-}\StringTok{ }\KeywordTok{residuals}\NormalTok{(mcavblchresmod)}
\NormalTok{mcavsensitivity}\OperatorTok{$}\NormalTok{untransformed_predicted<-}\StringTok{ }\OperatorTok{-}\NormalTok{(mcavsensitivity}\OperatorTok{$}\NormalTok{predicted}\OperatorTok{^}\FloatTok{0.5}\NormalTok{) }\CommentTok{#all changes are negative, reverse transform sqaured response variable.   }

\KeywordTok{ggplot}\NormalTok{()}\OperatorTok{+}
\StringTok{  }\KeywordTok{geom_point}\NormalTok{(}\DataTypeTok{data=}\NormalTok{mcavsensitivity,}\KeywordTok{aes}\NormalTok{(}\DataTypeTok{x=}\NormalTok{rel_drop_y2,}\DataTypeTok{y=}\NormalTok{rel_drop_sh,}\DataTypeTok{colour=}\NormalTok{Batch))}\OperatorTok{+}
\StringTok{  }\KeywordTok{geom_smooth}\NormalTok{(}\DataTypeTok{data=}\NormalTok{mcavsensitivity,}\DataTypeTok{method=}\StringTok{'loess'}\NormalTok{,}
  \KeywordTok{aes}\NormalTok{(}\DataTypeTok{x=}\NormalTok{rel_drop_y2,}\DataTypeTok{y=}\NormalTok{untransformed_predicted,}\DataTypeTok{color=}\NormalTok{Batch),}\DataTypeTok{show.legend=}\NormalTok{F,}\DataTypeTok{se=}\NormalTok{T, }\DataTypeTok{alpha=}\FloatTok{0.5}\NormalTok{, }\DataTypeTok{size=}\FloatTok{0.5}\NormalTok{)}\OperatorTok{+}
\StringTok{  }\KeywordTok{coord_cartesian}\NormalTok{(}\DataTypeTok{clip=}\StringTok{'off'}\NormalTok{, }\DataTypeTok{ylim=}\KeywordTok{c}\NormalTok{(}\OperatorTok{-}\DecValTok{120}\NormalTok{,}\DecValTok{0}\NormalTok{))}\OperatorTok{+}
\StringTok{  }\KeywordTok{theme_minimal}\NormalTok{(}\DataTypeTok{base_size =} \DecValTok{13}\NormalTok{)}\OperatorTok
\StringTok{    }\KeywordTok{theme}\NormalTok{(}\DataTypeTok{panel.border =} \KeywordTok{element_rect}\NormalTok{(}\DataTypeTok{colour=}\StringTok{'grey20'}\NormalTok{,}\DataTypeTok{size=}\FloatTok{0.5}\NormalTok{,}\DataTypeTok{fill=}\OtherTok{NA}\NormalTok{))}\OperatorTok{+}
\StringTok{   }\KeywordTok{scale_colour_manual}\NormalTok{(}\DataTypeTok{values=}\KeywordTok{c}\NormalTok{(}\StringTok{'#3C5488B2'}\NormalTok{,}\StringTok{'#00A087B2'}\NormalTok{),}\DataTypeTok{labels=}\KeywordTok{c}\NormalTok{(}\StringTok{'April'}\NormalTok{,}\StringTok{'October'}\NormalTok{))}\OperatorTok{+}
\StringTok{ }\KeywordTok{theme}\NormalTok{(}\DataTypeTok{panel.spacing =} \KeywordTok{unit}\NormalTok{(}\FloatTok{1.2}\NormalTok{, }\StringTok{"lines"}\NormalTok{),}\DataTypeTok{legend.position =} \StringTok{'right'}\NormalTok{)}\OperatorTok{+}
\StringTok{  }\KeywordTok{theme}\NormalTok{(}\DataTypeTok{legend.title=}\KeywordTok{element_text}\NormalTok{(}\DataTypeTok{size=}\DecValTok{0}\NormalTok{), }\DataTypeTok{axis.title.y =} \KeywordTok{element_markdown}\NormalTok{())}\OperatorTok{+}
\KeywordTok{theme}\NormalTok{(}\DataTypeTok{legend.position =} \KeywordTok{c}\NormalTok{(}\FloatTok{0.8}\NormalTok{,}\FloatTok{0.5}\NormalTok{))}\OperatorTok{+}
\StringTok{  }\KeywordTok{labs}\NormalTok{(}\DataTypeTok{x=}\StringTok{'% change in Fv/Fm'}\NormalTok{,}\DataTypeTok{y=}\StringTok{'% change in symbionts per *M. cavernosa* cell'}\NormalTok{)}
\end{Highlighting}
\end{Shaded}

\includegraphics{SEASONAL_SHUFFLING_files/figure-latex/unnamed-chunk-7-7.pdf}

\begin{Shaded}
\begin{Highlighting}[]
  \CommentTok{#ggsave('bleaching_sensitivity_batches.pdf',device='pdf',width=7,height=5)}


\CommentTok{#UPDATES FOR ROSS: This model is now also a mixed effects model, performed on transformed response data. A glmm was used instead of a lmm due to uneven sample sizes of from each colony in each treatemnt due to exclusion of control cores. This now shows the linear regression and 95% confidence interval for predicted values from the model. The plot shows MCAV only but stats have been done for all three species. }


\CommentTok{#what if we try plotting end fv/fm and s:h rather than the relative change? And use prediction ellipses or mean and error bars rather than try to fit a linear model. }

\KeywordTok{ggplot}\NormalTok{()}\OperatorTok{+}
\StringTok{  }\KeywordTok{geom_point}\NormalTok{(}\DataTypeTok{data=}\NormalTok{mcavsensitivity,}\KeywordTok{aes}\NormalTok{(}\DataTypeTok{x=}\NormalTok{post_y2,}\DataTypeTok{y=}\NormalTok{post_sh,}\DataTypeTok{colour=}\NormalTok{Batch))}\OperatorTok{+}
\StringTok{  }\KeywordTok{theme_minimal}\NormalTok{(}\DataTypeTok{base_size =} \DecValTok{13}\NormalTok{)}\OperatorTok
\StringTok{    }\KeywordTok{theme}\NormalTok{(}\DataTypeTok{panel.border =} \KeywordTok{element_rect}\NormalTok{(}\DataTypeTok{colour=}\StringTok{'grey20'}\NormalTok{,}\DataTypeTok{size=}\FloatTok{0.5}\NormalTok{,}\DataTypeTok{fill=}\OtherTok{NA}\NormalTok{))}\OperatorTok{+}
\StringTok{   }\KeywordTok{scale_colour_manual}\NormalTok{(}\DataTypeTok{values=}\KeywordTok{c}\NormalTok{(}\StringTok{'#3C5488B2'}\NormalTok{,}\StringTok{'#00A087B2'}\NormalTok{),}\DataTypeTok{labels=}\KeywordTok{c}\NormalTok{(}\StringTok{'April'}\NormalTok{,}\StringTok{'October'}\NormalTok{))}\OperatorTok{+}
\StringTok{ }\KeywordTok{theme}\NormalTok{(}\DataTypeTok{panel.spacing =} \KeywordTok{unit}\NormalTok{(}\FloatTok{1.2}\NormalTok{, }\StringTok{"lines"}\NormalTok{),}\DataTypeTok{legend.position =} \StringTok{'right'}\NormalTok{)}\OperatorTok{+}
\StringTok{  }\KeywordTok{theme}\NormalTok{(}\DataTypeTok{legend.title=}\KeywordTok{element_text}\NormalTok{(}\DataTypeTok{size=}\DecValTok{0}\NormalTok{), }\DataTypeTok{axis.title.y =} \KeywordTok{element_markdown}\NormalTok{())}\OperatorTok{+}
\KeywordTok{theme}\NormalTok{(}\DataTypeTok{legend.position =} \KeywordTok{c}\NormalTok{(}\FloatTok{0.8}\NormalTok{,}\FloatTok{0.5}\NormalTok{))}\OperatorTok{+}
\StringTok{  }\KeywordTok{labs}\NormalTok{(}\DataTypeTok{x=}\StringTok{'Fv/Fm after heat stress'}\NormalTok{,}\DataTypeTok{y=}\StringTok{'symbionts per *M. cavernosa* cell after heat stress'}\NormalTok{)}
\end{Highlighting}
\end{Shaded}

\includegraphics{SEASONAL_SHUFFLING_files/figure-latex/unnamed-chunk-7-8.pdf}

\begin{Shaded}
\begin{Highlighting}[]
\KeywordTok{ggplot}\NormalTok{()}\OperatorTok{+}
\StringTok{  }\KeywordTok{geom_point}\NormalTok{(}\DataTypeTok{data=}\NormalTok{mcavsensitivity,}\KeywordTok{aes}\NormalTok{(}\DataTypeTok{x=}\NormalTok{change_y2,}\DataTypeTok{y=}\NormalTok{post_sh,}\DataTypeTok{colour=}\NormalTok{Batch))}\OperatorTok{+}
\StringTok{  }\KeywordTok{theme_minimal}\NormalTok{(}\DataTypeTok{base_size =} \DecValTok{13}\NormalTok{)}\OperatorTok
\StringTok{    }\KeywordTok{theme}\NormalTok{(}\DataTypeTok{panel.border =} \KeywordTok{element_rect}\NormalTok{(}\DataTypeTok{colour=}\StringTok{'grey20'}\NormalTok{,}\DataTypeTok{size=}\FloatTok{0.5}\NormalTok{,}\DataTypeTok{fill=}\OtherTok{NA}\NormalTok{))}\OperatorTok{+}
\StringTok{   }\KeywordTok{scale_colour_manual}\NormalTok{(}\DataTypeTok{values=}\KeywordTok{c}\NormalTok{(}\StringTok{'#3C5488B2'}\NormalTok{,}\StringTok{'#00A087B2'}\NormalTok{),}\DataTypeTok{labels=}\KeywordTok{c}\NormalTok{(}\StringTok{'April'}\NormalTok{,}\StringTok{'October'}\NormalTok{))}\OperatorTok{+}
\StringTok{ }\KeywordTok{theme}\NormalTok{(}\DataTypeTok{panel.spacing =} \KeywordTok{unit}\NormalTok{(}\FloatTok{1.2}\NormalTok{, }\StringTok{"lines"}\NormalTok{),}\DataTypeTok{legend.position =} \StringTok{'right'}\NormalTok{)}\OperatorTok{+}
\StringTok{  }\KeywordTok{theme}\NormalTok{(}\DataTypeTok{legend.title=}\KeywordTok{element_text}\NormalTok{(}\DataTypeTok{size=}\DecValTok{0}\NormalTok{), }\DataTypeTok{axis.title.y =} \KeywordTok{element_markdown}\NormalTok{())}\OperatorTok{+}
\KeywordTok{theme}\NormalTok{(}\DataTypeTok{legend.position =} \KeywordTok{c}\NormalTok{(}\FloatTok{0.8}\NormalTok{,}\FloatTok{0.5}\NormalTok{))}\OperatorTok{+}
\StringTok{  }\KeywordTok{labs}\NormalTok{(}\DataTypeTok{x=}\StringTok{'reduction in Fv/Fm'}\NormalTok{,}\DataTypeTok{y=}\StringTok{'symbionts per *M. cavernosa* cell after heat stress'}\NormalTok{)}
\end{Highlighting}
\end{Shaded}

\includegraphics{SEASONAL_SHUFFLING_files/figure-latex/unnamed-chunk-7-9.pdf}

\begin{Shaded}
\begin{Highlighting}[]
\NormalTok{allbleach=}\KeywordTok{rbind}\NormalTok{(ofav_DHW,ssid_DHW,mcav_DHW)}
\NormalTok{allbleach}\OperatorTok{$}\NormalTok{Species=}\KeywordTok{factor}\NormalTok{(allbleach}\OperatorTok{$}\NormalTok{Species,}\DataTypeTok{levels=}\KeywordTok{c}\NormalTok{(}\StringTok{'M.cavernosa'}\NormalTok{,}\StringTok{'O.faveolata'}\NormalTok{,}\StringTok{'S.siderea'}\NormalTok{))}
\NormalTok{allrecov=}\KeywordTok{rbind}\NormalTok{(mcav_recov, ofav_recov, ssid_recov)}
\NormalTok{allrecov}\OperatorTok{$}\NormalTok{Species=}\KeywordTok{factor}\NormalTok{(allrecov}\OperatorTok{$}\NormalTok{Species,}\DataTypeTok{levels=}\KeywordTok{c}\NormalTok{(}\StringTok{'M.cavernosa'}\NormalTok{,}\StringTok{'O.faveolata'}\NormalTok{,}\StringTok{'S.siderea'}\NormalTok{))}
\NormalTok{allrecov}\OperatorTok{$}\NormalTok{InitialDom=}\KeywordTok{revalue}\NormalTok{(allrecov}\OperatorTok{$}\NormalTok{InitialDom,}\KeywordTok{c}\NormalTok{(}\StringTok{'D'}\NormalTok{=}\StringTok{'d'}\NormalTok{,}\StringTok{'B'}\NormalTok{=}\StringTok{'nond'}\NormalTok{,}\StringTok{'C'}\NormalTok{=}\StringTok{'nond'}\NormalTok{))}
\NormalTok{allrecov}\OperatorTok{$}\NormalTok{InitialDom=}\KeywordTok{factor}\NormalTok{(allrecov}\OperatorTok{$}\NormalTok{InitialDom)}
\NormalTok{allbleach}\OperatorTok{$}\NormalTok{InitialDom=}\KeywordTok{revalue}\NormalTok{(allbleach}\OperatorTok{$}\NormalTok{InitialDom,}\KeywordTok{c}\NormalTok{(}\StringTok{'D'}\NormalTok{=}\StringTok{'d'}\NormalTok{,}\StringTok{'B'}\NormalTok{=}\StringTok{'nond'}\NormalTok{,}\StringTok{'C'}\NormalTok{=}\StringTok{'nond'}\NormalTok{))}
\NormalTok{allbleach}\OperatorTok{$}\NormalTok{InitialDom=}\KeywordTok{factor}\NormalTok{(allbleach}\OperatorTok{$}\NormalTok{InitialDom)}
\NormalTok{allbleach}\OperatorTok{$}\NormalTok{Batch=}\KeywordTok{factor}\NormalTok{(allbleach}\OperatorTok{$}\NormalTok{Batch)}
\NormalTok{allrecov}\OperatorTok{$}\NormalTok{Batch=}\KeywordTok{factor}\NormalTok{(allrecov}\OperatorTok{$}\NormalTok{Batch)}
\NormalTok{allbleach}\OperatorTok{$}\NormalTok{Colony=}\KeywordTok{factor}\NormalTok{(allbleach}\OperatorTok{$}\NormalTok{Colony)}
\NormalTok{allrecov}\OperatorTok{$}\NormalTok{Colony=}\KeywordTok{factor}\NormalTok{(allrecov}\OperatorTok{$}\NormalTok{Colony)}


\NormalTok{allbleach<-allbleach[}\OperatorTok{-}\KeywordTok{c}\NormalTok{(}\DecValTok{167}\NormalTok{),] }
\NormalTok{allrecov<-allrecov[}\OperatorTok{-}\KeywordTok{c}\NormalTok{(}\DecValTok{86}\NormalTok{,}\DecValTok{239}\NormalTok{),]}\CommentTok{#model highlighted this one datapoint as an outlier in both dataframes}
  

\NormalTok{bleachingmod2<-}\KeywordTok{glmer}\NormalTok{(Shprop}\OperatorTok{~}\NormalTok{DHW}\OperatorTok{*}\NormalTok{Species}\OperatorTok{*}\NormalTok{InitialDom}\OperatorTok{+}\NormalTok{(}\DecValTok{1}\OperatorTok{|}\NormalTok{Colony), }\DataTypeTok{data=}\NormalTok{allbleach)}
\KeywordTok{plot_resqq}\NormalTok{(bleachingmod2) }\CommentTok{#perform statistical tests on mixed effects model}
\end{Highlighting}
\end{Shaded}

\includegraphics{SEASONAL_SHUFFLING_files/figure-latex/unnamed-chunk-9-1.pdf}

\begin{Shaded}
\begin{Highlighting}[]
\NormalTok{bleachingmod3<-}\KeywordTok{as_lmerModLmerTest}\NormalTok{(bleachingmod2)}
  \KeywordTok{summary}\NormalTok{(bleachingmod3)}
\end{Highlighting}
\end{Shaded}

\begin{verbatim}
## Linear mixed model fit by REML. t-tests use Satterthwaite's method [
## lmerModLmerTest]
## Formula: Shprop ~ DHW * Species * InitialDom + (1 | Colony)
##    Data: allbleach
## 
## REML criterion at convergence: 62.9
## 
## Scaled residuals: 
##     Min      1Q  Median      3Q     Max 
## -2.1750 -0.4013 -0.0358  0.3297  5.1995 
## 
## Random effects:
##  Groups   Name        Variance Std.Dev.
##  Colony   (Intercept) 0.006468 0.08042 
##  Residual             0.060175 0.24531 
## Number of obs: 147, groups:  Colony, 29
## 
## Fixed effects:
##                                         Estimate Std. Error         df t value
## (Intercept)                             1.054054   0.125491  77.511446   8.399
## DHW                                    -0.059963   0.017904 120.365702  -3.349
## SpeciesO.faveolata                     -0.057977   0.159516  67.688528  -0.363
## SpeciesS.siderea                       -0.099000   0.102079  67.803938  -0.970
## InitialDomnond                         -0.023200   0.113848  86.620959  -0.204
## DHW:SpeciesO.faveolata                 -0.009859   0.020228 119.185291  -0.487
## DHW:SpeciesS.siderea                    0.002114   0.016147 121.484396   0.131
## DHW:InitialDomnond                     -0.044485   0.015250 121.587128  -2.917
## SpeciesO.faveolata:InitialDomnond      -0.033899   0.164764  68.629139  -0.206
## DHW:SpeciesO.faveolata:InitialDomnond   0.023099   0.019905 120.391812   1.161
##                                       Pr(>|t|)    
## (Intercept)                           1.65e-12 ***
## DHW                                    0.00108 ** 
## SpeciesO.faveolata                     0.71740    
## SpeciesS.siderea                       0.33557    
## InitialDomnond                         0.83901    
## DHW:SpeciesO.faveolata                 0.62690    
## DHW:SpeciesS.siderea                   0.89605    
## DHW:InitialDomnond                     0.00421 ** 
## SpeciesO.faveolata:InitialDomnond      0.83760    
## DHW:SpeciesO.faveolata:InitialDomnond  0.24814    
## ---
## Signif. codes:  0 '***' 0.001 '**' 0.01 '*' 0.05 '.' 0.1 ' ' 1
## 
## Correlation of Fixed Effects:
##             (Intr) DHW    SpcsO. SpcsS. IntlDm DHW:SpO. DHW:SS DHW:ID SO.:ID
## DHW         -0.592                                                          
## SpecsO.fvlt -0.787  0.466                                                   
## SpecisS.sdr -0.798  0.536  0.627                                            
## InitilDmnnd -0.907  0.507  0.714  0.639                                     
## DHW:SpcsO.f  0.524 -0.885 -0.590 -0.475 -0.449                              
## DHW:SpcsS.s  0.482 -0.901 -0.379 -0.597 -0.370  0.798                       
## DHW:IntlDmn  0.540 -0.852 -0.425 -0.440 -0.596  0.754    0.701              
## SpcsO.fv:ID  0.627 -0.351 -0.862 -0.442 -0.691  0.483    0.256  0.412       
## DHW:SpO.:ID -0.414  0.653  0.507  0.337  0.456 -0.798   -0.537 -0.766 -0.597
## fit warnings:
## fixed-effect model matrix is rank deficient so dropping 2 columns / coefficients
\end{verbatim}

\begin{Shaded}
\begin{Highlighting}[]
  \KeywordTok{anova}\NormalTok{(bleachingmod3)}\CommentTok{#no significant difference in bleaching (DHW interaction with species) between species p=0.818834, but a significant interaction between DHW and initial symbiont type (p=0.00421), with nond-hosting corals bleaching more. }
\end{Highlighting}
\end{Shaded}

\begin{verbatim}
## Type III Analysis of Variance Table with Satterthwaite's method
##                         Sum Sq Mean Sq NumDF   DenDF  F value    Pr(>F)    
## DHW                    22.5923 22.5923     1 118.404 375.4414 < 2.2e-16 ***
## Species                 0.0952  0.0476     2  56.230   0.7907  0.458514    
## InitialDom              0.0143  0.0143     1  68.629   0.2375  0.627558    
## DHW:Species             0.0241  0.0120     2 120.106   0.2002  0.818834    
## DHW:InitialDom          0.6590  0.6590     1 120.392  10.9518  0.001234 ** 
## Species:InitialDom      0.0025  0.0025     1  68.629   0.0423  0.837599    
## DHW:Species:InitialDom  0.0810  0.0810     1 120.392   1.3468  0.248139    
## ---
## Signif. codes:  0 '***' 0.001 '**' 0.01 '*' 0.05 '.' 0.1 ' ' 1
\end{verbatim}

\begin{Shaded}
\begin{Highlighting}[]
\CommentTok{#now looking at any batch differences within each coral species:}
\NormalTok{  mcavbleachingmod<-}\KeywordTok{glmer}\NormalTok{(Shprop}\OperatorTok{~}\NormalTok{DHW}\OperatorTok{*}\NormalTok{Batch}\OperatorTok{+}\NormalTok{(}\DecValTok{1}\OperatorTok{|}\NormalTok{Colony), }\DataTypeTok{data=}\KeywordTok{filter}\NormalTok{(allbleach,allbleach}\OperatorTok{$}\NormalTok{Species}\OperatorTok{==}\StringTok{'M.cavernosa'}\NormalTok{))}
  \KeywordTok{plot_resqq}\NormalTok{(mcavbleachingmod)}
\end{Highlighting}
\end{Shaded}

\includegraphics{SEASONAL_SHUFFLING_files/figure-latex/unnamed-chunk-9-2.pdf}

\begin{Shaded}
\begin{Highlighting}[]
\NormalTok{  mcavbleachingmod2<-}\KeywordTok{as_lmerModLmerTest}\NormalTok{(mcavbleachingmod)}
  \KeywordTok{summary}\NormalTok{(mcavbleachingmod2)}
\end{Highlighting}
\end{Shaded}

\begin{verbatim}
## Linear mixed model fit by REML. t-tests use Satterthwaite's method [
## lmerModLmerTest]
## Formula: Shprop ~ DHW * Batch + (1 | Colony)
##    Data: filter(allbleach, allbleach$Species == "M.cavernosa")
## 
## REML criterion at convergence: -5.1
## 
## Scaled residuals: 
##     Min      1Q  Median      3Q     Max 
## -2.4227 -0.0828  0.0082  0.0650  5.2633 
## 
## Random effects:
##  Groups   Name        Variance Std.Dev.
##  Colony   (Intercept) 0.00000  0.0000  
##  Residual             0.03716  0.1928  
## Number of obs: 55, groups:  Colony, 10
## 
## Fixed effects:
##             Estimate Std. Error       df t value Pr(>|t|)    
## (Intercept)  1.01595    0.05271 51.00000  19.274  < 2e-16 ***
## DHW         -0.04102    0.01307 51.00000  -3.138  0.00283 ** 
## Batch2      -0.02849    0.07227 51.00000  -0.394  0.69507    
## DHW:Batch2  -0.08481    0.01590 51.00000  -5.335 2.21e-06 ***
## ---
## Signif. codes:  0 '***' 0.001 '**' 0.01 '*' 0.05 '.' 0.1 ' ' 1
## 
## Correlation of Fixed Effects:
##            (Intr) DHW    Batch2
## DHW        -0.682              
## Batch2     -0.729  0.497       
## DHW:Batch2  0.561 -0.822 -0.682
## optimizer (nloptwrap) convergence code: 0 (OK)
## boundary (singular) fit: see ?isSingular
\end{verbatim}

\begin{Shaded}
\begin{Highlighting}[]
  \KeywordTok{anova}\NormalTok{(mcavbleachingmod2) }\CommentTok{#significant interaction between DHW and batch on proportion of symbionts retained, p=2.21e-06, with the october batch losing more.}
\end{Highlighting}
\end{Shaded}

\begin{verbatim}
## Type III Analysis of Variance Table with Satterthwaite's method
##           Sum Sq Mean Sq NumDF DenDF  F value    Pr(>F)    
## DHW       4.0937  4.0937     1    51 110.1738 2.410e-14 ***
## Batch     0.0058  0.0058     1    51   0.1554    0.6951    
## DHW:Batch 1.0577  1.0577     1    51  28.4652 2.207e-06 ***
## ---
## Signif. codes:  0 '***' 0.001 '**' 0.01 '*' 0.05 '.' 0.1 ' ' 1
\end{verbatim}

\begin{Shaded}
\begin{Highlighting}[]
\NormalTok{recovmod<-}\KeywordTok{glm}\NormalTok{(Shprop}\OperatorTok{~}\NormalTok{Recov.Days}\OperatorTok{*}\NormalTok{Species}\OperatorTok{*}\NormalTok{InitialDom, }\DataTypeTok{data=}\NormalTok{allrecov, }\DataTypeTok{family=}\StringTok{'quasipoisson'}\NormalTok{)}
\KeywordTok{plot}\NormalTok{(recovmod)}
\end{Highlighting}
\end{Shaded}

\includegraphics{SEASONAL_SHUFFLING_files/figure-latex/unnamed-chunk-9-3.pdf}
\includegraphics{SEASONAL_SHUFFLING_files/figure-latex/unnamed-chunk-9-4.pdf}
\includegraphics{SEASONAL_SHUFFLING_files/figure-latex/unnamed-chunk-9-5.pdf}
\includegraphics{SEASONAL_SHUFFLING_files/figure-latex/unnamed-chunk-9-6.pdf}

\begin{Shaded}
\begin{Highlighting}[]
\NormalTok{recoveryfit=}\StringTok{ }\KeywordTok{with}\NormalTok{(}\KeywordTok{summary}\NormalTok{(recovmod), }\DecValTok{1} \OperatorTok{-}\StringTok{ }\NormalTok{deviance}\OperatorTok{/}\NormalTok{null.deviance)}
\NormalTok{recoveryfit }\CommentTok{#0.5469, R^2 for plotted quasipoisson model (without batch)}
\end{Highlighting}
\end{Shaded}

\begin{verbatim}
## [1] 0.5469235
\end{verbatim}

\begin{Shaded}
\begin{Highlighting}[]
\NormalTok{recovmod2<-}\KeywordTok{glmer}\NormalTok{(Shprop}\OperatorTok{~}\NormalTok{Recov.Days}\OperatorTok{*}\NormalTok{Species}\OperatorTok{*}\NormalTok{InitialDom}\OperatorTok{+}\NormalTok{(}\DecValTok{1}\OperatorTok{|}\NormalTok{Colony), }\DataTypeTok{data=}\NormalTok{allrecov)}
\KeywordTok{plot_resqq}\NormalTok{(recovmod2)}
\end{Highlighting}
\end{Shaded}

\includegraphics{SEASONAL_SHUFFLING_files/figure-latex/unnamed-chunk-9-7.pdf}

\begin{Shaded}
\begin{Highlighting}[]
\NormalTok{recovmod3<-}\KeywordTok{as_lmerModLmerTest}\NormalTok{(recovmod2)}
\KeywordTok{summary}\NormalTok{(recovmod3)}
\end{Highlighting}
\end{Shaded}

\begin{verbatim}
## Linear mixed model fit by REML. t-tests use Satterthwaite's method [
## lmerModLmerTest]
## Formula: Shprop ~ Recov.Days * Species * InitialDom + (1 | Colony)
##    Data: allrecov
## 
## REML criterion at convergence: 1250.9
## 
## Scaled residuals: 
##     Min      1Q  Median      3Q     Max 
## -2.7901 -0.2597 -0.0424  0.0247 10.8057 
## 
## Random effects:
##  Groups   Name        Variance Std.Dev.
##  Colony   (Intercept)  0.1652  0.4064  
##  Residual             17.4242  4.1742  
## Number of obs: 217, groups:  Colony, 29
## 
## Fixed effects:
##                                               Estimate Std. Error        df
## (Intercept)                                    3.67046    1.79581 120.33950
## Recov.Days                                    -0.12507    0.03986 188.75632
## SpeciesO.faveolata                            -3.00438    2.22969 110.98030
## SpeciesS.siderea                              -3.44919    1.48290 124.40494
## InitialDomnond                                -3.65091    1.65218 130.04791
## Recov.Days:SpeciesO.faveolata                  0.14223    0.05070 188.65474
## Recov.Days:SpeciesS.siderea                    0.15282    0.03292 187.61232
## Recov.Days:InitialDomnond                      0.16279    0.03721 188.75877
## SpeciesO.faveolata:InitialDomnond              2.96527    2.31684 114.40581
## Recov.Days:SpeciesO.faveolata:InitialDomnond  -0.17354    0.05372 188.60438
##                                              t value Pr(>|t|)    
## (Intercept)                                    2.044  0.04315 *  
## Recov.Days                                    -3.138  0.00197 ** 
## SpeciesO.faveolata                            -1.347  0.18058    
## SpeciesS.siderea                              -2.326  0.02164 *  
## InitialDomnond                                -2.210  0.02887 *  
## Recov.Days:SpeciesO.faveolata                  2.805  0.00556 ** 
## Recov.Days:SpeciesS.siderea                    4.642 6.48e-06 ***
## Recov.Days:InitialDomnond                      4.376 2.00e-05 ***
## SpeciesO.faveolata:InitialDomnond              1.280  0.20318    
## Recov.Days:SpeciesO.faveolata:InitialDomnond  -3.231  0.00146 ** 
## ---
## Signif. codes:  0 '***' 0.001 '**' 0.01 '*' 0.05 '.' 0.1 ' ' 1
## 
## Correlation of Fixed Effects:
##             (Intr) Rcv.Dy SpcsO. SpcsS. IntlDm Rc.D:SO. R.D:SS R.D:ID SO.:ID
## Recov.Days  -0.748                                                          
## SpecsO.fvlt -0.805  0.603                                                   
## SpecisS.sdr -0.825  0.625  0.664                                            
## InitilDmnnd -0.920  0.702  0.741  0.694                                     
## Rcv.Dys:SO.  0.588 -0.786 -0.746 -0.491 -0.552                              
## Rcv.Dys:SS.  0.624 -0.826 -0.503 -0.756 -0.543  0.649                       
## Rcv.Dys:InD  0.692 -0.933 -0.557 -0.536 -0.752  0.734    0.718              
## SpcsO.fv:ID  0.656 -0.500 -0.867 -0.495 -0.713  0.655    0.387  0.536       
## Rc.D:SO.:ID -0.479  0.646  0.643  0.371  0.521 -0.869   -0.497 -0.693 -0.746
## fit warnings:
## fixed-effect model matrix is rank deficient so dropping 2 columns / coefficients
\end{verbatim}

\begin{Shaded}
\begin{Highlighting}[]
\KeywordTok{anova}\NormalTok{(recovmod3)}\CommentTok{# symbiont recovery was significantly differnet between species (recovery days * symbiont density interaction), with ofav p=0.00556 and ssid p=6.48e-06 recovering more than mcav for a given amount of recovery (likely indicative of the delay in symbiont recovery seen in mcav). There was also a significant interaction between initial symbiont genus and recovery days p=2.00e-05, with corals initially not hosting Durusdinium recovering more with a given amount of recovery. }
\end{Highlighting}
\end{Shaded}

\begin{verbatim}
## Type III Analysis of Variance Table with Satterthwaite's method
##                               Sum Sq Mean Sq NumDF  DenDF F value    Pr(>F)    
## Recov.Days                    336.61  336.61     1 188.77 19.3184 1.844e-05 ***
## Species                        81.15   40.57     2 103.20  2.3286 0.1025280    
## InitialDom                     61.05   61.05     1 114.41  3.5035 0.0637941 .  
## Recov.Days:Species            304.78  152.39     2 188.33  8.7458 0.0002333 ***
## Recov.Days:InitialDom         139.59  139.59     1 188.60  8.0111 0.0051531 ** 
## Species:InitialDom             28.54   28.54     1 114.41  1.6381 0.2031778    
## Recov.Days:Species:InitialDom 181.85  181.85     1 188.60 10.4364 0.0014578 ** 
## ---
## Signif. codes:  0 '***' 0.001 '**' 0.01 '*' 0.05 '.' 0.1 ' ' 1
\end{verbatim}

\begin{Shaded}
\begin{Highlighting}[]
\NormalTok{    mcavrecovmod<-}\KeywordTok{glmer}\NormalTok{(Shprop}\OperatorTok{~}\NormalTok{Recov.Days}\OperatorTok{*}\NormalTok{Batch}\OperatorTok{+}\NormalTok{(}\DecValTok{1}\OperatorTok{|}\NormalTok{Colony), }\DataTypeTok{data=}\KeywordTok{filter}\NormalTok{(allrecov,allrecov}\OperatorTok{$}\NormalTok{Species}\OperatorTok{==}\StringTok{'M.cavernosa'}\NormalTok{))}
  \KeywordTok{plot_resqq}\NormalTok{(mcavrecovmod)}
\end{Highlighting}
\end{Shaded}

\includegraphics{SEASONAL_SHUFFLING_files/figure-latex/unnamed-chunk-9-8.pdf}

\begin{Shaded}
\begin{Highlighting}[]
\NormalTok{  mcavrecovmod2<-}\KeywordTok{as_lmerModLmerTest}\NormalTok{(mcavrecovmod)}
  \KeywordTok{summary}\NormalTok{(mcavrecovmod2)}
\end{Highlighting}
\end{Shaded}

\begin{verbatim}
## Linear mixed model fit by REML. t-tests use Satterthwaite's method [
## lmerModLmerTest]
## Formula: Shprop ~ Recov.Days * Batch + (1 | Colony)
##    Data: filter(allrecov, allrecov$Species == "M.cavernosa")
## 
## REML criterion at convergence: 416.1
## 
## Scaled residuals: 
##     Min      1Q  Median      3Q     Max 
## -1.5973 -0.4961 -0.1152  0.1772  5.0981 
## 
## Random effects:
##  Groups   Name        Variance Std.Dev.
##  Colony   (Intercept) 0.1195   0.3457  
##  Residual             8.3119   2.8830  
## Number of obs: 82, groups:  Colony, 10
## 
## Fixed effects:
##                   Estimate Std. Error       df t value Pr(>|t|)    
## (Intercept)       -0.44603    0.75590 66.15752  -0.590  0.55716    
## Recov.Days         0.08342    0.01772 70.99172   4.706 1.21e-05 ***
## Batch2             0.23704    0.98147 74.25824   0.242  0.80982    
## Recov.Days:Batch2 -0.06258    0.02139 70.90131  -2.926  0.00461 ** 
## ---
## Signif. codes:  0 '***' 0.001 '**' 0.01 '*' 0.05 '.' 0.1 ' ' 1
## 
## Correlation of Fixed Effects:
##             (Intr) Rcv.Dy Batch2
## Recov.Days  -0.769              
## Batch2      -0.753  0.592       
## Rcv.Dys:Bt2  0.637 -0.829 -0.755
\end{verbatim}

\begin{Shaded}
\begin{Highlighting}[]
  \KeywordTok{anova}\NormalTok{(mcavrecovmod2) }\CommentTok{#there was a significant interaction p=0.004611 of batch on the relationship between symbiont density and recovery days, with the October batch recovering less.}
\end{Highlighting}
\end{Shaded}

\begin{verbatim}
## Type III Analysis of Variance Table with Satterthwaite's method
##                   Sum Sq Mean Sq NumDF  DenDF F value    Pr(>F)    
## Recov.Days       197.495 197.495     1 70.951 23.7606 6.437e-06 ***
## Batch              0.485   0.485     1 74.258  0.0583  0.809820    
## Recov.Days:Batch  71.158  71.158     1 70.901  8.5610  0.004611 ** 
## ---
## Signif. codes:  0 '***' 0.001 '**' 0.01 '*' 0.05 '.' 0.1 ' ' 1
\end{verbatim}

\begin{Shaded}
\begin{Highlighting}[]
\NormalTok{  Spec.labs=}\KeywordTok{c}\NormalTok{(}\StringTok{'M. cavernosa'}\NormalTok{,}\StringTok{'O. faveolata'}\NormalTok{,}\StringTok{'S. siderea'}\NormalTok{)}
\KeywordTok{names}\NormalTok{(Spec.labs)=}\KeywordTok{c}\NormalTok{(}\StringTok{'M.cavernosa'}\NormalTok{,}\StringTok{'O.faveolata'}\NormalTok{,}\StringTok{'S.siderea'}\NormalTok{)}
\end{Highlighting}
\end{Shaded}

\hypertarget{shuffling}{%
\subsection{SHUFFLING}\label{shuffling}}

\hypertarget{figure-2-inter--and-intra-species-variation-in-the-shuffling-towards-durusdinium-after-recovery-from-heat-stress.}{%
\subsubsection{Figure 2: Inter- and intra-species variation in the
shuffling towards durusdinium after recovery from heat
stress.}\label{figure-2-inter--and-intra-species-variation-in-the-shuffling-towards-durusdinium-after-recovery-from-heat-stress.}}

Comparison of Proportion D before start of heat stress compared to two
months into recovery. Data are binned at intervals of 0.02 (for variable
Proportion D), and size aesthetic relates to number of cores in each bin
to reduce overplotting. Data from April and October are both included
here.

\begin{Shaded}
\begin{Highlighting}[]
\NormalTok{mcavshift<-}\KeywordTok{filter}\NormalTok{(mcav_wide_batches,}\OperatorTok{!}\KeywordTok{is.na}\NormalTok{(postDcat),}\OperatorTok{!}\KeywordTok{is.na}\NormalTok{(preDcat))}
\NormalTok{ofavshift<-}\KeywordTok{filter}\NormalTok{(ofav_wide_batches,}\OperatorTok{!}\KeywordTok{is.na}\NormalTok{(postDcat),}\OperatorTok{!}\KeywordTok{is.na}\NormalTok{(preDcat))}
\NormalTok{ssidshift<-}\KeywordTok{filter}\NormalTok{(ssid_wide_batches,}\OperatorTok{!}\KeywordTok{is.na}\NormalTok{(postDcat),}\OperatorTok{!}\KeywordTok{is.na}\NormalTok{(preDcat))}
\NormalTok{beforeaftershift<-}\KeywordTok{rbind}\NormalTok{(mcavshift,ofavshift,ssidshift)}
\NormalTok{beforeaftershift}\OperatorTok{$}\NormalTok{Treatment<-}\KeywordTok{factor}\NormalTok{(beforeaftershift}\OperatorTok{$}\NormalTok{Treatment,}\DataTypeTok{levels=}\KeywordTok{c}\NormalTok{(}\StringTok{'Manipulated'}\NormalTok{,}\StringTok{'Control'}\NormalTok{))}
\NormalTok{gaindd=}\KeywordTok{expression}\NormalTok{(}\KeywordTok{paste}\NormalTok{(}\StringTok{"Gained"}\NormalTok{,}\KeywordTok{italic}\NormalTok{(}\StringTok{" Durusdinium"}\NormalTok{)))}
\NormalTok{lostd=}\KeywordTok{expression}\NormalTok{(}\KeywordTok{paste}\NormalTok{(}\StringTok{"Lost"}\NormalTok{,}\KeywordTok{italic}\NormalTok{(}\StringTok{" Durusdinium"}\NormalTok{)))}

   \KeywordTok{ggplot}\NormalTok{()}\OperatorTok{+}\KeywordTok{geom_count}\NormalTok{(}\DataTypeTok{data=}\NormalTok{(beforeaftershift),}
     \KeywordTok{aes}\NormalTok{(}\DataTypeTok{x=}\NormalTok{preDcat,}\DataTypeTok{y=}\NormalTok{postDcat,}\DataTypeTok{colour=}\NormalTok{Treatment))}\OperatorTok{+}
\KeywordTok{facet_grid}\NormalTok{(}\DataTypeTok{cols=}\KeywordTok{vars}\NormalTok{(Species),}\DataTypeTok{labeller=}\KeywordTok{labeller}\NormalTok{(}\DataTypeTok{Species=}\NormalTok{Spec.labs))}\OperatorTok{+}
\KeywordTok{geom_abline}\NormalTok{(}\DataTypeTok{slope=}\DecValTok{1}\NormalTok{,}\DataTypeTok{intercept =} \DecValTok{0}\NormalTok{,}\DataTypeTok{linetype=}\StringTok{'dotted'}\NormalTok{)}\OperatorTok{+}
\StringTok{  }\KeywordTok{scale_color_manual}\NormalTok{(}\DataTypeTok{values=}\KeywordTok{c}\NormalTok{(}\StringTok{'brown2'}\NormalTok{,}\StringTok{'blue3'}\NormalTok{),}\DataTypeTok{labels=}\KeywordTok{c}\NormalTok{(}\StringTok{'Bleached'}\NormalTok{,}\StringTok{'Control'}\NormalTok{))}\OperatorTok{+}
\StringTok{  }\KeywordTok{theme_minimal}\NormalTok{(}\DataTypeTok{base_size =} \DecValTok{15}\NormalTok{)}\OperatorTok
\StringTok{    }\KeywordTok{theme}\NormalTok{(}\DataTypeTok{panel.border =} \KeywordTok{element_rect}\NormalTok{(}\DataTypeTok{colour=}\StringTok{'grey20'}\NormalTok{,}\DataTypeTok{fill=}\OtherTok{NA}\NormalTok{),}
          \DataTypeTok{panel.grid.minor.x =} \KeywordTok{element_blank}\NormalTok{(),}
          \DataTypeTok{panel.grid.minor.y =} \KeywordTok{element_blank}\NormalTok{())}\OperatorTok{+}
\StringTok{  }\KeywordTok{labs}\NormalTok{(}\DataTypeTok{x=}\StringTok{'Proportion *Durusdinium* before'}\NormalTok{,}\DataTypeTok{y=}\StringTok{'Proportion *Durusdinium* after'}\NormalTok{)}\OperatorTok{+}
\StringTok{ }\KeywordTok{guides}\NormalTok{(}\DataTypeTok{size=}\KeywordTok{guide_legend}\NormalTok{(}\DataTypeTok{title=}\StringTok{'Binned n'}\NormalTok{))}\OperatorTok{+}
\StringTok{  }\KeywordTok{scale_size}\NormalTok{(}\DataTypeTok{range=}\KeywordTok{c}\NormalTok{(}\DecValTok{1}\NormalTok{,}\DecValTok{14}\NormalTok{),}\DataTypeTok{breaks=}\NormalTok{(}\KeywordTok{c}\NormalTok{(}\DecValTok{1}\NormalTok{,}\DecValTok{2}\NormalTok{,}\DecValTok{6}\NormalTok{,}\DecValTok{10}\NormalTok{)))}\OperatorTok{+}
\StringTok{  }\KeywordTok{theme}\NormalTok{(}\DataTypeTok{axis.title.x =} \KeywordTok{element_markdown}\NormalTok{(),}\DataTypeTok{axis.title.y=}\KeywordTok{element_markdown}\NormalTok{())}\OperatorTok{+}
\StringTok{  }\KeywordTok{theme}\NormalTok{(}\DataTypeTok{strip.text.x =} \KeywordTok{element_text}\NormalTok{(}\DataTypeTok{face =} \StringTok{'italic'}\NormalTok{),}
      \DataTypeTok{panel.spacing =} \KeywordTok{unit}\NormalTok{(}\FloatTok{1.5}\NormalTok{, }\StringTok{"lines"}\NormalTok{), }\DataTypeTok{legend.position =} \StringTok{'bottom'}\NormalTok{, }\DataTypeTok{legend.title=}\KeywordTok{element_text}\NormalTok{(}\DataTypeTok{size=}\DecValTok{12}\NormalTok{))}\OperatorTok{+}
\StringTok{      }\KeywordTok{guides}\NormalTok{(}\DataTypeTok{colour =} \KeywordTok{guide_legend}\NormalTok{(}\DataTypeTok{title=}\StringTok{''}\NormalTok{))}
\end{Highlighting}
\end{Shaded}

\includegraphics{SEASONAL_SHUFFLING_files/figure-latex/unnamed-chunk-10-1.pdf}

\begin{Shaded}
\begin{Highlighting}[]
\CommentTok{#ggsave('redblue_shift.pdf',device='pdf', width=10,height=5) #add 'gained durusdinium'/'lost durusdinium' labels}
\end{Highlighting}
\end{Shaded}

\hypertarget{figure-3a-shuffling-towards-high-proportions-of-durusdinium-after-heat-stress-recovery-was-greatest-in-m.-cavernosa-followed-by-o.-faveolata-followed-by-s.-siderea.}{%
\subsubsection{Figure 3a: Shuffling towards high proportions of
Durusdinium after heat stress recovery was greatest in M. cavernosa,
followed by O. faveolata, followed by S.
siderea.}\label{figure-3a-shuffling-towards-high-proportions-of-durusdinium-after-heat-stress-recovery-was-greatest-in-m.-cavernosa-followed-by-o.-faveolata-followed-by-s.-siderea.}}

Emulating the `symbiont shuffling' plot to compare coral species, as in
Cunning et al 2018 figure 3b. In order to be able to include mcav, which
has no variation in initial proportion d, the following mcav models are
independent of initial proportion d, then integrated in with the
previous ofav \& ssid predicted effects.

\begin{Shaded}
\begin{Highlighting}[]
\NormalTok{shufflemod=}\KeywordTok{lmer}\NormalTok{(post_propD }\OperatorTok{~}\StringTok{ }\NormalTok{pre_propD }\OperatorTok{+}\StringTok{ }\NormalTok{Treatment}\OperatorTok{*}\NormalTok{Species}\OperatorTok{*}\NormalTok{Batch}\OperatorTok{+}\NormalTok{(}\DecValTok{1}\OperatorTok{|}\NormalTok{Colony),}
                 \DataTypeTok{data=}\NormalTok{batches)}
\NormalTok{shufflemod2=}\KeywordTok{glm}\NormalTok{(post_propD}\OperatorTok{~}\NormalTok{pre_propD }\OperatorTok{+}\StringTok{ }\NormalTok{Treatment}\OperatorTok{*}\NormalTok{Species}\OperatorTok{*}\NormalTok{Batch, }\DataTypeTok{data=}\NormalTok{batches, }\DataTypeTok{family=}\StringTok{'quasibinomial'}\NormalTok{)}

\KeywordTok{plot_resqq}\NormalTok{(shufflemod) }
\end{Highlighting}
\end{Shaded}

\includegraphics{SEASONAL_SHUFFLING_files/figure-latex/unnamed-chunk-11-1.pdf}

\begin{Shaded}
\begin{Highlighting}[]
\KeywordTok{plot}\NormalTok{(shufflemod2)}
\end{Highlighting}
\end{Shaded}

\includegraphics{SEASONAL_SHUFFLING_files/figure-latex/unnamed-chunk-11-2.pdf}
\includegraphics{SEASONAL_SHUFFLING_files/figure-latex/unnamed-chunk-11-3.pdf}
\includegraphics{SEASONAL_SHUFFLING_files/figure-latex/unnamed-chunk-11-4.pdf}
\includegraphics{SEASONAL_SHUFFLING_files/figure-latex/unnamed-chunk-11-5.pdf}

\begin{Shaded}
\begin{Highlighting}[]
\CommentTok{#these two models (quasibinomial family vs linear mixed effects) give fairly differnet results...}

\KeywordTok{summary}\NormalTok{(shufflemod)}\CommentTok{# Model summary}
\end{Highlighting}
\end{Shaded}

\begin{verbatim}
## Linear mixed model fit by REML. t-tests use Satterthwaite's method [
## lmerModLmerTest]
## Formula: post_propD ~ pre_propD + Treatment * Species * Batch + (1 | Colony)
##    Data: batches
## 
## REML criterion at convergence: 38.2
## 
## Scaled residuals: 
##      Min       1Q   Median       3Q      Max 
## -2.70264 -0.41847 -0.03774  0.41929  2.53018 
## 
## Random effects:
##  Groups   Name        Variance Std.Dev.
##  Colony   (Intercept) 0.01996  0.1413  
##  Residual             0.05266  0.2295  
## Number of obs: 105, groups:  Colony, 29
## 
## Fixed effects:
##                                                  Estimate Std. Error       df
## (Intercept)                                       0.88340    0.07730 53.50301
## pre_propD                                         0.59718    0.09715 31.47720
## TreatmentControl                                 -0.89499    0.12754 78.92189
## SpeciesO.faveolata                               -0.25219    0.11305 50.69438
## SpeciesS.siderea                                 -0.42798    0.11937 47.68955
## BatchOctober                                      0.12362    0.08833 68.59613
## TreatmentControl:SpeciesO.faveolata               0.58869    0.18219 81.18425
## TreatmentControl:SpeciesS.siderea                 0.51816    0.18053 77.77864
## TreatmentControl:BatchOctober                    -0.11775    0.17995 78.02351
## SpeciesO.faveolata:BatchOctober                  -0.18054    0.13902 72.46015
## SpeciesS.siderea:BatchOctober                    -0.13107    0.13288 74.16874
## TreatmentControl:SpeciesO.faveolata:BatchOctober -0.10542    0.26833 83.10055
## TreatmentControl:SpeciesS.siderea:BatchOctober    0.36282    0.26333 82.61570
##                                                  t value Pr(>|t|)    
## (Intercept)                                       11.428 5.62e-16 ***
## pre_propD                                          6.147 7.59e-07 ***
## TreatmentControl                                  -7.017 6.95e-10 ***
## SpeciesO.faveolata                                -2.231  0.03015 *  
## SpeciesS.siderea                                  -3.585  0.00079 ***
## BatchOctober                                       1.400  0.16616    
## TreatmentControl:SpeciesO.faveolata                3.231  0.00178 ** 
## TreatmentControl:SpeciesS.siderea                  2.870  0.00528 ** 
## TreatmentControl:BatchOctober                     -0.654  0.51482    
## SpeciesO.faveolata:BatchOctober                   -1.299  0.19817    
## SpeciesS.siderea:BatchOctober                     -0.986  0.32718    
## TreatmentControl:SpeciesO.faveolata:BatchOctober  -0.393  0.69541    
## TreatmentControl:SpeciesS.siderea:BatchOctober     1.378  0.17198    
## ---
## Signif. codes:  0 '***' 0.001 '**' 0.01 '*' 0.05 '.' 0.1 ' ' 1
\end{verbatim}

\begin{Shaded}
\begin{Highlighting}[]
\CommentTok{#bleaching and initial proportion dusursdinium both had significant effects on shuffling (p=6.95e-10 and p=7.59e-07 respectively). Batch did not significantly affect shuffling (p=0.16616).}
                              

\CommentTok{#Get fitted values averaging across initial proportion durusdinium}
\NormalTok{eff <-}\StringTok{ }\KeywordTok{effect}\NormalTok{(}\KeywordTok{c}\NormalTok{(}\StringTok{'pre_propD'}\NormalTok{, }\StringTok{'Species'}\NormalTok{,}\StringTok{'Treatment'}\NormalTok{), shufflemod, }
              \DataTypeTok{xlevels=}\KeywordTok{list}\NormalTok{(}\DataTypeTok{pre_propD=}\KeywordTok{seq}\NormalTok{(}\DecValTok{0}\NormalTok{, }\DecValTok{1}\NormalTok{, }\DataTypeTok{by=}\FloatTok{0.01}\NormalTok{)))}
\CommentTok{# Get all fitted values and subsets for each treatment}
\NormalTok{res <-}\StringTok{ }\KeywordTok{droplevels}\NormalTok{(}\KeywordTok{data.frame}\NormalTok{(eff))}

\NormalTok{res}\OperatorTok{$}\NormalTok{Species <-}\StringTok{ }\KeywordTok{factor}\NormalTok{(res}\OperatorTok{$}\NormalTok{Species, }\DataTypeTok{levels=}\KeywordTok{c}\NormalTok{(}\StringTok{"O.faveolata"}\NormalTok{, }\StringTok{"S.siderea"}\NormalTok{, }\StringTok{"M.cavernosa"}\NormalTok{))}
\NormalTok{res.Bl <-}\StringTok{ }\KeywordTok{subset}\NormalTok{(}\KeywordTok{data.frame}\NormalTok{(eff), Treatment}\OperatorTok{==}\StringTok{"Manipulated"}\NormalTok{)}
\NormalTok{res.Ct <-}\StringTok{ }\KeywordTok{subset}\NormalTok{(}\KeywordTok{data.frame}\NormalTok{(eff), Treatment}\OperatorTok{==}\StringTok{"Control"}\NormalTok{)}
\CommentTok{# Get AUC for fitted values, lower and upper confidence limits}
\NormalTok{auc <-}\StringTok{ }\KeywordTok{aggregate}\NormalTok{(res[, }\KeywordTok{c}\NormalTok{(}\StringTok{"fit"}\NormalTok{, }\StringTok{"lower"}\NormalTok{, }\StringTok{"upper"}\NormalTok{)], }
                 \DataTypeTok{by=}\KeywordTok{list}\NormalTok{(}\DataTypeTok{Species=}\NormalTok{res}\OperatorTok{$}\NormalTok{Species, }\DataTypeTok{Treatment=}\NormalTok{res}\OperatorTok{$}\NormalTok{Treatment), }
                 \DataTypeTok{FUN=}\ControlFlowTok{function}\NormalTok{(x) (}\KeywordTok{mean}\NormalTok{(x)}\OperatorTok{-}\FloatTok{0.5}\NormalTok{)}\OperatorTok{/}\FloatTok{0.5}\NormalTok{)}
                
\CommentTok{#force limits of 1 and -1 on ofav and ssid}
\NormalTok{auc[}\DecValTok{4}\NormalTok{,}\DecValTok{5}\NormalTok{]=}\FloatTok{1.0}
\NormalTok{auc.list <-}\StringTok{ }\KeywordTok{split}\NormalTok{(auc, }\KeywordTok{list}\NormalTok{(auc}\OperatorTok{$}\NormalTok{Treatment))}
\NormalTok{auc}\OperatorTok{$}\NormalTok{Treatment=}\KeywordTok{factor}\NormalTok{(auc}\OperatorTok{$}\NormalTok{Treatment, }\DataTypeTok{levels=}\KeywordTok{c}\NormalTok{(}\StringTok{'Manipulated'}\NormalTok{,}\StringTok{'Control'}\NormalTok{))}
\KeywordTok{levels}\NormalTok{(auc}\OperatorTok{$}\NormalTok{Treatment)<-}\KeywordTok{c}\NormalTok{(}\StringTok{'Bleached'}\NormalTok{,}\StringTok{'Control'}\NormalTok{)}
\NormalTok{auc}\OperatorTok{$}\NormalTok{Species=}\KeywordTok{factor}\NormalTok{(auc}\OperatorTok{$}\NormalTok{Species, }\DataTypeTok{levels=}\KeywordTok{c}\NormalTok{(}\StringTok{'M.cavernosa'}\NormalTok{, }\StringTok{'O.faveolata'}\NormalTok{, }\StringTok{'S.siderea'}\NormalTok{))}

\CommentTok{#UPDATES FOR ROSS: I have now changed this so it is based on a mixed effects model (colony as a random factor as in all other analyses), and the result looks fairly differnet from before (which was based on a glm with quasibinomial distribution). Limited predicted values to between 1 and -1. What is the significance of the (-0.5/0.5) in the aggregation function?}
\end{Highlighting}
\end{Shaded}

\begin{Shaded}
\begin{Highlighting}[]
\CommentTok{#using model independent of initial prop d for mcav}
\NormalTok{shufflemod2=}\KeywordTok{lmer}\NormalTok{(post_propD }\OperatorTok{~}\NormalTok{Treatment}\OperatorTok{*}\NormalTok{Species}\OperatorTok{*}\NormalTok{Batch}\OperatorTok{+}\NormalTok{(}\DecValTok{1}\OperatorTok{|}\NormalTok{Colony),}
                 \DataTypeTok{data=}\NormalTok{batches)}
\KeywordTok{plot_resqq}\NormalTok{(shufflemod2) }
\end{Highlighting}
\end{Shaded}

\includegraphics{SEASONAL_SHUFFLING_files/figure-latex/unnamed-chunk-12-1.pdf}

\begin{Shaded}
\begin{Highlighting}[]
\KeywordTok{summary}\NormalTok{(shufflemod2)}
\end{Highlighting}
\end{Shaded}

\begin{verbatim}
## Linear mixed model fit by REML. t-tests use Satterthwaite's method [
## lmerModLmerTest]
## Formula: post_propD ~ Treatment * Species * Batch + (1 | Colony)
##    Data: batches
## 
## REML criterion at convergence: 68.5
## 
## Scaled residuals: 
##      Min       1Q   Median       3Q      Max 
## -2.56595 -0.29433  0.06883  0.35489  2.28865 
## 
## Random effects:
##  Groups   Name        Variance Std.Dev.
##  Colony   (Intercept) 0.05966  0.2443  
##  Residual             0.06123  0.2475  
## Number of obs: 108, groups:  Colony, 30
## 
## Fixed effects:
##                                                  Estimate Std. Error       df
## (Intercept)                                       0.88490    0.10345 44.86844
## TreatmentControl                                 -0.90026    0.14093 75.47542
## SpeciesO.faveolata                               -0.08228    0.14647 44.85383
## SpeciesS.siderea                                 -0.20658    0.14575 44.55474
## BatchOctober                                      0.12790    0.09591 69.67963
## TreatmentControl:SpeciesO.faveolata               0.73218    0.20115 77.38500
## TreatmentControl:SpeciesS.siderea                 0.57034    0.19799 75.06017
## TreatmentControl:BatchOctober                    -0.12784    0.19847 74.76893
## SpeciesO.faveolata:BatchOctober                  -0.14478    0.15171 72.75896
## SpeciesS.siderea:BatchOctober                     0.08178    0.13793 71.67140
## TreatmentControl:SpeciesO.faveolata:BatchOctober -0.31940    0.29652 78.95345
## TreatmentControl:SpeciesS.siderea:BatchOctober    0.22211    0.28892 77.88603
##                                                  t value Pr(>|t|)    
## (Intercept)                                        8.554 5.59e-11 ***
## TreatmentControl                                  -6.388 1.25e-08 ***
## SpeciesO.faveolata                                -0.562  0.57707    
## SpeciesS.siderea                                  -1.417  0.16334    
## BatchOctober                                       1.334  0.18670    
## TreatmentControl:SpeciesO.faveolata                3.640  0.00049 ***
## TreatmentControl:SpeciesS.siderea                  2.881  0.00517 ** 
## TreatmentControl:BatchOctober                     -0.644  0.52144    
## SpeciesO.faveolata:BatchOctober                   -0.954  0.34308    
## SpeciesS.siderea:BatchOctober                      0.593  0.55513    
## TreatmentControl:SpeciesO.faveolata:BatchOctober  -1.077  0.28469    
## TreatmentControl:SpeciesS.siderea:BatchOctober     0.769  0.44437    
## ---
## Signif. codes:  0 '***' 0.001 '**' 0.01 '*' 0.05 '.' 0.1 ' ' 1
## 
## Correlation of Fixed Effects:
##             (Intr) TrtmnC SpcsO. SpcsS. BtchOc TrC:SO. TrC:SS. TrC:BO SO.:BO
## TrtmntCntrl -0.371                                                          
## SpecsO.fvlt -0.706  0.262                                                   
## SpecisS.sdr -0.710  0.263  0.501                                            
## BatchOctobr -0.473  0.396  0.334  0.336                                     
## TrtmntC:SO.  0.260 -0.701 -0.374 -0.185 -0.278                              
## TrtmntC:SS.  0.264 -0.712 -0.187 -0.362 -0.282  0.499                       
## TrtmntCn:BO  0.261 -0.718 -0.184 -0.185 -0.538  0.503   0.511               
## SpcsO.fv:BO  0.299 -0.250 -0.436 -0.212 -0.632  0.381   0.178   0.340       
## SpcsS.sd:BO  0.329 -0.275 -0.232 -0.473 -0.695  0.193   0.408   0.374  0.440
## TrtC:SO.:BO -0.175  0.481  0.265  0.124  0.360 -0.719  -0.342  -0.669 -0.580
## TrtC:SS.:BO -0.179  0.493  0.127  0.264  0.370 -0.346  -0.729  -0.687 -0.234
##             SS.:BO TC:SO.:
## TrtmntCntrl               
## SpecsO.fvlt               
## SpecisS.sdr               
## BatchOctobr               
## TrtmntC:SO.               
## TrtmntC:SS.               
## TrtmntCn:BO               
## SpcsO.fv:BO               
## SpcsS.sd:BO               
## TrtC:SO.:BO -0.251        
## TrtC:SS.:BO -0.560  0.460
\end{verbatim}

\begin{Shaded}
\begin{Highlighting}[]
\CommentTok{#again for the initial durusdinium independent model, batch is not a significant driver of shuffling, p= 0.18670. }
\NormalTok{shufflemodmcav=}\KeywordTok{glm}\NormalTok{(post_propD }\OperatorTok{~}\NormalTok{Treatment}\OperatorTok{*}\NormalTok{Batch,}
                 \DataTypeTok{data=}\KeywordTok{subset}\NormalTok{(batches, batches}\OperatorTok{$}\NormalTok{Species}\OperatorTok{==}\StringTok{'M.cavernosa'}\NormalTok{), }\DataTypeTok{family=}\StringTok{'quasibinomial'}\NormalTok{)}
\KeywordTok{plot}\NormalTok{(shufflemodmcav)}
\end{Highlighting}
\end{Shaded}

\includegraphics{SEASONAL_SHUFFLING_files/figure-latex/unnamed-chunk-12-2.pdf}
\includegraphics{SEASONAL_SHUFFLING_files/figure-latex/unnamed-chunk-12-3.pdf}
\includegraphics{SEASONAL_SHUFFLING_files/figure-latex/unnamed-chunk-12-4.pdf}
\includegraphics{SEASONAL_SHUFFLING_files/figure-latex/unnamed-chunk-12-5.pdf}

\begin{Shaded}
\begin{Highlighting}[]
\KeywordTok{summary}\NormalTok{(shufflemodmcav)}
\end{Highlighting}
\end{Shaded}

\begin{verbatim}
## 
## Call:
## glm(formula = post_propD ~ Treatment * Batch, family = "quasibinomial", 
##     data = subset(batches, batches$Species == "M.cavernosa"))
## 
## Deviance Residuals: 
##      Min        1Q    Median        3Q       Max  
## -1.46467  -0.00008   0.05084   0.35840   0.50746  
## 
## Coefficients:
##                                Estimate Std. Error t value Pr(>|t|)    
## (Intercept)                      1.9848     0.4277   4.641 4.99e-05 ***
## TreatmentControl               -21.5508  2508.1870  -0.009    0.993    
## BatchOctober                     4.6658     3.9044   1.195    0.240    
## TreatmentControl:BatchOctober   10.9378  2508.1906   0.004    0.997    
## ---
## Signif. codes:  0 '***' 0.001 '**' 0.01 '*' 0.05 '.' 0.1 ' ' 1
## 
## (Dispersion parameter for quasibinomial family taken to be 0.2719874)
## 
##     Null deviance: 42.0350  on 37  degrees of freedom
## Residual deviance:  6.7725  on 34  degrees of freedom
##   (2 observations deleted due to missingness)
## AIC: NA
## 
## Number of Fisher Scoring iterations: 18
\end{verbatim}

\begin{Shaded}
\begin{Highlighting}[]
\NormalTok{eff2 <-}\StringTok{ }\KeywordTok{effect}\NormalTok{(}\KeywordTok{c}\NormalTok{(}\StringTok{'Species'}\NormalTok{,}\StringTok{'Treatment'}\NormalTok{,}\StringTok{'Batch'}\NormalTok{), shufflemod2)}

\CommentTok{# Get all fitted values and subsets for each treatment level}
\NormalTok{res2 <-}\StringTok{ }\KeywordTok{droplevels}\NormalTok{(}\KeywordTok{data.frame}\NormalTok{(eff2))}
\CommentTok{#force limits of 1 and 0 on mcav confidence intervals}
\NormalTok{res2[}\DecValTok{1}\NormalTok{,}\DecValTok{7}\NormalTok{]=}\DecValTok{1}
\NormalTok{res2[}\DecValTok{7}\NormalTok{,}\DecValTok{4}\NormalTok{]=}\DecValTok{1}
\NormalTok{res2[}\DecValTok{7}\NormalTok{,}\DecValTok{7}\NormalTok{]=}\DecValTok{1}
\NormalTok{res2[}\DecValTok{4}\NormalTok{,}\DecValTok{4}\NormalTok{]=}\DecValTok{0}
\NormalTok{res2[}\DecValTok{4}\NormalTok{,}\DecValTok{6}\NormalTok{]=}\DecValTok{0}
\NormalTok{res2[}\DecValTok{10}\NormalTok{,}\DecValTok{4}\NormalTok{]=}\DecValTok{0}
\NormalTok{res2[}\DecValTok{10}\NormalTok{,}\DecValTok{6}\NormalTok{]=}\DecValTok{0}

\NormalTok{res2}\OperatorTok{$}\NormalTok{Species <-}\StringTok{ }\KeywordTok{factor}\NormalTok{(res2}\OperatorTok{$}\NormalTok{Species, }\DataTypeTok{levels=}\KeywordTok{c}\NormalTok{(}\StringTok{'M.cavernosa'}\NormalTok{,}\StringTok{"O.faveolata"}\NormalTok{, }\StringTok{"S.siderea"}\NormalTok{))}
\NormalTok{res2.Bl <-}\StringTok{ }\KeywordTok{subset}\NormalTok{(}\KeywordTok{data.frame}\NormalTok{(eff2), Treatment}\OperatorTok{==}\StringTok{"Manipulated"}\NormalTok{)}
\NormalTok{res2.Ct <-}\StringTok{ }\KeywordTok{subset}\NormalTok{(}\KeywordTok{data.frame}\NormalTok{(eff2), Treatment}\OperatorTok{==}\StringTok{"Control"}\NormalTok{)}
\CommentTok{# Get AUC for fitted values, lower and upper confidence limits}
\NormalTok{auc2 <-}\StringTok{ }\KeywordTok{aggregate}\NormalTok{(res2[, }\KeywordTok{c}\NormalTok{(}\StringTok{"fit"}\NormalTok{, }\StringTok{"lower"}\NormalTok{, }\StringTok{"upper"}\NormalTok{)], }
                 \DataTypeTok{by=}\KeywordTok{list}\NormalTok{(}\DataTypeTok{Species=}\NormalTok{res2}\OperatorTok{$}\NormalTok{Species, }\DataTypeTok{Treatment=}\NormalTok{res2}\OperatorTok{$}\NormalTok{Treatment),}
                 \DataTypeTok{FUN=}\ControlFlowTok{function}\NormalTok{(x) (}\KeywordTok{mean}\NormalTok{(x)))}
\KeywordTok{levels}\NormalTok{(auc2}\OperatorTok{$}\NormalTok{Treatment)<-}\KeywordTok{c}\NormalTok{(}\StringTok{'Control'}\NormalTok{,}\StringTok{'Bleached'}\NormalTok{)}

\NormalTok{auc2.list <-}\StringTok{ }\KeywordTok{split}\NormalTok{(auc2, }\KeywordTok{list}\NormalTok{(auc2}\OperatorTok{$}\NormalTok{Treatment))}

\NormalTok{ofavssidshuff<-auc[}\KeywordTok{c}\NormalTok{(}\DecValTok{1}\NormalTok{,}\DecValTok{2}\NormalTok{,}\DecValTok{4}\NormalTok{,}\DecValTok{5}\NormalTok{),]}
\NormalTok{mcavshuff<-auc2[}\KeywordTok{c}\NormalTok{(}\DecValTok{1}\NormalTok{,}\DecValTok{4}\NormalTok{),]}
\NormalTok{speciesshuff<-}\KeywordTok{rbind}\NormalTok{(ofavssidshuff,mcavshuff)}
\NormalTok{speciesshuff}\OperatorTok{$}\NormalTok{Species=}\KeywordTok{factor}\NormalTok{(speciesshuff}\OperatorTok{$}\NormalTok{Species, }\DataTypeTok{levels=}\KeywordTok{c}\NormalTok{(}\StringTok{'M.cavernosa'}\NormalTok{, }\StringTok{'O.faveolata'}\NormalTok{, }\StringTok{'S.siderea'}\NormalTok{))}

\CommentTok{#quote the shuffling metric for bleached cores of each species, 'S': mcav=0.94245186, ofav=0.80645955, ssid=0.50105511. }

\KeywordTok{ggplot}\NormalTok{(}\DataTypeTok{data=}\NormalTok{speciesshuff)}\OperatorTok{+}
\StringTok{  }\KeywordTok{geom_hline}\NormalTok{(}\DataTypeTok{yintercept=}\DecValTok{0}\NormalTok{,}\DataTypeTok{linetype=}\StringTok{'dashed'}\NormalTok{)}\OperatorTok{+}
\StringTok{  }\KeywordTok{geom_errorbar}\NormalTok{(}\KeywordTok{aes}\NormalTok{(}\DataTypeTok{ymin=}\NormalTok{lower, }\DataTypeTok{ymax=}\NormalTok{upper, }\DataTypeTok{x=}\NormalTok{Species, }\DataTypeTok{colour=}\NormalTok{Species, }\DataTypeTok{group=}\NormalTok{Treatment),}\DataTypeTok{size=}\FloatTok{0.5}\NormalTok{, }\DataTypeTok{position=}\KeywordTok{position_dodge}\NormalTok{(}\DataTypeTok{width=}\FloatTok{0.5}\NormalTok{), }\DataTypeTok{width=}\FloatTok{0.2}\NormalTok{)}\OperatorTok{+}
\StringTok{   }\KeywordTok{geom_point}\NormalTok{(}\KeywordTok{aes}\NormalTok{(}\DataTypeTok{y=}\NormalTok{fit, }\DataTypeTok{x=}\NormalTok{Species,}\DataTypeTok{shape=}\NormalTok{Treatment, }\DataTypeTok{colour=}\NormalTok{Species),}\DataTypeTok{size=}\DecValTok{2}\NormalTok{, }\DataTypeTok{position=}\KeywordTok{position_dodge}\NormalTok{(}\DataTypeTok{width=}\FloatTok{0.5}\NormalTok{))}\OperatorTok{+}
\StringTok{  }\KeywordTok{theme_minimal}\NormalTok{(}\DataTypeTok{base_size =} \DecValTok{13}\NormalTok{)}\OperatorTok
\StringTok{   }\KeywordTok{theme}\NormalTok{(}\DataTypeTok{panel.border =} \KeywordTok{element_rect}\NormalTok{(}\DataTypeTok{colour=}\StringTok{'grey20'}\NormalTok{,}\DataTypeTok{fill=}\OtherTok{NA}\NormalTok{),}
          \DataTypeTok{panel.grid.minor.x =} \KeywordTok{element_blank}\NormalTok{(),}
         \DataTypeTok{panel.grid.major.x =} \KeywordTok{element_blank}\NormalTok{(),}
          \DataTypeTok{panel.grid.minor.y =} \KeywordTok{element_blank}\NormalTok{())}\OperatorTok{+}
\StringTok{  }\KeywordTok{scale_color_manual}\NormalTok{(}\DataTypeTok{values=}\KeywordTok{c}\NormalTok{(}\StringTok{'deeppink2'}\NormalTok{,}\StringTok{'darkorange1'}\NormalTok{, }\StringTok{'darkturquoise'}\NormalTok{))}\OperatorTok{+}
\StringTok{  }\KeywordTok{guides}\NormalTok{(}\DataTypeTok{colour=}\NormalTok{F, }\DataTypeTok{shape=}\KeywordTok{guide_legend}\NormalTok{(}\DataTypeTok{title=}\StringTok{''}\NormalTok{))}\OperatorTok{+}
\StringTok{  }\KeywordTok{theme}\NormalTok{(}\DataTypeTok{legend.position=}\KeywordTok{c}\NormalTok{(}\FloatTok{0.1}\NormalTok{,}\FloatTok{0.15}\NormalTok{))}\OperatorTok{+}
\StringTok{  }\KeywordTok{labs}\NormalTok{(}\DataTypeTok{y=}\StringTok{'Symbiont Shuffling'}\NormalTok{, }\DataTypeTok{x=}\StringTok{''}\NormalTok{)}\OperatorTok{+}
\StringTok{  }\KeywordTok{scale_y_continuous}\NormalTok{(}\DataTypeTok{limits=}\KeywordTok{c}\NormalTok{(}\OperatorTok{-}\DecValTok{1}\NormalTok{,}\DecValTok{1}\NormalTok{),}\DataTypeTok{expand=}\KeywordTok{c}\NormalTok{(}\DecValTok{0}\NormalTok{,}\DecValTok{0}\NormalTok{))}\OperatorTok{+}
\StringTok{   }\KeywordTok{scale_x_discrete}\NormalTok{(}\DataTypeTok{labels=}\KeywordTok{c}\NormalTok{(mcav,ofav,ssid), }\DataTypeTok{expand=}\KeywordTok{expansion}\NormalTok{(}\DataTypeTok{mult=}\KeywordTok{c}\NormalTok{(}\FloatTok{0.5}\NormalTok{,}\FloatTok{0.2}\NormalTok{)))}
\end{Highlighting}
\end{Shaded}

\includegraphics{SEASONAL_SHUFFLING_files/figure-latex/unnamed-chunk-12-6.pdf}

\begin{Shaded}
\begin{Highlighting}[]
\CommentTok{#ggsave('speciesshuffle.pdf',device='pdf', width=7,height=5)}
\CommentTok{#add 'more durusdinium/less durusdinium' labels post-save}

\CommentTok{#QUESTION FOR ROSS: unsure whether or not to use the '-0.5/0.5' in the aggregate function here (it seems to give a more conservative estimate for shuffling in bleached mcav, but does not work for the mcav controls)......as shown below}


\CommentTok{# Get AUC for fitted values, lower and upper confidence limits}
\NormalTok{auc2 <-}\StringTok{ }\KeywordTok{aggregate}\NormalTok{(res2[, }\KeywordTok{c}\NormalTok{(}\StringTok{"fit"}\NormalTok{, }\StringTok{"lower"}\NormalTok{, }\StringTok{"upper"}\NormalTok{)], }
                 \DataTypeTok{by=}\KeywordTok{list}\NormalTok{(}\DataTypeTok{Species=}\NormalTok{res2}\OperatorTok{$}\NormalTok{Species, }\DataTypeTok{Treatment=}\NormalTok{res2}\OperatorTok{$}\NormalTok{Treatment),}
                 \DataTypeTok{FUN=}\ControlFlowTok{function}\NormalTok{(x) (}\KeywordTok{mean}\NormalTok{(x)}\OperatorTok{-}\FloatTok{0.5}\NormalTok{)}\OperatorTok{/}\FloatTok{0.5}\NormalTok{)}
\KeywordTok{levels}\NormalTok{(auc2}\OperatorTok{$}\NormalTok{Treatment)<-}\KeywordTok{c}\NormalTok{(}\StringTok{'Control'}\NormalTok{,}\StringTok{'Bleached'}\NormalTok{)}
\NormalTok{auc2}
\end{Highlighting}
\end{Shaded}

\begin{verbatim}
##       Species Treatment           fit      lower      upper
## 1 M.cavernosa   Control -1.0000000000 -1.0000000 -0.4732779
## 2 O.faveolata   Control -0.1950366838 -0.7576407  0.3675674
## 3   S.siderea   Control  0.0007565134 -0.5582910  0.5598040
## 4 M.cavernosa  Bleached  0.8849037164  0.4888545  1.0000000
## 5 O.faveolata  Bleached  0.5883544891  0.1428467  1.0338623
## 6   S.siderea  Bleached  0.5663219334  0.1568355  0.9758084
\end{verbatim}

\hypertarget{shuffling-and-photochemical-efficiency}{%
\subsection{shuffling and photochemical
efficiency}\label{shuffling-and-photochemical-efficiency}}

\hypertarget{figure-s1-differnet-photochemical-disadvantages-of-hosting-durusdinium-at-ambient-temperatures-cannot-explain-the-difference-in-shuffling-between-o.-faveolata-and-s.-siderea.}{%
\subsubsection{Figure S1: Differnet photochemical disadvantages of
hosting Durusdinium at ambient temperatures cannot explain the
difference in shuffling between O. faveolata and S.
siderea.}\label{figure-s1-differnet-photochemical-disadvantages-of-hosting-durusdinium-at-ambient-temperatures-cannot-explain-the-difference-in-shuffling-between-o.-faveolata-and-s.-siderea.}}

Recreating Cunning et al 2018 plot of the relative photochemical
disadvantages of hosting durusdinium at ambient temperatures to explain
species hierarchy (ofav\textgreater{}ssid) in shuffling.

\begin{Shaded}
\begin{Highlighting}[]
\NormalTok{ipam=}\KeywordTok{filter}\NormalTok{(allbleach,allbleach}\OperatorTok{$}\NormalTok{Timepoint}\OperatorTok{==}\StringTok{'Pre-bleach'}\NormalTok{,allbleach}\OperatorTok{$}\NormalTok{Species}\OperatorTok{!=}\StringTok{'M.cavernosa'}\NormalTok{)}

\KeywordTok{ggplot}\NormalTok{()}\OperatorTok{+}
\StringTok{  }\KeywordTok{geom_smooth}\NormalTok{(}\DataTypeTok{method=}\StringTok{'glm'}\NormalTok{,}\DataTypeTok{data=}\NormalTok{ipam,}\KeywordTok{aes}\NormalTok{(}\DataTypeTok{y=}\NormalTok{Y2,}\DataTypeTok{x=}\NormalTok{PropD,}\DataTypeTok{colour=}\NormalTok{Species, }\DataTypeTok{linetype=}\NormalTok{Batch),}\DataTypeTok{method.args =} \KeywordTok{list}\NormalTok{(}\DataTypeTok{family =} \StringTok{"quasibinomial"}\NormalTok{))}\OperatorTok{+}
\StringTok{  }\KeywordTok{geom_point}\NormalTok{(}\DataTypeTok{data=}\NormalTok{ipam,}\KeywordTok{aes}\NormalTok{(}\DataTypeTok{y=}\NormalTok{Y2,}\DataTypeTok{x=}\NormalTok{PropD,}\DataTypeTok{colour=}\NormalTok{Species, }\DataTypeTok{shape=}\NormalTok{Batch))}\OperatorTok{+}
\StringTok{  }\KeywordTok{scale_colour_manual}\NormalTok{(}\DataTypeTok{values=}\KeywordTok{c}\NormalTok{(}\StringTok{'darkorange1'}\NormalTok{,}\StringTok{'darkturquoise'}\NormalTok{),}\DataTypeTok{labels=}\KeywordTok{c}\NormalTok{(ofav,ssid))}\OperatorTok{+}
\StringTok{  }\KeywordTok{theme_minimal}\NormalTok{(}\DataTypeTok{base_size =} \DecValTok{15}\NormalTok{)}\OperatorTok{+}
\StringTok{  }\KeywordTok{labs}\NormalTok{(}\DataTypeTok{y=}\StringTok{'Initial (ambient) Fv/Fm'}\NormalTok{,}\DataTypeTok{x=}\StringTok{'Proportion *Durusdinium*'}\NormalTok{)}\OperatorTok{+}
\StringTok{  }\KeywordTok{theme}\NormalTok{(}\DataTypeTok{axis.title.x=} \KeywordTok{element_markdown}\NormalTok{(),}\DataTypeTok{legend.title=}\KeywordTok{element_blank}\NormalTok{(),}\DataTypeTok{legend.position=}\StringTok{'bottom'}\NormalTok{)}
\end{Highlighting}
\end{Shaded}

\begin{verbatim}
## `geom_smooth()` using formula 'y ~ x'
\end{verbatim}

\includegraphics{SEASONAL_SHUFFLING_files/figure-latex/unnamed-chunk-13-1.pdf}

\begin{Shaded}
\begin{Highlighting}[]
\CommentTok{#ggsave('inity2.pdf',device='pdf', width=7,height=5)}

\NormalTok{ipaminitmod=}\KeywordTok{lmer}\NormalTok{(Y2}\OperatorTok{~}\NormalTok{PropD}\OperatorTok{*}\NormalTok{Species}\OperatorTok{*}\NormalTok{Batch}\OperatorTok{+}\NormalTok{(}\DecValTok{1}\OperatorTok{|}\NormalTok{Colony),}\DataTypeTok{data=}\NormalTok{ipam)}
\KeywordTok{plot_resqq}\NormalTok{(ipaminitmod)}
\end{Highlighting}
\end{Shaded}

\includegraphics{SEASONAL_SHUFFLING_files/figure-latex/unnamed-chunk-13-2.pdf}

\begin{Shaded}
\begin{Highlighting}[]
\KeywordTok{summary}\NormalTok{(ipaminitmod)}
\end{Highlighting}
\end{Shaded}

\begin{verbatim}
## Linear mixed model fit by REML. t-tests use Satterthwaite's method [
## lmerModLmerTest]
## Formula: Y2 ~ PropD * Species * Batch + (1 | Colony)
##    Data: ipam
## 
## REML criterion at convergence: -203.3
## 
## Scaled residuals: 
##      Min       1Q   Median       3Q      Max 
## -1.86465 -0.49506 -0.03011  0.50441  1.80649 
## 
## Random effects:
##  Groups   Name        Variance  Std.Dev.
##  Colony   (Intercept) 0.0001145 0.01070 
##  Residual             0.0002989 0.01729 
## Number of obs: 51, groups:  Colony, 19
## 
## Fixed effects:
##                                 Estimate Std. Error         df t value Pr(>|t|)
## (Intercept)                    0.5293324  0.0070538 25.0437420  75.042  < 2e-16
## PropD                         -0.0032823  0.0134075 27.9645585  -0.245 0.808391
## SpeciesS.siderea               0.0001896  0.0108844 26.2466965   0.017 0.986235
## Batch2                         0.0363517  0.0096538 37.8622808   3.766 0.000564
## PropD:SpeciesS.siderea        -0.0216401  0.0186729 31.1020055  -1.159 0.255317
## PropD:Batch2                  -0.0017877  0.0163661 33.5505728  -0.109 0.913669
## SpeciesS.siderea:Batch2       -0.0099099  0.0158472 33.9985376  -0.625 0.535923
## PropD:SpeciesS.siderea:Batch2 -0.0053723  0.0239426 32.1793643  -0.224 0.823879
##                                  
## (Intercept)                   ***
## PropD                            
## SpeciesS.siderea                 
## Batch2                        ***
## PropD:SpeciesS.siderea           
## PropD:Batch2                     
## SpeciesS.siderea:Batch2          
## PropD:SpeciesS.siderea:Batch2    
## ---
## Signif. codes:  0 '***' 0.001 '**' 0.01 '*' 0.05 '.' 0.1 ' ' 1
## 
## Correlation of Fixed Effects:
##             (Intr) PropD  SpcsS. Batch2 PrD:SS. PrD:B2 SS.:B2
## PropD       -0.554                                           
## SpecisS.sdr -0.648  0.359                                    
## Batch2      -0.517  0.289  0.335                             
## PrpD:SpcsS.  0.398 -0.718 -0.621 -0.207                      
## PropD:Btch2  0.317 -0.564 -0.205 -0.634  0.405               
## SpcsS.sd:B2  0.315 -0.176 -0.420 -0.609  0.246   0.386       
## PrpD:SS.:B2 -0.217  0.386  0.312  0.434 -0.526  -0.684 -0.731
\end{verbatim}

\begin{Shaded}
\begin{Highlighting}[]
\CommentTok{#UPDATE FOR ROSS: I'm not sure this plot is needed, maybe just the stats, given there isn't a significant finding here. No significant differnece in the interaction between proportion durusdinium and coral species on Fv/Fm (p=0.255). However, initial Fv/Fm values were generally significantly higher in October (compared to april), regardless of coral species and proportion durusdinium (p=0.0005), although I don't think this is particularly relevant to the story of this paper?}
\end{Highlighting}
\end{Shaded}

\hypertarget{figure-s2-despite-greater-symbiont-loss-in-the-october-m.-cavernosa-corals-shuffling-between-april-and-october-were-similar-for-all-three-coral-species.}{%
\subsubsection{(Figure S2:) Despite greater symbiont loss in the October
M. cavernosa corals, shuffling between April and October were similar
for all three coral
species.}\label{figure-s2-despite-greater-symbiont-loss-in-the-october-m.-cavernosa-corals-shuffling-between-april-and-october-were-similar-for-all-three-coral-species.}}

\begin{Shaded}
\begin{Highlighting}[]
\CommentTok{#before getting fitted values, test statistical significance of batch for each species inidivually:}
\NormalTok{ofavbatchmod=}\KeywordTok{lmer}\NormalTok{(post_propD }\OperatorTok{~}\StringTok{ }\NormalTok{pre_propD }\OperatorTok{+}\StringTok{ }\NormalTok{Batch}\OperatorTok{+}\NormalTok{(}\DecValTok{1}\OperatorTok{|}\NormalTok{Colony),}
                 \DataTypeTok{data=}\KeywordTok{filter}\NormalTok{(batches, batches}\OperatorTok{$}\NormalTok{Species}\OperatorTok{==}\StringTok{'O.faveolata'}\NormalTok{, Treatment}\OperatorTok{==}\StringTok{'Manipulated'}\NormalTok{))}
\NormalTok{ssidbatchmod=}\KeywordTok{lmer}\NormalTok{(post_propD }\OperatorTok{~}\StringTok{ }\NormalTok{pre_propD }\OperatorTok{+}\StringTok{ }\NormalTok{Batch}\OperatorTok{+}\NormalTok{(}\DecValTok{1}\OperatorTok{|}\NormalTok{Colony),}
                 \DataTypeTok{data=}\KeywordTok{filter}\NormalTok{(batches, batches}\OperatorTok{$}\NormalTok{Species}\OperatorTok{==}\StringTok{'S.siderea'}\NormalTok{, Treatment}\OperatorTok{==}\StringTok{'Manipulated'}\NormalTok{))  }
\NormalTok{mcavbatchmod=}\KeywordTok{lmer}\NormalTok{(post_propD }\OperatorTok{~}\NormalTok{Batch}\OperatorTok{+}\NormalTok{(}\DecValTok{1}\OperatorTok{|}\NormalTok{Colony),}
                 \DataTypeTok{data=}\KeywordTok{filter}\NormalTok{(batches, batches}\OperatorTok{$}\NormalTok{Species}\OperatorTok{==}\StringTok{'M.cavernosa'}\NormalTok{, Treatment}\OperatorTok{==}\StringTok{'Manipulated'}\NormalTok{))}
\KeywordTok{summary}\NormalTok{(ofavbatchmod) }\CommentTok{#p=0.546701}
\end{Highlighting}
\end{Shaded}

\begin{verbatim}
## Linear mixed model fit by REML. t-tests use Satterthwaite's method [
## lmerModLmerTest]
## Formula: post_propD ~ pre_propD + Batch + (1 | Colony)
##    Data: filter(batches, batches$Species == "O.faveolata", Treatment ==  
##     "Manipulated")
## 
## REML criterion at convergence: 11.9
## 
## Scaled residuals: 
##      Min       1Q   Median       3Q      Max 
## -1.84265 -0.10735  0.09364  0.20476  1.67041 
## 
## Random effects:
##  Groups   Name        Variance Std.Dev.
##  Colony   (Intercept) 0.08313  0.2883  
##  Residual             0.03990  0.1997  
## Number of obs: 22, groups:  Colony, 10
## 
## Fixed effects:
##              Estimate Std. Error       df t value Pr(>|t|)    
## (Intercept)   0.70531    0.12833  7.45285   5.496 0.000736 ***
## pre_propD     0.35253    0.23111  6.33618   1.525 0.175414    
## BatchOctober -0.06274    0.10098 11.32362  -0.621 0.546701    
## ---
## Signif. codes:  0 '***' 0.001 '**' 0.01 '*' 0.05 '.' 0.1 ' ' 1
## 
## Correlation of Fixed Effects:
##             (Intr) pr_prD
## pre_propD   -0.539       
## BatchOctobr -0.202 -0.145
\end{verbatim}

\begin{Shaded}
\begin{Highlighting}[]
\KeywordTok{summary}\NormalTok{(ssidbatchmod) }\CommentTok{#p=0.85918}
\end{Highlighting}
\end{Shaded}

\begin{verbatim}
## Linear mixed model fit by REML. t-tests use Satterthwaite's method [
## lmerModLmerTest]
## Formula: post_propD ~ pre_propD + Batch + (1 | Colony)
##    Data: filter(batches, batches$Species == "S.siderea", Treatment ==  
##     "Manipulated")
## 
## REML criterion at convergence: 17.4
## 
## Scaled residuals: 
##      Min       1Q   Median       3Q      Max 
## -2.15335 -0.08276 -0.01649  0.13550  1.81479 
## 
## Random effects:
##  Groups   Name        Variance Std.Dev.
##  Colony   (Intercept) 0.03293  0.1815  
##  Residual             0.07492  0.2737  
## Number of obs: 25, groups:  Colony, 9
## 
## Fixed effects:
##              Estimate Std. Error       df t value Pr(>|t|)   
## (Intercept)   0.50616    0.13036 11.20440   3.883  0.00247 **
## pre_propD     0.48309    0.19318 11.98865   2.501  0.02790 * 
## BatchOctober  0.02277    0.12658 18.68964   0.180  0.85918   
## ---
## Signif. codes:  0 '***' 0.001 '**' 0.01 '*' 0.05 '.' 0.1 ' ' 1
## 
## Correlation of Fixed Effects:
##             (Intr) pr_prD
## pre_propD   -0.658       
## BatchOctobr -0.101 -0.420
\end{verbatim}

\begin{Shaded}
\begin{Highlighting}[]
\KeywordTok{summary}\NormalTok{(mcavbatchmod) }\CommentTok{#p=0.0505}
\end{Highlighting}
\end{Shaded}

\begin{verbatim}
## Linear mixed model fit by REML. t-tests use Satterthwaite's method [
## lmerModLmerTest]
## Formula: post_propD ~ Batch + (1 | Colony)
##    Data: filter(batches, batches$Species == "M.cavernosa", Treatment ==  
##     "Manipulated")
## 
## REML criterion at convergence: -10.5
## 
## Scaled residuals: 
##     Min      1Q  Median      3Q     Max 
## -2.7890 -0.2055  0.1068  0.4819  1.2017 
## 
## Random effects:
##  Groups   Name        Variance Std.Dev.
##  Colony   (Intercept) 0.01003  0.1002  
##  Residual             0.02478  0.1574  
## Number of obs: 28, groups:  Colony, 10
## 
## Fixed effects:
##              Estimate Std. Error       df t value Pr(>|t|)    
## (Intercept)   0.88288    0.05391 18.20147  16.377 2.43e-12 ***
## BatchOctober  0.12636    0.06082 20.57959   2.078   0.0505 .  
## ---
## Signif. codes:  0 '***' 0.001 '**' 0.01 '*' 0.05 '.' 0.1 ' ' 1
## 
## Correlation of Fixed Effects:
##             (Intr)
## BatchOctobr -0.576
\end{verbatim}

\begin{Shaded}
\begin{Highlighting}[]
\NormalTok{eff <-}\StringTok{ }\KeywordTok{effect}\NormalTok{(}\KeywordTok{c}\NormalTok{(}\StringTok{'pre_propD'}\NormalTok{, }\StringTok{'Species'}\NormalTok{,}\StringTok{'Treatment'}\NormalTok{, }\StringTok{'Batch'}\NormalTok{), shufflemod, }
              \DataTypeTok{xlevels=}\KeywordTok{list}\NormalTok{(}\DataTypeTok{pre_propD=}\KeywordTok{seq}\NormalTok{(}\DecValTok{0}\NormalTok{, }\DecValTok{1}\NormalTok{, }\DataTypeTok{by=}\FloatTok{0.01}\NormalTok{)))}
\CommentTok{# Get all fitted values and subsets for each treatment and now also batch}
\NormalTok{res <-}\StringTok{ }\KeywordTok{droplevels}\NormalTok{(}\KeywordTok{data.frame}\NormalTok{(eff))}

\NormalTok{res}\OperatorTok{$}\NormalTok{Species <-}\StringTok{ }\KeywordTok{factor}\NormalTok{(res}\OperatorTok{$}\NormalTok{Species, }\DataTypeTok{levels=}\KeywordTok{c}\NormalTok{(}\StringTok{"O.faveolata"}\NormalTok{, }\StringTok{"S.siderea"}\NormalTok{, }\StringTok{"M.cavernosa"}\NormalTok{))}
\NormalTok{res.Bl <-}\StringTok{ }\KeywordTok{subset}\NormalTok{(}\KeywordTok{data.frame}\NormalTok{(eff), Treatment}\OperatorTok{==}\StringTok{"Manipulated"}\NormalTok{)}
\NormalTok{res.Ct <-}\StringTok{ }\KeywordTok{subset}\NormalTok{(}\KeywordTok{data.frame}\NormalTok{(eff), Treatment}\OperatorTok{==}\StringTok{"Control"}\NormalTok{)}
\NormalTok{res.Ap<-}\StringTok{ }\KeywordTok{subset}\NormalTok{(}\KeywordTok{data.frame}\NormalTok{(eff), Batch}\OperatorTok{==}\StringTok{'April'}\NormalTok{)}
\NormalTok{res.Oc<-}\StringTok{ }\KeywordTok{subset}\NormalTok{(}\KeywordTok{data.frame}\NormalTok{(eff), Batch}\OperatorTok{==}\StringTok{'October'}\NormalTok{)}
\CommentTok{# Get AUC for fitted values, lower and upper confidence limits}
\NormalTok{auc <-}\StringTok{ }\KeywordTok{aggregate}\NormalTok{(res[, }\KeywordTok{c}\NormalTok{(}\StringTok{"fit"}\NormalTok{, }\StringTok{"lower"}\NormalTok{, }\StringTok{"upper"}\NormalTok{)], }
                 \DataTypeTok{by=}\KeywordTok{list}\NormalTok{(}\DataTypeTok{Species=}\NormalTok{res}\OperatorTok{$}\NormalTok{Species, }\DataTypeTok{Treatment=}\NormalTok{res}\OperatorTok{$}\NormalTok{Treatment, }\DataTypeTok{Batch=}\NormalTok{res}\OperatorTok{$}\NormalTok{Batch), }
                 \DataTypeTok{FUN=}\ControlFlowTok{function}\NormalTok{(x) (}\KeywordTok{mean}\NormalTok{(x)}\OperatorTok{-}\FloatTok{0.5}\NormalTok{)}\OperatorTok{/}\FloatTok{0.5}\NormalTok{)}

\NormalTok{auc[}\DecValTok{4}\NormalTok{,}\DecValTok{6}\NormalTok{]=}\FloatTok{1.0}
\NormalTok{auc[}\DecValTok{10}\NormalTok{,}\DecValTok{6}\NormalTok{]=}\FloatTok{1.0}
\NormalTok{auc}\OperatorTok{$}\NormalTok{Treatment=}\KeywordTok{factor}\NormalTok{(auc}\OperatorTok{$}\NormalTok{Treatment, }\DataTypeTok{levels=}\KeywordTok{c}\NormalTok{(}\StringTok{'Manipulated'}\NormalTok{,}\StringTok{'Control'}\NormalTok{))}
\KeywordTok{levels}\NormalTok{(auc}\OperatorTok{$}\NormalTok{Treatment)<-}\KeywordTok{c}\NormalTok{(}\StringTok{'Bleached'}\NormalTok{,}\StringTok{'Control'}\NormalTok{)}
\NormalTok{auc}\OperatorTok{$}\NormalTok{Species=}\KeywordTok{factor}\NormalTok{(auc}\OperatorTok{$}\NormalTok{Species, }\DataTypeTok{levels=}\KeywordTok{c}\NormalTok{(}\StringTok{'M.cavernosa'}\NormalTok{, }\StringTok{'O.faveolata'}\NormalTok{, }\StringTok{'S.siderea'}\NormalTok{))}
\NormalTok{auc}\OperatorTok{$}\NormalTok{Batch=}\KeywordTok{factor}\NormalTok{(auc}\OperatorTok{$}\NormalTok{Batch, }\DataTypeTok{levels=}\KeywordTok{c}\NormalTok{(}\StringTok{'April'}\NormalTok{,}\StringTok{'October'}\NormalTok{))}
\NormalTok{ofavssidbatchshuff<-}\StringTok{ }\NormalTok{auc[}\OperatorTok{-}\KeywordTok{c}\NormalTok{(}\DecValTok{3}\NormalTok{,}\DecValTok{6}\NormalTok{,}\DecValTok{9}\NormalTok{,}\DecValTok{12}\NormalTok{),]}


\NormalTok{eff2 <-}\StringTok{ }\KeywordTok{effect}\NormalTok{(}\KeywordTok{c}\NormalTok{(}\StringTok{'Species'}\NormalTok{,}\StringTok{'Treatment'}\NormalTok{,}\StringTok{'Batch'}\NormalTok{), shufflemod2)}
\end{Highlighting}
\end{Shaded}

\begin{verbatim}
## NOTE: SpeciesTreatmentBatch is not a high-order term in the model
\end{verbatim}

\begin{Shaded}
\begin{Highlighting}[]
\NormalTok{res2 <-}\StringTok{ }\KeywordTok{droplevels}\NormalTok{(}\KeywordTok{data.frame}\NormalTok{(eff2))}
\CommentTok{#force limits of 1 and 0 on mcav confidence intervals}
\NormalTok{res2[}\DecValTok{1}\NormalTok{,}\DecValTok{7}\NormalTok{]=}\DecValTok{1}
\NormalTok{res2[}\DecValTok{7}\NormalTok{,}\DecValTok{4}\NormalTok{]=}\DecValTok{1}
\NormalTok{res2[}\DecValTok{7}\NormalTok{,}\DecValTok{7}\NormalTok{]=}\DecValTok{1}
\NormalTok{res2[}\DecValTok{4}\NormalTok{,}\DecValTok{4}\NormalTok{]=}\DecValTok{0}
\NormalTok{res2[}\DecValTok{4}\NormalTok{,}\DecValTok{6}\NormalTok{]=}\DecValTok{0}
\NormalTok{res2[}\DecValTok{10}\NormalTok{,}\DecValTok{4}\NormalTok{]=}\DecValTok{0}
\NormalTok{res2[}\DecValTok{10}\NormalTok{,}\DecValTok{6}\NormalTok{]=}\DecValTok{0}

\NormalTok{res2}\OperatorTok{$}\NormalTok{Species <-}\StringTok{ }\KeywordTok{factor}\NormalTok{(res2}\OperatorTok{$}\NormalTok{Species, }\DataTypeTok{levels=}\KeywordTok{c}\NormalTok{(}\StringTok{'M.cavernosa'}\NormalTok{,}\StringTok{"O.faveolata"}\NormalTok{, }\StringTok{"S.siderea"}\NormalTok{))}
\NormalTok{res2.Bl <-}\StringTok{ }\KeywordTok{subset}\NormalTok{(}\KeywordTok{data.frame}\NormalTok{(eff2), Treatment}\OperatorTok{==}\StringTok{"Manipulated"}\NormalTok{)}
\NormalTok{res2.Ct <-}\StringTok{ }\KeywordTok{subset}\NormalTok{(}\KeywordTok{data.frame}\NormalTok{(eff2), Treatment}\OperatorTok{==}\StringTok{"Control"}\NormalTok{)}
\NormalTok{res2.Ap<-}\StringTok{ }\KeywordTok{subset}\NormalTok{(}\KeywordTok{data.frame}\NormalTok{(eff2), Batch}\OperatorTok{==}\StringTok{'April'}\NormalTok{)}
\NormalTok{res2.Oc<-}\StringTok{ }\KeywordTok{subset}\NormalTok{(}\KeywordTok{data.frame}\NormalTok{(eff2), Batch}\OperatorTok{==}\StringTok{'October'}\NormalTok{)}

\CommentTok{# Get AUC for fitted values, lower and upper confidence limits}
\NormalTok{auc2 <-}\StringTok{ }\KeywordTok{aggregate}\NormalTok{(res2[, }\KeywordTok{c}\NormalTok{(}\StringTok{"fit"}\NormalTok{, }\StringTok{"lower"}\NormalTok{, }\StringTok{"upper"}\NormalTok{)], }
                 \DataTypeTok{by=}\KeywordTok{list}\NormalTok{(}\DataTypeTok{Species=}\NormalTok{res2}\OperatorTok{$}\NormalTok{Species, }\DataTypeTok{Treatment=}\NormalTok{res2}\OperatorTok{$}\NormalTok{Treatment, }\DataTypeTok{Batch=}\NormalTok{res2}\OperatorTok{$}\NormalTok{Batch),}
                 \DataTypeTok{FUN=}\ControlFlowTok{function}\NormalTok{(x) (}\KeywordTok{mean}\NormalTok{(x)))}
\KeywordTok{levels}\NormalTok{(auc2}\OperatorTok{$}\NormalTok{Treatment)<-}\KeywordTok{c}\NormalTok{(}\StringTok{'Control'}\NormalTok{,}\StringTok{'Bleached'}\NormalTok{)}
\NormalTok{mcavbatchshuff<-}\StringTok{ }\NormalTok{auc2[}\KeywordTok{c}\NormalTok{(}\DecValTok{1}\NormalTok{,}\DecValTok{4}\NormalTok{,}\DecValTok{7}\NormalTok{,}\DecValTok{10}\NormalTok{),]}
\NormalTok{batchshuff<-}\StringTok{ }\KeywordTok{rbind}\NormalTok{(ofavssidbatchshuff, mcavbatchshuff)}
\NormalTok{batchshuff}\OperatorTok{$}\NormalTok{spbatch<-}\StringTok{ }\KeywordTok{interaction}\NormalTok{ (batchshuff}\OperatorTok{$}\NormalTok{Batch,batchshuff}\OperatorTok{$}\NormalTok{Species)}

\KeywordTok{ggplot}\NormalTok{(}\DataTypeTok{data=}\KeywordTok{filter}\NormalTok{(batchshuff,batchshuff}\OperatorTok{$}\NormalTok{Treatment}\OperatorTok{==}\StringTok{'Bleached'}\NormalTok{))}\OperatorTok{+}
\StringTok{  }\KeywordTok{geom_errorbar}\NormalTok{(}\KeywordTok{aes}\NormalTok{(}\DataTypeTok{ymin=}\NormalTok{lower, }\DataTypeTok{ymax=}\NormalTok{upper, }\DataTypeTok{x=}\NormalTok{Species, }\DataTypeTok{colour=}\NormalTok{Species, }\DataTypeTok{group=}\NormalTok{Batch),}\DataTypeTok{size=}\FloatTok{0.5}\NormalTok{, }\DataTypeTok{position=}\KeywordTok{position_dodge}\NormalTok{(}\DataTypeTok{width=}\FloatTok{0.5}\NormalTok{), }\DataTypeTok{width=}\FloatTok{0.2}\NormalTok{)}\OperatorTok{+}
\StringTok{   }\KeywordTok{geom_point}\NormalTok{(}\KeywordTok{aes}\NormalTok{(}\DataTypeTok{y=}\NormalTok{fit, }\DataTypeTok{x=}\NormalTok{Species, }\DataTypeTok{shape=}\NormalTok{Batch, }\DataTypeTok{colour=}\NormalTok{Species),}\DataTypeTok{size=}\DecValTok{2}\NormalTok{, }\DataTypeTok{position=}\KeywordTok{position_dodge}\NormalTok{(}\DataTypeTok{width=}\FloatTok{0.5}\NormalTok{))}\OperatorTok{+}
\StringTok{  }\KeywordTok{theme_minimal}\NormalTok{(}\DataTypeTok{base_size =} \DecValTok{13}\NormalTok{)}\OperatorTok
\StringTok{   }\KeywordTok{theme}\NormalTok{(}\DataTypeTok{panel.border =} \KeywordTok{element_rect}\NormalTok{(}\DataTypeTok{colour=}\StringTok{'grey20'}\NormalTok{,}\DataTypeTok{fill=}\OtherTok{NA}\NormalTok{),}
          \DataTypeTok{panel.grid.minor.x =} \KeywordTok{element_blank}\NormalTok{(),}
         \DataTypeTok{panel.grid.major.x =} \KeywordTok{element_blank}\NormalTok{(),}
          \DataTypeTok{panel.grid.minor.y =} \KeywordTok{element_blank}\NormalTok{())}\OperatorTok{+}
\StringTok{  }\KeywordTok{scale_color_manual}\NormalTok{(}\DataTypeTok{values=}\KeywordTok{c}\NormalTok{(}\StringTok{'deeppink2'}\NormalTok{,}\StringTok{'darkorange1'}\NormalTok{, }\StringTok{'darkturquoise'}\NormalTok{))}\OperatorTok{+}
\StringTok{  }\KeywordTok{guides}\NormalTok{(}\DataTypeTok{colour=}\NormalTok{F, }\DataTypeTok{shape=}\KeywordTok{guide_legend}\NormalTok{(}\DataTypeTok{title=}\StringTok{''}\NormalTok{))}\OperatorTok{+}
\StringTok{  }\KeywordTok{theme}\NormalTok{(}\DataTypeTok{legend.position=}\KeywordTok{c}\NormalTok{(}\FloatTok{0.1}\NormalTok{,}\FloatTok{0.15}\NormalTok{))}\OperatorTok{+}
\StringTok{  }\KeywordTok{labs}\NormalTok{(}\DataTypeTok{y=}\StringTok{'Symbiont Shuffling in Bleached Corals'}\NormalTok{, }\DataTypeTok{x=}\StringTok{''}\NormalTok{)}\OperatorTok{+}
\StringTok{  }\KeywordTok{scale_y_continuous}\NormalTok{(}\DataTypeTok{limits=}\KeywordTok{c}\NormalTok{(}\DecValTok{0}\NormalTok{,}\DecValTok{1}\NormalTok{),}\DataTypeTok{expand=}\KeywordTok{c}\NormalTok{(}\DecValTok{0}\NormalTok{,}\DecValTok{0}\NormalTok{))}
\end{Highlighting}
\end{Shaded}

\includegraphics{SEASONAL_SHUFFLING_files/figure-latex/unnamed-chunk-14-1.pdf}

\begin{Shaded}
\begin{Highlighting}[]
\CommentTok{#ggsave('shufflebatches.pdf',device='pdf',width=7,height=5)}
\end{Highlighting}
\end{Shaded}

\hypertarget{figure-3b-shuffling-occurred-later-in-recovery-in-m.-cavernosa-compared-to-o.-faveolata-and-s.-siderea.}{%
\subsubsection{Figure 3b: Shuffling occurred later in recovery in M.
cavernosa compared to O. faveolata and S.
siderea.}\label{figure-3b-shuffling-occurred-later-in-recovery-in-m.-cavernosa-compared-to-o.-faveolata-and-s.-siderea.}}

\begin{Shaded}
\begin{Highlighting}[]
\NormalTok{shuffletimes=}\KeywordTok{read.csv}\NormalTok{(}\StringTok{'shuffle_timepoints.csv'}\NormalTok{,}\DataTypeTok{header =}\NormalTok{T)}
\NormalTok{shuffletimes}\OperatorTok{$}\NormalTok{Treatment=}\KeywordTok{factor}\NormalTok{(shuffletimes}\OperatorTok{$}\NormalTok{Treatment)}
\NormalTok{shuff=}\KeywordTok{read.csv}\NormalTok{(}\StringTok{'shuff.csv'}\NormalTok{,}\DataTypeTok{header =}\NormalTok{ T)}
\NormalTok{shuff}\OperatorTok{$}\NormalTok{Colony=}\KeywordTok{factor}\NormalTok{(shuff}\OperatorTok{$}\NormalTok{Colony)}
\NormalTok{shuff}\OperatorTok{$}\NormalTok{Species=}\KeywordTok{factor}\NormalTok{(shuff}\OperatorTok{$}\NormalTok{Species)}
\NormalTok{shuff}\OperatorTok{$}\NormalTok{Timepoint=}\KeywordTok{as.integer}\NormalTok{(shuff}\OperatorTok{$}\NormalTok{Timepoint,}\DataTypeTok{length=}\DecValTok{3}\NormalTok{)}


\NormalTok{shuffmod=}\KeywordTok{lmer}\NormalTok{(PropD}\OperatorTok{~}\NormalTok{PropD0}\OperatorTok{+}\NormalTok{Species}\OperatorTok{*}\NormalTok{Timepoint}\OperatorTok{+}\NormalTok{(}\DecValTok{1}\OperatorTok{|}\NormalTok{Colony),}\DataTypeTok{data=}\NormalTok{shuff)}
\KeywordTok{plot_resqq}\NormalTok{(shuffmod)}
\end{Highlighting}
\end{Shaded}

\includegraphics{SEASONAL_SHUFFLING_files/figure-latex/unnamed-chunk-15-1.pdf}

\begin{Shaded}
\begin{Highlighting}[]
\KeywordTok{summary}\NormalTok{(shuffmod)}
\end{Highlighting}
\end{Shaded}

\begin{verbatim}
## Linear mixed model fit by REML. t-tests use Satterthwaite's method [
## lmerModLmerTest]
## Formula: PropD ~ PropD0 + Species * Timepoint + (1 | Colony)
##    Data: shuff
## 
## REML criterion at convergence: 104.6
## 
## Scaled residuals: 
##     Min      1Q  Median      3Q     Max 
## -2.5126 -0.4678  0.0141  0.4194  3.7174 
## 
## Random effects:
##  Groups   Name        Variance Std.Dev.
##  Colony   (Intercept) 0.03418  0.1849  
##  Residual             0.07168  0.2677  
## Number of obs: 208, groups:  Colony, 29
## 
## Fixed effects:
##                               Estimate Std. Error        df t value Pr(>|t|)
## (Intercept)                   -0.42076    0.09910  92.91601  -4.246 5.16e-05
## PropD0                         0.39576    0.09866  45.80985   4.011 0.000221
## SpeciesO.faveolata             0.77968    0.14591  96.40165   5.344 6.09e-07
## SpeciesS.siderea               0.94733    0.16291  97.27006   5.815 7.71e-08
## Timepoint                      0.44927    0.03626 174.74719  12.391  < 2e-16
## SpeciesO.faveolata:Timepoint  -0.32474    0.05375 174.54374  -6.042 8.96e-09
## SpeciesS.siderea:Timepoint    -0.42711    0.05572 175.65745  -7.665 1.17e-12
##                                 
## (Intercept)                  ***
## PropD0                       ***
## SpeciesO.faveolata           ***
## SpeciesS.siderea             ***
## Timepoint                    ***
## SpeciesO.faveolata:Timepoint ***
## SpeciesS.siderea:Timepoint   ***
## ---
## Signif. codes:  0 '***' 0.001 '**' 0.01 '*' 0.05 '.' 0.1 ' ' 1
## 
## Correlation of Fixed Effects:
##             (Intr) PropD0 SpcsO. SpcsS. Timpnt SpO.:T
## PropD0      -0.002                                   
## SpecsO.fvlt -0.679 -0.202                            
## SpecisS.sdr -0.608 -0.358  0.485                     
## Timepoint   -0.736  0.000  0.500  0.447              
## SpcsO.fvl:T  0.496 -0.013 -0.729 -0.297 -0.675       
## SpcsS.sdr:T  0.479  0.012 -0.327 -0.720 -0.651  0.439
\end{verbatim}

\begin{Shaded}
\begin{Highlighting}[]
\KeywordTok{anova}\NormalTok{(shuffmod,}\DataTypeTok{test=}\StringTok{'F'}\NormalTok{)}
\end{Highlighting}
\end{Shaded}

\begin{verbatim}
## Type III Analysis of Variance Table with Satterthwaite's method
##                   Sum Sq Mean Sq NumDF  DenDF F value    Pr(>F)    
## PropD0            1.1534  1.1534     1  45.81  16.091 0.0002209 ***
## Species           3.0202  1.5101     2 100.17  21.066 2.308e-08 ***
## Timepoint         5.4417  5.4417     1 175.23  75.914 2.180e-15 ***
## Species:Timepoint 4.8481  2.4241     2 175.19  33.817 3.808e-13 ***
## ---
## Signif. codes:  0 '***' 0.001 '**' 0.01 '*' 0.05 '.' 0.1 ' ' 1
\end{verbatim}

\begin{Shaded}
\begin{Highlighting}[]
\NormalTok{propD1=}\KeywordTok{lmer}\NormalTok{(PropD }\OperatorTok{~}\StringTok{ }\NormalTok{PropD0}\OperatorTok{+}\NormalTok{Timepoint}\OperatorTok{*}\NormalTok{Species}\OperatorTok{+}\NormalTok{(}\DecValTok{1}\OperatorTok{|}\NormalTok{Colony),}
             \DataTypeTok{data=}\NormalTok{shuff)}

\NormalTok{eff1 <-}\StringTok{ }\KeywordTok{effect}\NormalTok{(}\KeywordTok{c}\NormalTok{(}\StringTok{'PropD0'}\NormalTok{, }\StringTok{'Timepoint'}\NormalTok{, }\StringTok{'Species'}\NormalTok{), propD1, }
              \DataTypeTok{xlevels=}\KeywordTok{list}\NormalTok{(}\DataTypeTok{PropD0=}\KeywordTok{seq}\NormalTok{(}\DecValTok{0}\NormalTok{, }\DecValTok{1}\NormalTok{, }\DataTypeTok{by=}\FloatTok{0.01}\NormalTok{)))}

\NormalTok{res <-}\StringTok{ }\KeywordTok{droplevels}\NormalTok{(}\KeywordTok{data.frame}\NormalTok{(eff1))}
\NormalTok{res}\OperatorTok{$}\NormalTok{Species <-}\StringTok{ }\KeywordTok{factor}\NormalTok{(res}\OperatorTok{$}\NormalTok{Species, }\DataTypeTok{levels=}\KeywordTok{c}\NormalTok{(}\StringTok{"O.faveolata"}\NormalTok{, }\StringTok{"S.siderea"}\NormalTok{))}
\CommentTok{# Get AUC for fitted values, lower and upper confidence limits}
\NormalTok{auc1 <-}\StringTok{ }\KeywordTok{aggregate}\NormalTok{(res[, }\KeywordTok{c}\NormalTok{(}\StringTok{"fit"}\NormalTok{, }\StringTok{"lower"}\NormalTok{, }\StringTok{"upper"}\NormalTok{)], }
                 \DataTypeTok{by=}\KeywordTok{list}\NormalTok{(}\DataTypeTok{Species=}\NormalTok{res}\OperatorTok{$}\NormalTok{Species,}\DataTypeTok{Timepoint=}\NormalTok{res}\OperatorTok{$}\NormalTok{Timepoint), }
                 \DataTypeTok{FUN=}\ControlFlowTok{function}\NormalTok{(x) (}\KeywordTok{mean}\NormalTok{(x)}\OperatorTok{-}\FloatTok{0.5}\NormalTok{)}\OperatorTok{/}\FloatTok{0.5}\NormalTok{)}

\CommentTok{#now an intial prop d independent model for mcav}
\NormalTok{propD2=}\KeywordTok{lmer}\NormalTok{(PropD }\OperatorTok{~}\StringTok{ }\NormalTok{Timepoint}\OperatorTok{*}\NormalTok{Species}\OperatorTok{+}\NormalTok{(}\DecValTok{1}\OperatorTok{|}\NormalTok{Colony),}
             \DataTypeTok{data=}\NormalTok{shuff)}

\NormalTok{eff2 <-}\StringTok{ }\KeywordTok{effect}\NormalTok{(}\KeywordTok{c}\NormalTok{(}\StringTok{'Timepoint'}\NormalTok{, }\StringTok{'Species'}\NormalTok{), propD2)}
\end{Highlighting}
\end{Shaded}

\begin{verbatim}
## NOTE: TimepointSpecies is not a high-order term in the model
\end{verbatim}

\begin{Shaded}
\begin{Highlighting}[]
\NormalTok{res <-}\StringTok{ }\KeywordTok{droplevels}\NormalTok{(}\KeywordTok{data.frame}\NormalTok{(eff2))}
\NormalTok{res}\OperatorTok{$}\NormalTok{Species <-}\StringTok{ }\KeywordTok{factor}\NormalTok{(res}\OperatorTok{$}\NormalTok{Species, }\DataTypeTok{levels=}\KeywordTok{c}\NormalTok{(}\StringTok{"M.cavernosa"}\NormalTok{))}
\CommentTok{# Get AUC for fitted values, lower and upper confidence limits}
\NormalTok{auc2 <-}\StringTok{ }\KeywordTok{aggregate}\NormalTok{(res[, }\KeywordTok{c}\NormalTok{(}\StringTok{"fit"}\NormalTok{, }\StringTok{"lower"}\NormalTok{, }\StringTok{"upper"}\NormalTok{)], }
                 \DataTypeTok{by=}\KeywordTok{list}\NormalTok{(}\DataTypeTok{Species=}\NormalTok{res}\OperatorTok{$}\NormalTok{Species,}\DataTypeTok{Timepoint=}\NormalTok{res}\OperatorTok{$}\NormalTok{Timepoint), }
                 \DataTypeTok{FUN=}\ControlFlowTok{function}\NormalTok{(x) (}\KeywordTok{mean}\NormalTok{(x)))}


\NormalTok{aucall <-}\StringTok{ }\KeywordTok{rbind}\NormalTok{(auc1, auc2)}
\NormalTok{aucall}\OperatorTok{$}\NormalTok{Timepoint=}\KeywordTok{factor}\NormalTok{(aucall}\OperatorTok{$}\NormalTok{Timepoint, }\DataTypeTok{levels=}\KeywordTok{c}\NormalTok{(}\DecValTok{1}\NormalTok{,}\DecValTok{2}\NormalTok{,}\DecValTok{3}\NormalTok{))}
\NormalTok{aucall[}\DecValTok{11}\NormalTok{,}\DecValTok{4}\NormalTok{]=}\DecValTok{0}
\NormalTok{aucall[}\DecValTok{9}\NormalTok{,}\DecValTok{5}\NormalTok{]=}\DecValTok{1}
\NormalTok{aucall[}\DecValTok{15}\NormalTok{,}\DecValTok{5}\NormalTok{]=}\DecValTok{1}
\NormalTok{aucall=aucall[}\OperatorTok{-}\KeywordTok{c}\NormalTok{(}\DecValTok{3}\NormalTok{,}\DecValTok{4}\NormalTok{,}\DecValTok{7}\NormalTok{,}\DecValTok{8}\NormalTok{,}\DecValTok{12}\NormalTok{,}\DecValTok{14}\NormalTok{),]}
\NormalTok{aucall}\OperatorTok{$}\NormalTok{Species=}\KeywordTok{factor}\NormalTok{(aucall}\OperatorTok{$}\NormalTok{Species, }\DataTypeTok{levels=}\KeywordTok{c}\NormalTok{(}\StringTok{'M.cavernosa'}\NormalTok{,}\StringTok{'O.faveolata'}\NormalTok{,}\StringTok{'S.siderea'}\NormalTok{))}


\CommentTok{##########}
\NormalTok{ofav=}\KeywordTok{expression}\NormalTok{(}\KeywordTok{paste}\NormalTok{(}\KeywordTok{italic}\NormalTok{(}\StringTok{"O. faveolata"}\NormalTok{)))}
\NormalTok{ssid=}\KeywordTok{expression}\NormalTok{(}\KeywordTok{paste}\NormalTok{(}\KeywordTok{italic}\NormalTok{(}\StringTok{"S. siderea"}\NormalTok{)))}
\NormalTok{mcav=}\KeywordTok{expression}\NormalTok{(}\KeywordTok{paste}\NormalTok{(}\KeywordTok{italic}\NormalTok{(}\StringTok{"M. cavernosa"}\NormalTok{)))}

\KeywordTok{ggplot}\NormalTok{(aucall, }\KeywordTok{aes}\NormalTok{(}\DataTypeTok{x =}\NormalTok{ Timepoint, }\DataTypeTok{y =}\NormalTok{ fit, }\DataTypeTok{group =}\NormalTok{ Species)) }\OperatorTok{+}
\StringTok{  }\KeywordTok{geom_errorbar}\NormalTok{(}\DataTypeTok{data=}\NormalTok{aucall, }\KeywordTok{aes}\NormalTok{(}\DataTypeTok{ymin =}\NormalTok{ lower, }\DataTypeTok{ymax =}\NormalTok{ upper), }
                \DataTypeTok{position =} \KeywordTok{position_dodge}\NormalTok{(}\FloatTok{0.2}\NormalTok{), }\DataTypeTok{lwd =} \FloatTok{0.2}\NormalTok{, }\DataTypeTok{width =} \FloatTok{0.2}\NormalTok{) }\OperatorTok{+}
\StringTok{  }\KeywordTok{geom_point}\NormalTok{(}\KeywordTok{aes}\NormalTok{(}\DataTypeTok{color =}\NormalTok{ Species),}
             \DataTypeTok{position =} \KeywordTok{position_dodge}\NormalTok{(}\FloatTok{0.2}\NormalTok{), }\DataTypeTok{size =} \FloatTok{2.5}\NormalTok{)}\OperatorTok{+}
\StringTok{  }\KeywordTok{geom_hline}\NormalTok{(}\DataTypeTok{yintercept =} \DecValTok{0}\NormalTok{, }\DataTypeTok{lwd =} \FloatTok{0.1}\NormalTok{) }\OperatorTok{+}
\StringTok{  }\KeywordTok{scale_y_continuous}\NormalTok{(}\DataTypeTok{limits =} \KeywordTok{c}\NormalTok{(}\OperatorTok{-}\FloatTok{0.25}\NormalTok{, }\DecValTok{1}\NormalTok{))}\OperatorTok{+}
\StringTok{  }\KeywordTok{scale_x_discrete}\NormalTok{(}\DataTypeTok{labels=}\KeywordTok{c}\NormalTok{(}\StringTok{'Post heat stress'}\NormalTok{,}\StringTok{'1 month recovery'}\NormalTok{,}\StringTok{'2 month recovery'}\NormalTok{),}\DataTypeTok{expand=}\KeywordTok{expansion}\NormalTok{(}\DataTypeTok{mult=}\KeywordTok{c}\NormalTok{(}\FloatTok{0.2}\NormalTok{,}\FloatTok{0.4}\NormalTok{)))}\OperatorTok{+}
\StringTok{  }\KeywordTok{theme_minimal}\NormalTok{(}\DataTypeTok{base_size =} \DecValTok{13}\NormalTok{)}\OperatorTok
\StringTok{   }\KeywordTok{theme}\NormalTok{(}\DataTypeTok{panel.border =} \KeywordTok{element_rect}\NormalTok{(}\DataTypeTok{colour=}\StringTok{'grey20'}\NormalTok{,}\DataTypeTok{fill=}\OtherTok{NA}\NormalTok{),}
          \DataTypeTok{panel.grid.minor.x =} \KeywordTok{element_blank}\NormalTok{(),}
         \DataTypeTok{panel.grid.major.x =} \KeywordTok{element_blank}\NormalTok{(),}
          \DataTypeTok{panel.grid.minor.y =} \KeywordTok{element_blank}\NormalTok{())}\OperatorTok{+}
\StringTok{  }\KeywordTok{geom_line}\NormalTok{(}\DataTypeTok{linetype=}\StringTok{'dashed'}\NormalTok{,}\DataTypeTok{alpha=}\FloatTok{0.6}\NormalTok{,}\KeywordTok{aes}\NormalTok{(}\DataTypeTok{colour=}\NormalTok{Species),}\DataTypeTok{position =} \KeywordTok{position_dodge}\NormalTok{(}\FloatTok{0.2}\NormalTok{))}\OperatorTok{+}
\StringTok{  }\KeywordTok{scale_colour_manual}\NormalTok{(}\DataTypeTok{values=}\KeywordTok{c}\NormalTok{(}\StringTok{'deeppink2'}\NormalTok{,}\StringTok{'darkorange1'}\NormalTok{,}\StringTok{'darkturquoise'}\NormalTok{),}\DataTypeTok{labels=}\KeywordTok{c}\NormalTok{(mcav,ofav,ssid), }\DataTypeTok{name=}\StringTok{''}\NormalTok{)}\OperatorTok{+}
\StringTok{  }\KeywordTok{labs}\NormalTok{(}\DataTypeTok{y=}\StringTok{'Cumulative Symbiont Shuffling'}\NormalTok{,}\DataTypeTok{x=}\StringTok{''}\NormalTok{)}\OperatorTok{+}
\StringTok{  }\KeywordTok{annotate}\NormalTok{(}\DataTypeTok{geom=}\StringTok{'text'}\NormalTok{,}\DataTypeTok{x=}\FloatTok{3.4}\NormalTok{,}\DataTypeTok{y=}\FloatTok{0.1}\NormalTok{,}\DataTypeTok{label=}\NormalTok{gaindd,}\DataTypeTok{size=}\DecValTok{4}\NormalTok{)}\OperatorTok{+}
\StringTok{   }\KeywordTok{annotate}\NormalTok{(}\DataTypeTok{geom=}\StringTok{'text'}\NormalTok{,}\DataTypeTok{x=}\FloatTok{3.4}\NormalTok{,}\DataTypeTok{y=}\OperatorTok{-}\FloatTok{0.1}\NormalTok{,}\DataTypeTok{label=}\NormalTok{lostd,}\DataTypeTok{size=}\DecValTok{4}\NormalTok{)}\OperatorTok{+}
\StringTok{  }\KeywordTok{theme}\NormalTok{(}\DataTypeTok{legend.position =} \StringTok{'bottom'}\NormalTok{)}
\end{Highlighting}
\end{Shaded}

\includegraphics{SEASONAL_SHUFFLING_files/figure-latex/unnamed-chunk-15-2.pdf}

\begin{Shaded}
\begin{Highlighting}[]
\CommentTok{#ggsave('shuffletimingcumulative.pdf',device='pdf',height=5,width=7)}

\CommentTok{###UPDATE FOR ROSS: again, I have changed this to usea mixed effects model, and have forced limits of 1 and -1 for the errorbars (1 and 0 for mcav which had no initial durusdinium). The predicted values are based of a model which controls for pre-heat stress proportion durusdinium (rather than proportion durusdinium at the previous timepoint). Again, I'm unsure of the '-0.5/0.5' part of the aggregate function- this plot shows lower predicted shuffling in ofav and ssid compared to original plots based on a glm with quasibinomial distribution. Making 'timepoint' an integer rather than a factor in the model also changed things. Would liek to perform stats on the interaction effect on shuffling between timepoint and species, but unsire how since mcav fitted to a separate model. }
\end{Highlighting}
\end{Shaded}

\hypertarget{figure-3c-o.-faveolata-and-s.siderea-corals-that-initially-hosted-background-durusdinium-shuffled-more-quickly-and-more-fully-to-durusdinium-than-those-with-no-initial-durusdinium.}{%
\subsubsection{Figure 3c: O. faveolata and S.siderea corals that
initially hosted background Durusdinium shuffled more quickly and more
fully to Durusdinium than those with no initial
Durusdinium.}\label{figure-3c-o.-faveolata-and-s.siderea-corals-that-initially-hosted-background-durusdinium-shuffled-more-quickly-and-more-fully-to-durusdinium-than-those-with-no-initial-durusdinium.}}

\begin{Shaded}
\begin{Highlighting}[]
\CommentTok{#now look at timing for ofav and ssid switching and shuffling}
\NormalTok{shufftimesofss=}\KeywordTok{read.csv}\NormalTok{(}\StringTok{'shuffle_timepointsofavssid.csv'}\NormalTok{,}\DataTypeTok{head=}\NormalTok{T)}
\NormalTok{shufftimesofss=}\KeywordTok{filter}\NormalTok{(shufftimesofss,(shufftimesofss}\OperatorTok{$}\NormalTok{Treatment}\OperatorTok{==}\StringTok{'Manipulated'}\NormalTok{))}
\NormalTok{shufftimesofss=}\KeywordTok{filter}\NormalTok{(shufftimesofss,(shufftimesofss}\OperatorTok{$}\NormalTok{initial.d}\OperatorTok{!=}\StringTok{'NA'}\NormalTok{))}
\NormalTok{shuffmod2=}\KeywordTok{lmer}\NormalTok{(PropD2}\OperatorTok{~}\NormalTok{initial.d}\OperatorTok{*}\NormalTok{Species}\OperatorTok{+}\NormalTok{(}\DecValTok{1}\OperatorTok{|}\NormalTok{Colony),}\DataTypeTok{data=}\NormalTok{shufftimesofss)}
\KeywordTok{plot_resqq}\NormalTok{(shuffmod2)}
\end{Highlighting}
\end{Shaded}

\includegraphics{SEASONAL_SHUFFLING_files/figure-latex/unnamed-chunk-16-1.pdf}

\begin{Shaded}
\begin{Highlighting}[]
\KeywordTok{summary}\NormalTok{(shuffmod2) }
\end{Highlighting}
\end{Shaded}

\begin{verbatim}
## Linear mixed model fit by REML. t-tests use Satterthwaite's method [
## lmerModLmerTest]
## Formula: PropD2 ~ initial.d * Species + (1 | Colony)
##    Data: shufftimesofss
## 
## REML criterion at convergence: 14.1
## 
## Scaled residuals: 
##      Min       1Q   Median       3Q      Max 
## -2.06511 -0.05183 -0.03832  0.37651  1.61534 
## 
## Random effects:
##  Groups   Name        Variance Std.Dev.
##  Colony   (Intercept) 0.07282  0.2699  
##  Residual             0.06080  0.2466  
## Number of obs: 19, groups:  Colony, 11
## 
## Fixed effects:
##                             Estimate Std. Error       df t value Pr(>|t|)   
## (Intercept)                  0.04340    0.15703  7.73680   0.276  0.78952   
## initial.dy                   0.59597    0.19799 14.11762   3.010  0.00929 **
## SpeciesS.siderea            -0.04340    0.30244  9.04668  -0.143  0.88905   
## initial.dy:SpeciesS.siderea -0.09597    0.41571 10.65192  -0.231  0.82180   
## ---
## Signif. codes:  0 '***' 0.001 '**' 0.01 '*' 0.05 '.' 0.1 ' ' 1
## 
## Correlation of Fixed Effects:
##             (Intr) intl.d SpcsS.
## initial.dy  -0.633              
## SpecisS.sdr -0.519  0.329       
## intl.dy:SS.  0.302 -0.476 -0.688
\end{verbatim}

\begin{Shaded}
\begin{Highlighting}[]
\KeywordTok{anova}\NormalTok{(shuffmod2,}\DataTypeTok{test=}\StringTok{'F'}\NormalTok{)}\CommentTok{# prop d significantly higher in those that started with some d p=0.02371}
\end{Highlighting}
\end{Shaded}

\begin{verbatim}
## Type III Analysis of Variance Table with Satterthwaite's method
##                    Sum Sq Mean Sq NumDF   DenDF F value  Pr(>F)  
## initial.d         0.42255 0.42255     1 10.6519  6.9504 0.02371 *
## Species           0.01054 0.01054     1  8.0401  0.1733 0.68804  
## initial.d:Species 0.00324 0.00324     1 10.6519  0.0533 0.82180  
## ---
## Signif. codes:  0 '***' 0.001 '**' 0.01 '*' 0.05 '.' 0.1 ' ' 1
\end{verbatim}

\begin{Shaded}
\begin{Highlighting}[]
\NormalTok{shuffmod3=}\KeywordTok{lmer}\NormalTok{(PropD3}\OperatorTok{~}\NormalTok{initial.d}\OperatorTok{*}\NormalTok{Species}\OperatorTok{+}\NormalTok{(}\DecValTok{1}\OperatorTok{|}\NormalTok{Colony),}\DataTypeTok{data=}\NormalTok{shufftimesofss)}
\KeywordTok{plot_resqq}\NormalTok{(shuffmod3)}
\end{Highlighting}
\end{Shaded}

\includegraphics{SEASONAL_SHUFFLING_files/figure-latex/unnamed-chunk-16-2.pdf}

\begin{Shaded}
\begin{Highlighting}[]
\KeywordTok{summary}\NormalTok{(shuffmod3) }
\end{Highlighting}
\end{Shaded}

\begin{verbatim}
## Linear mixed model fit by REML. t-tests use Satterthwaite's method [
## lmerModLmerTest]
## Formula: PropD3 ~ initial.d * Species + (1 | Colony)
##    Data: shufftimesofss
## 
## REML criterion at convergence: 24.8
## 
## Scaled residuals: 
##     Min      1Q  Median      3Q     Max 
## -1.2491 -0.8478  0.0000  0.9146  1.2343 
## 
## Random effects:
##  Groups   Name        Variance Std.Dev.
##  Colony   (Intercept) 0.0000   0.0000  
##  Residual             0.1641   0.4051  
## Number of obs: 22, groups:  Colony, 12
## 
## Fixed effects:
##                              Estimate Std. Error        df t value Pr(>|t|)   
## (Intercept)                  0.506007   0.143226 18.000000   3.533  0.00238 **
## initial.dy                   0.493993   0.202552 18.000000   2.439  0.02532 * 
## SpeciesS.siderea            -0.006007   0.248074 18.000000  -0.024  0.98095   
## initial.dy:SpeciesS.siderea  0.006007   0.405104 18.000000   0.015  0.98833   
## ---
## Signif. codes:  0 '***' 0.001 '**' 0.01 '*' 0.05 '.' 0.1 ' ' 1
## 
## Correlation of Fixed Effects:
##             (Intr) intl.d SpcsS.
## initial.dy  -0.707              
## SpecisS.sdr -0.577  0.408       
## intl.dy:SS.  0.354 -0.500 -0.612
## optimizer (nloptwrap) convergence code: 0 (OK)
## boundary (singular) fit: see ?isSingular
\end{verbatim}

\begin{Shaded}
\begin{Highlighting}[]
\KeywordTok{anova}\NormalTok{(shuffmod3,}\DataTypeTok{test=}\StringTok{'F'}\NormalTok{)}\CommentTok{# prop d significantly higher in those that started with some d p=0.02456}
\end{Highlighting}
\end{Shaded}

\begin{verbatim}
## Type III Analysis of Variance Table with Satterthwaite's method
##                    Sum Sq Mean Sq NumDF DenDF F value  Pr(>F)  
## initial.d         0.98802 0.98802     1    18  6.0205 0.02456 *
## Species           0.00004 0.00004     1    18  0.0002 0.98833  
## initial.d:Species 0.00004 0.00004     1    18  0.0002 0.98833  
## ---
## Signif. codes:  0 '***' 0.001 '**' 0.01 '*' 0.05 '.' 0.1 ' ' 1
\end{verbatim}

\begin{Shaded}
\begin{Highlighting}[]
\NormalTok{shuffmod4=}\KeywordTok{lmer}\NormalTok{(PropD4}\OperatorTok{~}\NormalTok{initial.d}\OperatorTok{*}\NormalTok{Species}\OperatorTok{+}\NormalTok{(}\DecValTok{1}\OperatorTok{|}\NormalTok{Colony),}\DataTypeTok{data=}\NormalTok{shufftimesofss)}
\KeywordTok{plot_resqq}\NormalTok{(shuffmod4)}
\end{Highlighting}
\end{Shaded}

\includegraphics{SEASONAL_SHUFFLING_files/figure-latex/unnamed-chunk-16-3.pdf}

\begin{Shaded}
\begin{Highlighting}[]
\KeywordTok{summary}\NormalTok{(shuffmod4)}
\end{Highlighting}
\end{Shaded}

\begin{verbatim}
## Linear mixed model fit by REML. t-tests use Satterthwaite's method [
## lmerModLmerTest]
## Formula: PropD4 ~ initial.d * Species + (1 | Colony)
##    Data: shufftimesofss
## 
## REML criterion at convergence: 20.7
## 
## Scaled residuals: 
##      Min       1Q   Median       3Q      Max 
## -1.22228 -0.70611 -0.04998  0.15747  1.72820 
## 
## Random effects:
##  Groups   Name        Variance Std.Dev.
##  Colony   (Intercept) 0.02019  0.1421  
##  Residual             0.11494  0.3390  
## Number of obs: 22, groups:  Colony, 12
## 
## Fixed effects:
##                             Estimate Std. Error       df t value Pr(>|t|)  
## (Intercept)                  0.48417    0.15660 10.12194   3.092   0.0113 *
## initial.dy                   0.46078    0.20573 11.85230   2.240   0.0451 *
## SpeciesS.siderea            -0.10025    0.22646  7.52952  -0.443   0.6704  
## initial.dy:SpeciesS.siderea  0.08997    0.36966 12.24422   0.243   0.8117  
## ---
## Signif. codes:  0 '***' 0.001 '**' 0.01 '*' 0.05 '.' 0.1 ' ' 1
## 
## Correlation of Fixed Effects:
##             (Intr) intl.d SpcsS.
## initial.dy  -0.731              
## SpecisS.sdr -0.692  0.506       
## intl.dy:SS.  0.407 -0.557 -0.601
\end{verbatim}

\begin{Shaded}
\begin{Highlighting}[]
\KeywordTok{anova}\NormalTok{(shuffmod4,}\DataTypeTok{test=}\StringTok{'F'}\NormalTok{) }\CommentTok{# prop d significantly higher in those that started with some d p=0.01777}
\end{Highlighting}
\end{Shaded}

\begin{verbatim}
## Type III Analysis of Variance Table with Satterthwaite's method
##                    Sum Sq Mean Sq NumDF  DenDF F value  Pr(>F)  
## initial.d         0.86064 0.86064     1 12.244  7.4878 0.01777 *
## Species           0.00999 0.00999     1  9.893  0.0869 0.77421  
## initial.d:Species 0.00681 0.00681     1 12.244  0.0592 0.81173  
## ---
## Signif. codes:  0 '***' 0.001 '**' 0.01 '*' 0.05 '.' 0.1 ' ' 1
\end{verbatim}

\begin{Shaded}
\begin{Highlighting}[]
\NormalTok{shufftimesofss=}\StringTok{ }\KeywordTok{pivot_longer}\NormalTok{(}\DataTypeTok{data=}\NormalTok{shufftimesofss, }\DataTypeTok{cols=}\KeywordTok{starts_with}\NormalTok{(}\StringTok{'PropD'}\NormalTok{),}
               \DataTypeTok{names_to=}\StringTok{'Timepoint'}\NormalTok{, }\DataTypeTok{names_prefix=}\StringTok{'PropD'}\NormalTok{, }\DataTypeTok{values_to=}\StringTok{'PropD'}\NormalTok{,}
               \DataTypeTok{values_drop_na=}\NormalTok{T)}

\NormalTok{switchex=}\KeywordTok{expression}\NormalTok{(}\KeywordTok{paste}\NormalTok{(}\StringTok{'no initial '}\NormalTok{, }\KeywordTok{italic}\NormalTok{(}\StringTok{"Durusdinium "}\NormalTok{)))}
\NormalTok{shuffex=}\KeywordTok{expression}\NormalTok{(}\KeywordTok{paste}\NormalTok{(}\StringTok{'>0 initial '}\NormalTok{, }\KeywordTok{italic}\NormalTok{(}\StringTok{"Durusdinium "}\NormalTok{), }\StringTok{'≤ 25%'}\NormalTok{))}

\KeywordTok{ggplot}\NormalTok{(shufftimesofss, }\KeywordTok{aes}\NormalTok{(}\DataTypeTok{x =}\NormalTok{ Timepoint, }\DataTypeTok{y =}\NormalTok{ PropD, }\DataTypeTok{group=}\NormalTok{initial.d)) }\OperatorTok{+}
\StringTok{  }\KeywordTok{stat_summary}\NormalTok{(}\KeywordTok{aes}\NormalTok{(}\DataTypeTok{shape=}\NormalTok{initial.d, }\DataTypeTok{colour=}\NormalTok{initial.d),}\DataTypeTok{fun.data=}\StringTok{'mean_se'}\NormalTok{,}
             \DataTypeTok{position =} \KeywordTok{position_dodge}\NormalTok{(}\FloatTok{0.2}\NormalTok{), }\DataTypeTok{size=}\FloatTok{0.5}\NormalTok{) }\OperatorTok{+}
\StringTok{  }\KeywordTok{scale_x_discrete}\NormalTok{(}\DataTypeTok{labels=}\KeywordTok{c}\NormalTok{(}\StringTok{'Pre heat stress'}\NormalTok{,}\StringTok{'Post heat stress'}\NormalTok{,}\StringTok{'1 month recovery'}\NormalTok{,}\StringTok{'2 month recovery'}\NormalTok{))}\OperatorTok{+}
\StringTok{  }\KeywordTok{scale_shape_manual}\NormalTok{(}\DataTypeTok{values=}\KeywordTok{c}\NormalTok{(}\DecValTok{0}\NormalTok{,}\DecValTok{15}\NormalTok{),}\DataTypeTok{labels=}\KeywordTok{c}\NormalTok{(switchex,shuffex))}\OperatorTok{+}
\StringTok{  }\KeywordTok{scale_colour_manual}\NormalTok{(}\DataTypeTok{values=}\KeywordTok{c}\NormalTok{(}\StringTok{'blue3'}\NormalTok{,}\StringTok{'brown2'}\NormalTok{),}\DataTypeTok{labels=}\KeywordTok{c}\NormalTok{(switchex,shuffex))}\OperatorTok{+}
\StringTok{  }\KeywordTok{stat_summary}\NormalTok{(}\DataTypeTok{geom=}\StringTok{'line'}\NormalTok{, }\KeywordTok{aes}\NormalTok{(}\DataTypeTok{linetype=}\NormalTok{initial.d, }\DataTypeTok{colour=}\NormalTok{initial.d),}\DataTypeTok{position =} \KeywordTok{position_dodge}\NormalTok{(}\FloatTok{0.2}\NormalTok{))}\OperatorTok{+}
\StringTok{  }\KeywordTok{scale_linetype_manual}\NormalTok{(}\DataTypeTok{values=}\KeywordTok{c}\NormalTok{(}\StringTok{'dashed'}\NormalTok{,}\StringTok{'solid'}\NormalTok{),}\DataTypeTok{labels=}\KeywordTok{c}\NormalTok{(switchex,shuffex))}\OperatorTok{+}
\StringTok{  }\KeywordTok{theme_minimal}\NormalTok{(}\DataTypeTok{base_size =} \DecValTok{15}\NormalTok{)}\OperatorTok
\StringTok{   }\KeywordTok{theme}\NormalTok{(}\DataTypeTok{panel.border =} \KeywordTok{element_rect}\NormalTok{(}\DataTypeTok{colour=}\StringTok{'grey20'}\NormalTok{,}\DataTypeTok{fill=}\OtherTok{NA}\NormalTok{),}
          \DataTypeTok{panel.grid.minor.x =} \KeywordTok{element_blank}\NormalTok{(),}
         \DataTypeTok{panel.grid.major.x =} \KeywordTok{element_blank}\NormalTok{(),}
          \DataTypeTok{panel.grid.minor.y =} \KeywordTok{element_blank}\NormalTok{())}\OperatorTok{+}
\StringTok{  }\KeywordTok{labs}\NormalTok{(}\DataTypeTok{y=}\StringTok{"Proportion *Durusdinium* <br /> (*O. faveolata* & *S. siderea*)"}\NormalTok{, }\DataTypeTok{x=}\StringTok{''}\NormalTok{)}\OperatorTok{+}
\StringTok{  }\KeywordTok{theme}\NormalTok{(}\DataTypeTok{legend.position =} \StringTok{'bottom'}\NormalTok{,}\DataTypeTok{legend.title=}\KeywordTok{element_text}\NormalTok{(}\DataTypeTok{size=}\DecValTok{0}\NormalTok{), }\DataTypeTok{axis.title.y =} \KeywordTok{element_markdown}\NormalTok{())}
\end{Highlighting}
\end{Shaded}

\includegraphics{SEASONAL_SHUFFLING_files/figure-latex/unnamed-chunk-16-4.pdf}

\begin{Shaded}
\begin{Highlighting}[]
\CommentTok{#ggsave('ofavshuffswitch.pdf',device='pdf',height=5,width=7)}

\CommentTok{###UPDATE FOR ROSS: I have corrected this analysis to be based on proportion d (mean +-SE), rather than the shuffling metric, given that this is a comparison between those with and without any initial d. 25% initial d was taken as the threshold, so those with no initial d were compared against those with >0 but <25% initial d. Due to small sample size, ofav and ssid are grouped together here (model finds no significant affect of species).  }
\end{Highlighting}
\end{Shaded}

\end{document}
